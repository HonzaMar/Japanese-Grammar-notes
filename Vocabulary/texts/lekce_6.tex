\section{6. lekce}

\subsection*{Seznam použitých slov}

\begin{itemize}
    \item のむ - pít
    \item あう - setkat se
    \item しょくじ - jídlo
    \item ひる - oběd
    \item てんらんかい - výstava
    \item うきよえ - ukiyoe (japonské dřevotisky)
    \item じかん - čas
    \item ばしょ - místo
    \item えき - nádraží
    \item みなみぐち - jižní východ
    \item かいさつぐち - turniket
    \item びじゅつかん - muzeum umění
    \item どこ - kde
    \item どう - jak
    \item どうも - hodně (díky)
    \item よかったら - pokud to tak chceš
    \item いっしょに - spolu
    \item かいもの - nakupování
    \item さんぽ - procházka
    \item こうえん - park
    \item どうぶつえん - zoo
    \item はくぶつかん - muzeum
    \item にしぐち - západní východ
    \item いりぐち - vchod
    \item えいが - film
    \item え - obraz
    \item さくら - sakura (třešňový strom)
    \item ちゅうかりょうり - čínská kuchyně
    \item ごご - odpoledne
    \item ごぜんちゅう - dopoledne
    \item なに - co
    \item きたぐち - severní východ
    \item ひがしぐち - východní východ
    \item こうちゃ - černý čaj
    \item でぐち - východ
    \item あさごはん - snídaně
    \item ひるごはん - oběd
    \item ばんごはん - večeře
\end{itemize}

昨日、私は友達と\textbf{びじゅつかん}で\textbf{うきよえ}の\textbf{てんらんかい}を見に行きました。待ち合わせは\textbf{えき}の\textbf{みなみぐち}でした。友達が少し遅れたので、\textbf{かいさつぐち}で少し待っていました。

展示が始まる前に、私たちは近くのカフェで\textbf{ひるごはん}を食べました。**どうぶつえん**にも行こうかと考えていましたが、時間がなかったので、\textbf{びじゅつかん}にまっすぐ向かいました。

展示の中で一番印象に残ったのは、美しい\textbf{さくら}の\textbf{え}でした。そのあと、**どうも**と友達にお礼を言い、夕方には一緒に**ちゅうかりょうり**を食べに行きました。とても充実した一日でした。

