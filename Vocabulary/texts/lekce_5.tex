\section{5. lekce}

\subsection*{Seznam použitých slov}

\begin{itemize}
    \item あそぶ - hrát si
    \item べんきょう - studovat
    \item けんどう - kendo
    \item しない - bambusový meč
    \item どうぐ - nářadí
    \item みなさん - všichni
    \item よく - často
    \item およぐ - plavat
    \item きく - poslouchat
    \item そうじ - uklízet
    \item れんしゅう - trénovat
    \item じゅうどう - judo
    \item おんがく - hudba
    \item ばんぐみ - televizní program
    \item かいわ - konverzace
    \item へや - pokoj
    \item りょうり - kuchyně
    \item ゆうがた - brzký večer
    \item たいてい - většinou
\end{itemize}

土曜日の午後、私は友達と公園で\textbf{あそぶ}ことにしました。朝には\textbf{べんきょう}をしていたので、午後はリラックスできる時間です。友達のアキラは\textbf{けんどう}をやっていて、彼の\textbf{しない}を見せてくれました。彼は\textbf{どうぐ}の手入れも上手です。

午後になると、\textbf{みなさん}が集まってきました。アキラは**よく**\textbf{じゅうどう}の\textbf{れんしゅう}もしていて、「今日は\textbf{れんしゅう}する?」と聞いてきましたが、私は疲れていたので断りました。

そのあと、私たちは川で\textbf{およぐ}ことにしました。水が冷たくて気持ちよかったです。**ゆうがた**、みんなで私の\textbf{へや}に戻って、\textbf{おんがく}を聴きながらリラックスしました。友達と**かいわ**をしていると、\textbf{ばんぐみ}もつけてみました。とても楽しい一日でした。
