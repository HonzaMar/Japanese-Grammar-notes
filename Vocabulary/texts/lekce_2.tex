\section{2. lekce}

\subsection*{Seznam použitých slov}

\begin{itemize}
    \item ぎゅうどん - rýže s hovězím
    \item てんどん - rýže s tempurou
    \item てんぷら - tempura
    \item ごはん - jídlo/rýže
    \item おかず - jídlo k rýži
    \item ぎゅうにく - hovězí
    \item たこ - chobotnice
    \item とりにく - kuřecí
    \item かつどん - rýže s vepřovým
    \item おちゃ - čaj
    \item うどん - nudle
    \item そば - tenké nudle
    \item ごちそうさまでした - poděkování za jídlo a pití
\end{itemize}

昨日、友達と一緒に日本のレストランに行きました。私は\textbf{ぎゅうどん}を注文して、友達は\textbf{てんどん}を頼みました。\textbf{てんぷら}もとてもおいしそうでした。

私の\textbf{ごはん}には\textbf{ぎゅうにく}が入っていましたが、友達のには\textbf{たこ}と\textbf{とりにく}がたくさんありました。彼はまた\textbf{かつどん}も注文しました。

食事の最後には、\textbf{おちゃ}を飲んで「\textbf{ごちそうさまでした}」と言いました。帰りに、私たちは\textbf{うどん}や\textbf{そば}も食べてみたいと話しました。
