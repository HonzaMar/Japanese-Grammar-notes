\section{3. lekce}

\subsection*{Seznam použitých slov}

\begin{itemize}
    \item くる - přijít
    \item かえる - vrátit se domů
    \item いってくる - jít a vrátit se
    \item でかけてくる - jít ven a zpět
    \item ねる - jít do postele
    \item ふろにはいる - jít do vany
    \item あした - zítra
    \item すぐ - hned
    \item また - znovu
    \item またあした - znovu zítra
    \item おやすみなさい - dobrou noc
    \item いってらっしゃい - jdi a vrať se
    \item でかける - jít ven
    \item いく - jít
    \item おく - vstát, probudit se
    \item うちにいます - zůstat doma
    \item ぎんこう - banka
    \item ゆうびんきょうく - pošta
    \item としょかん - knihovna
    \item しょくどう - jídelna
    \item あさって - pozítří
    \item こんばん - dnes večer
    \item あとで - později
\end{itemize}

\textbf{あした}は土曜日なので、\textbf{うちにいます}。でも、\textbf{あさって}は忙しいです。朝は\textbf{ぎんこう}に行って、それから\textbf{ゆうびんきょうく}と\textbf{としょかん}にも行く予定です。

今日は\textbf{こんばん}、宿題をしないといけませんが、その後\textbf{ふろにはいって}、\textbf{ねます}。夜遅くまで\textbf{でかけない}ので、\textbf{おやすみなさい}と言って、家族は寝ます。

明日の朝、友達が家に\textbf{くる}予定です。「ちょっと\textbf{でかけてくる}から、すぐ戻るよ!」と言って、友達が\textbf{いってらっしゃい}と答えました。私は銀行で用事を済ませた後、\textbf{また}図書館に戻ります。もし時間があれば、\textbf{しょくどう}で昼ご飯を食べます。

