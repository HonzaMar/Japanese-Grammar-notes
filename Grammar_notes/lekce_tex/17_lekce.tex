\section{17 lekce}
\label{sec:lekce_17}

\subsection{ 〜ので/~んで: Protože}
\begin{center}
\begin{tabular}{|c|c|c|}
 \hline
いそがしい &ので/ んで、& 剣道はやっていません。\\
\hline
Planá forma& no/n de&Hlavní klauzule\\
\hline
\end{tabular}
\end{center}
Byl jsem zaneprázdněn takže jsem netrénoval kendo.

1.Sub.klauze končící  ~ので/〜んで vyjadřuje důvod proč něco co je řečeného v hlavní klauzuli. ~ので/〜んで je spojeno s ~のです/~んです ve významu, vyznačující důvod jako vysvětlení pro jistou situaci. Forma která je před  ~ので/ んで  je stejná jako tak která je před  のです/んです tzn., な je použito místo  だ p na- adjectivech a podstatných jménech.  〜んで je ekvivalentní ale není použito v psaném projevu jako  〜ので 

a) いい天気なので、公園に行ってテニスを します。Jdu do parku hrát tenis, jelikož je hezky.

b) きのうはかぜをひいたので、うちで ねていました。Byl jsem včera doma v posteli jekože jsem chytil rýmu.

c) 
A: きょうはうちにいるんですか。 Budeš dneska doma?
B: ええ。 あしたしけんがあるんで、 一日中 べんきょうするんです。Ano, zítra mám zkoušku, takže budu celý den studovat.


2. Když je subjekt sub-klauzule jiný než hlavní klauzule, subjekt sub-klauzule je vždy označen が. Subjekt hlavní klauzule pokud je změněn na téma je označen は. 

Porovnejte:
a) 友だちが来るので、兄は駅へむかえに行きました。

b) 兄は友だちが来るので、駅へむかえに行きました。

\subsection{ 〜たい: Chci dělat ---}
\begin{center}
\begin{tabular}{|c|c|}
\hline
行き& たい\\
\hline
Vstem&tai\\
\hline
\end{tabular}
\end{center}
chci jít

Když. je stem-forma slovesa následována  ~たい, znamená to "chci (dělat) ~." ~たい se skloňuje jako i-adjektivum.
〜たい indikuje mluvčího osobní touhy a přání něco udělat a nemůže být použito pro popis touh ostatních.
a) 
A: 旅行に行かなかったんですか。Nechtěl si jet na výlet?
B: 行きたかったんですけど、母が 入院したんです。Chtěl jsem jet, ale mamka musela do nemocnice.

\subsection{Partikule}
\subsubsection{が : Subjekt (3)}

Subjekt  věty se velmi často  objevuje jako téma věty a je tak často nahrazeno. ALe v některých případech kdy  が nemůže být nahrazeno  は. Když se tázací slovo jako だれ, なに, nebo [どん な + noun] objevuje jako subjekt věty. označeno が, není nikdo změněno na téma. Neboli は nikdy přímo nenásleduje tázací slovo.
a)
A: だれが行ったんですか。Kdo tam šel?
B:わたしが行きました。Já.

b) A: 地下1階には何があるんですか。Co je v prvním suterénu?

B: いろんな店があります。Různé obchody.

\subsubsection{ で: Důvod, příčina}
で indikuje že předcházející podstatné jméno je důvod nebo příčina. Zhruba to koresponduje s "protože" nebo "jelikož"
a) 私はかぜでねていました。Byl jsem v posteli s nachlazením

b)リーさんは病気で学校を休みました。Pan Lee nebyl ve škole kvůli nemoci.

\subsubsection{ を: (Odejít) Z}


を označuje místo ze kterého někdo/něco odejde, pokud je to využito se slovesy jako そつぎょうする "maturovat." でる "jít/přijít ven," たいいんする "být propuštěn z nemocnice," atd.  から "z" nemůže být použito s těmito slovesy místo を.

a)北村さんは 来年大学を そつぎょうします。 Paní Kitamura bude promovat na vysoké příští rok.

b)私はけさ9時にうちを出ました。Dnes ráno jsem odešel z domova v 9 hodin.


\subsubsection{ とか〜 : Věci jako}
とか je partikule pro vyjmenovávání lidí/věcí jako příklady. とか je hlavně používané v mluvené řeči, zatím co や , které taky spojuje podstatná jména pro vyjmenovávání věcí, první partikule může a nemusí být za pposledním podstatným jménem, druhá nemůže.

Porovnejte:
a)
 A: 何を飲んだんですか。Co jsi pil?
 
B: ビールとかワイン(とか)を飲みました。Pil jsem pivo a víno (a podobně).

b) 私はビールやワインを飲みました。Pil jsem pivo a víno (a podobně).


とか také může být použito jako v 〜 とか 〜とかいろいろな/いろんなひと/ もの “různé lidi/ různé věci/jako například ~." V takovém případě, とか přidané po posledním podstatném jméně je nutné.

A:どんな人が来たんですか。Co za lidi přišlo?

B:安田さんとかブラウンさんとかいろんな人が来て(い)ました。Paní Yasuda, pan Brown a různá lidé tam byli.

\subsection{Výrazy indikující čas}
\subsubsection{ Rok, Měsíc a Den}

Následující sufixy jsou používané pro sdělení roku, měsíce a dne.
(1) ねん- je používané pro roky.
(2) ~がつ je používané pro názvy měsíců.
(3) 〜にち přidanépřidáné k číslům Čínského původu jsou používané pro dny v měsíci

“Jaký rok/měsíc/den v měsíci jsou vyjádřené jako  なんねん、なんがつ a なんにち. 

Pokud se říká celé datum celé, rok, měsíc a den jsou psány v tomto pořadí. Rok 1995 je čtený jako せん きゅうひゃく きゅうじゅうご ねん.
1995年1月1日

1. ledna, 1995

\subsection{Délka trvání}
Sufixy 〜ねん (かん)a 〜かげつ(かん)znamenají  "na--let" a "na --měsíců" 〜かん což znamená "trvání."


\subsection{Jiné časové výrazy}
\subsubsection{1)}〜の  とき(に)následující podstatné jméno "(tehdy) když..." 
a) 私は高校の時(に)、日本へ来ました。Přišel jsem do Japonska když jsem byl na střední škole.

b) しけんの 時(に)、 気分がわるくなりました。Udělalo se mi špatně během zkoušky.


\subsubsection{2)}~まえ předcházející slovy indikující trvání času znamená "před ..."

 a)私は2年前(に)大学を そつぎょうしました。Promoval jsem z univerzity před dvěma lety.

\subsubsection{3)} ~ のはじめ a 〜のすえ předcházející slovo "trvání" znamená "začátek-" a "konec ---"

来月のはじめ(に) 友だちがインドから来ます。Můj kamarád přijede na začátku příštího měsíce z Indie.

\subsection{Zdvořilost: Respektující a skromná slovesa}

Některá slovesa mají unikátní ekvivalentní slovesa která jsou používaná pro referenci akce/události nadřízených. Například なさる je respektující ekvivalent する a いらっしゃる je ekvivalentem いる いく くる.

学生: あした試合を見にいらっしゃいますか。Půjdete se podívat zítra na zápas?
先生:ええ、いきます。Ano, půjdu.

Některá slovesa mají skromné ekvivalenty, například もうす je skromný od いう .










