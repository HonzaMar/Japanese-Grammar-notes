
\section{37. lekce}
\label{sec:lekce_37}

\subsection{Pasivní forma sloves}
Pasivní forma sloves je používána když subjekt byl někým ovlivněn akcí někoho jiného. 

1) 1 だん slovesa
Odsuň 〜る a přidej 〜られる
みる → みられる    
ほめる →  ほめられる

2) 5だん slovesa
přidejte -れる k jádru nai-formy

\begin{center}
\begin{tabular}{ccc}
きく → きかれる&たつ → たたれる&たのむ → たのまれる\\
かつぐ → かつがれる&しぬ → しなれる&とる → とられる\\
おす → おされる&よぶ → よばれる&さそう→さそわれる\\
\end{tabular}
\end{center}

3) nepravidelná slovesa
する → される
くる →  こられる

Jakmile je sloveso změněno do pasivní formy, chová se jako 1 だん slovesa



\subsection{Pasivní věty}
Jsou dva typy pasivních vět: direktní a indirektní pasiv.

\subsubsection{Direktní pasiv zmiňující živé bytosti }
\begin{center}
\begin{tabular}{||c|c||c|c||c||}
\hline
学生&は/が&先生&に&ほめられました。\\
\hline
Subjekt&wa/ga&Agent&ni&V pass\\
\hline
\end{tabular}
\end{center}
Student byl pochválen učitelem.

Slovesa používaná v direktním pasivu jsou všechna tranzitivní. Pokud je aktivní věta která 
direktně zahrnuje dvě osoby, tak objekt (osoba) v aktivní větě se transformuje na subjekt v pasivní větě a subjekt (osoba) v aktivní větě je popsán jako agent který dělá tu akci a je vyznačen に

\begin{center}
\begin{tabular}{cc}
Aktivní&Pasivní\\
Aは/がBを〜&Bは/がAに〜\\
ブラウンさんが小林さんをしょうたいした →& 小林さんがブラウンさんにしょうたいされた\\
Pan Brown pozval paní Kobajaši&Paní Kobajaši byla pozvána panem Brownem\\
\end{tabular}
\end{center}




\subsubsection{Indirektní pasiv}
Indirektní pasiv je používán na popis situace ve které je někdo nepřímo ovlivněn akcí někoho jiného. Subjekt indirektního pasivu je vždy životný (osoba). Narozdíl od direktního pasivu, oboje jak tranzitivní nebo intrazitivní slovesa se dají použít.

\paragraph{Tranzitivní slovesa}
\begin{center}
\begin{tabular}{||c|c||c|c||c||c||c||}
\hline
原さん&は/が&だれか&に&かさ&を&撮られました。\\
\hline
Subjekt&wa/ga&Agent&ni&objekt&o&V pass\\
\hline
\end{tabular}
\end{center}
Paní Haře byl někým ukraden deštník.

Pokud popisujeme situaci kdy někomu byli ukradeny/rozbity/...atd. věci. Osoba která dané věci vlastní se stává subjektem pasivní věci, indikující že subjekt je nepřímo ovlivněn něčím rozbitím, ukradením. Objekt aktivní věty (neživá věc) se neobjevuje jako subjekt v pasivní větě v tomto případě, ale zůstává objektem.

\begin{center}
\begin{tabular}{cc}
Aktivní&Pasivní\\
Aは/がBのNを〜&Bは/がAにNを〜\\
だれかがかわいさんのさいふをぬすんだ →&かわいさんはだれかにさいふをぬすまれた\\
někdo ukradl Panu Kawaiovi peněženku&Panu Kawaiovi byla někým ukradena peněženka\\
\end{tabular}
\end{center}

\paragraph{Intranzitivní slovesa}
Pasivní věty s intranzitivními slovesy jako くる "přijít" (雨が)ふる "pršet" なく"brečet" 死ぬ"zemřít" indikují že subjektovaná osoba není nadšená z nepříjemných zkušeností způsobené akcí někoho jiného.

よるおそくともだちがきた  →     私はよるおそくともだちにこられた
kamarád mě navštívil pozdě večer       →      Byl jsem nepříjemně zaskočen návštěvou kamaráda

\subsubsection{Direktní pasivum s neživými věcmi}
Jak bylo vysvětleno v předchozích částech, pasivní věty jsou často spojené s živými věcmi.. Agent té akce většinou není zmíněn.

a) 1964年に東京でオリンピックがひらかれました。 Olympijské hry byly pořádány v Tokiu v roce 1964

\subsection{〜たら: Když -- (tak jsem zjistil že --)}
\begin{center}
\begin{tabular}{||c||c||}
\hline
山口さんに聞いたら&説明してくれました。\\
\hline
V tara&Main Clause\\
\hline
Sub-clause&\\
\hline
\end{tabular}
\end{center}
Když jsem se zeptal pana Jamaguciho, vysvětlil mi to.

Kondicionál a časové použití tara-klauzule bylo vysvětleno v předchozích lekcích. Pokud je sloveso v hlavní klauzuli v minulé formě 〜たら může znamenat  "když jsem udělal--, něco z toho vyplynulo" 

a) 橋本さんにたのんだら、ビデオカメラをかしてくれました。Když jsem se zeptal paní Hashimoto, půjčila mi její kameru.

\subsection{〜たばかりだ: Právě jsem udělal --}
\begin{center}
\begin{tabular}{||c|c||}
\hline
買った&ばかりです。\\
\hline
V ta&bakari da\\
\hline
\end{tabular}
\end{center}
Právě jsem to koupil

Když je ta-forma slovesa následována  たばかりだ, znamená to že daná akce/událost se udála chvíli před tím. Což znamená že referovaná osoba to právě udělala. "Před chvílí" není objektivní časový úsek ale subjektivní pocit.

a) 今の仕事は始めたばかりなので、おもしろいかどうかまだわからない。Jelikož jsem nastoupil do této práce nedávno, Nevím zda je zajímavá nebo ne.

\subsection{〜ようにいう: říct někomu aby udělal ---}

\begin{center}
\begin{tabular}{||c|c||c|c||c||}
\hline
11時に来る&ように&南さん&に&言いました。\\
\hline
V plain&yooni&Osoba&ni&iu\\
\hline
\end{tabular}
\end{center}
Řekl jsem panu Minami aby příšel v 11 hodin. 

〜ようにいう znamená "říct někomu aby udělal --". Před ním je vždy přítomná plain forma (jak kladná nebo záporná) slovesa. Slovesa jiná než いう jako například たのむ "požádat o laskavost" nebo  ちゅういする "varovat" つたえる "předat zprávu"

a) そのことはだれにも話さないように頼みました。Požádal jsem ho aby o tom nikomu neříkal.



\subsection{Vyjádření pocitů}
Výrazy na vyjádření pocitů jsou

うれしい šťastný
はずかしい být v rozpacích
びっくりした být překvapený
こまった mít problémy
かなしい  smutný
くるしい trpící
がっかりした být zklamaný

Tyto adjektiva  a slovesa jsou používány pouze na vyjádření pocitu mluvčího. Abychom vyjádřili jak se cítí někdo jiný přidáme 〜そうだ、〜らしい、〜ようだ、〜といっていた jako v うれしいそうだ "slyšel jsem že je šťastný/á" nebo  こまっているようだ "vypadá že má problémy"

slovesa berou partikuli に na vyjádření důvodu


