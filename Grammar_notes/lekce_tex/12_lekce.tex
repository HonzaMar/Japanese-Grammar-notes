\section{12. lekce}
\label{sec:lekce_12}

\subsection{Nominální věty: Minulost}
Minulá pozitivní forma 〜です v (Noun+ です) je 〜でした. Negativní minulá forma je 〜では/じゃ ありませんでした. 〜じゃ je zkrácená forma  では a je používaná pouze v mluvě.

\begin{center}
\begin{tabular}{|c|c|c|}
\hline
&pozitivní & negativní\\
\hline
Přítomnost&やすみです& やすみでは/じゃ ありません/ないです\\
\hline
Minulost&やすみでした& やすみ では/じゃ ありませんでした/なかったです。\\
\hline
\end{tabular}
\end{center}

a) きのうはいい天気でした。 でも、きょうはいい天気ではありませんでした。Včera bylo hezky. Dneska už hezky nebylo.

\subsection{Věty s adjektivy: minulost}
\subsubsection{I-adjektiva}
Minulá pozitivní respektive negativní forma i-adjektiv je vytvořena přidáním 〜かった respektive 〜くなかった k stem-formě.  いい "dobrý" se časuje nepravidelně na  pozitivní よかった a negativní よくなかった.

\begin{center}
\begin{tabular}{|c|c|c|}
\hline
&pozitivní & negativní\\
\hline
Přítomnost &あついです& あつくありません/あつくないです\\
\hline
Minulost &あつかったです& あつくありませんでした/あつくなかったです\\
\hline
\end{tabular}
\end{center}

a) きのう 新宿へえいがをみに行きました。えいがはとてもおもしろかったです。Včera jsem jel do Shinjuku na film. Film byl velmi zajímavý.

\subsubsection{Na-adjektiva}
Pozitivní a negativní minulá forma です v (na-adj + です) je stejná jako u (noun + です)

\begin{center}
\begin{tabular}{|c|c|c|}
\hline
&pozitivní & negativní\\
\hline
Přítomnost &しずかです& しずかでは/じゃ ありません/ないです\\
\hline
Minulost&しずかでした& しずか では/じゃ ありませんでした/なかったです\\
\hline
\end{tabular}
\end{center}

a) としょかんの中はとてもしずかでした。V knihovně je opravdu ticho.

\subsection{Příslovce rozsahu}
すこし "trochu, málo" ちょっと "trochu" だいたい "většinově" たくさん "hodně" おおぜい "mnoho lidí" とても "velmi" すごく "velmi" atd. jsou příslovce která vyjadřují rozsah.

a) ゆうべすこし雪がふりました。Včera večer trošku sněžilo.

\subsection{〜が/〜けど: Ačkoli}

\begin{center}
\begin{tabular}{|c|c|c|}
\hline
いい天気でした& が/けど& さむかったです。\\
\hline
&ga/kedo&\\
Sub&klauzule&Hlavní klauzule\\
\hline
\end{tabular}
\end{center}
Ačkoli bylo hezky, byla zima.

Obojí 〜が a 〜けど můžou být použity k významu "ale" nebo "ačkoli" když se dají na konec sub-klauzule. 〜が je hlavně ve formální řeči a psané formě. 〜けど je zase primárně používané v mluvě.

a) デパートへ行きましたが、きょうは休みでした。Šel jsem do nákupního centra, ale měli zavřeno.

〜が a 〜けど jsou často používané pro spojení dvou klauzulí a nemusí nutně znamenat "ale".

A: いま、12時ですけど、おひる、食べに行きませんか。Je dvanáct hodin, nepůjdeme na oběd?

B: あ、いいですね。行きましょう。Áá, to je dobrý nápad. Pojďme. 

Důležité je že je rozdíl mezi  〜が/〜けど a   でも.  〜が a  〜けど se obejvují na konci klauzule, která je následována další klauzulí a neobjevují se na začátku věty. Na druhou stranu でも se vždy objevuje na začátku věty a nemůže být použito na místech kde se používají  〜が a  〜けど.

Porovnej: 

a) まいにち5時間べんきょうしましたが、せいせきはよくありませんでした。Přestože jsem studoval 5 hodin denně, moje známky byly špatné.

b) まいにち5時間べんきょうしました。でも、せいせきはよくありませんでした。Studoval jsem 5 hodin denně. Moje známky ale byly špatné.

\subsection{Partikule}
\subsubsection{は: kontrast}
 
は, která označuje téma věty, také může indikovat kontrast. Při porovnávání lidí, věcí, časů, míst atd. は je přidáno po obou srovnávaných elementech.

a) 父は中国人で、母は日本人です。Můj otec je z Číny a matka z Japonska.

\subsubsection{N1もN2も: Obojí N1 a N2 }
Pokud je partikule も použita po dvou různých podstatných jménech v pozitivní větě, znamená to "N1 stejně jako N2" 

a) スパゲテイもピザもちゅうもんしました。Objednal jsem si špagety a stejně tak i pizzu.


\subsubsection{に: Destinace} 
Partikule に je používána pro označení "destinace" nějaké akce (vůči nějakému místu) a je používaná se slovesy jako  のぼります "lézt"  a はいります "vstoupit"

先週の日曜日にともだちと山にのぼりました。 Minulou něděli jsme s kamarádem vylezli na horu.