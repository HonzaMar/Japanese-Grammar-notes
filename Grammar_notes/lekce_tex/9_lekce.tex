\section{9. lekce}
\label{sec:lekce_9}

\subsection{I-adjektiva a Na-adjektiva}

Jdou dva druhy adjektiv (přídavných jmen) v japonštině. Jeden typ, končící na 〜い jako například ちかい "blízký" a とおい "vzdálený" jsou nazývána i-adjektiva. Dalším typem jsou adjektiva zvaná na-adjektiva. Na-adjektiva nekončí na  〜い jako しずか "tichý" にぎやか "rušný" たいへん "složitý", přesto jsou ale výjimky jako  きれい "hezký/čistý".

\subsection{Věty s adjektivy}
\subsubsection{Věty s I-adjektivy}
\begin{center}
\begin{tabular}{|c|c|c|c|}
\hline
えき& は &ちかい& です。\\
\hline
Noun&wa&I-adj&desu\\
\hline
\end{tabular}
\end{center}
Stanice je blízko. (Doslova: "Stanice je blízká")

\vspace{1 cm}
\begin{center}
\begin{tabular}{|c|c|c|c|}
\hline
えき& は& ちかく& ありません/ないです。\\
\hline
Noun&wa&I-adj-ku&arimasen/naidesu\\
\hline
\end{tabular}
\end{center}
Stanice není blízko.

1)

Adjektiva v japonštině mění tvar. Část před -い je nazývána stem-forma, která se nikdy nemění. Stem-forma následována く je nazývána ku-forma. Negativum i-adjektiva je vytvořeno přidáním  〜ない ke ku-formě. Tedy negativ  ちかい je  ちかくない.

いい "dobrý" je nepravidelné adjektivum, kdy negativum je よくない.


2) I-adjektiva jako predikát

I-adjektiva mohou být použita jako predikát ve větě přidáním 〜です. Neminulé věty se vytváří přidáním 〜です ke kladné formě nebo negativní formě i-adjektiva, jako v  ちかいです "je blízko(ý)" nebo ちかくないです "není blízko(ý) ありません také může být použito místo ないです  jako  v ちかくありません což je více formální.

a) 大学のとしょかんは新しいです。 Univerzitní knihovna je nová

\subsubsection{Věty s Na-adjektivy}
\begin{center}
\begin{tabular}{|c|c|c|c|}
\hline
こうえん& は& しずか& です。\\
\hline
Noun&wa&Na-adj.&desu\\
\hline
\end{tabular}
\end{center}
Park je tichý


\begin{center}
\begin{tabular}{|c|c|c|c|}
\hline
こうえん& は& しずか& ではありません/じゃありません。\\
\hline
Noun&wa&Na-adj.&dewa/ja arimasen\\
\hline
\end{tabular}
\end{center}
Park není tichý.

Narozdíl od i-adjektiv, na-adjektiva se chovají jako podstatná jména když se používají jako predikáty a musí být následovány 〜です pokud se vytváří ne-minulá pozitivní věty. Negativ od  〜です je 〜ではりません nebo じゃありません.

a) 新宿はいつもにぎやかです。Shinjuku je vždy rušné.



〜ではないです nebo  〜じゃないです také může být využito místo  (na-adj/noun +  では/じゃありません). 〜ではありません je více formální než ostatní formální negativní formy a je často použito v psaném projevu. 

\subsubsection{Příslovce modifikující přídavná jména}
とても a  そんなに jsou oboje příslovce modifikující přídavná jména.  とても "velmi" je používáno v kladných větách, zatím co そんなに "ne tolik" je používáno v negativních větách.

a) 私のアパートはとても古いです。Můj byt je velmi starý.



\subsection{Spojování vět s adjektivy}
\begin{center}
\begin{tabular}{|c|c|}
\hline
安くて、& おいしいです。\\
\hline
I-adj-kute&Adjektivum\\
\hline
\end{tabular}
\end{center}
levný a výborný


\begin{center}
\begin{tabular}{|c|c|c|}
\hline
きれい& で、& しずかです。\\
\hline
Na-adj & de&Adjektivum\\
\hline
\end{tabular}
\end{center}
hezký a tichý

Při spojování dvou a více adjektiv za sebou, všechny kromě posledního se změní do te-formy . Ta je udělána přidáním 〜くて ke stem-formě. ( いい se změní na よくて).  〜で je te-forma  〜です v (na-adj +  です). 


a) この店は高くて、まずいです。Jídlo v této restauraci je drahé a není dobré.

\subsection{Adjektiva jako modifikátory podstatných jmen}
\begin{center}
\begin{tabular}{|c|c|}
\hline
おいしい& スパゲテイ\\
\hline
I-adj&Noun\\
\hline
\end{tabular}
\end{center}
chutné špagety


\begin{center}
\begin{tabular}{|c|c|c|}
\hline
しずか &な& へや\\
\hline
\end{tabular}
\end{center}
tichá místnost

I-adjektiva modifikují podstatná jména které hned přecházejí. Na-adjektiva také modifikují podstatná jména, ale 〜な musí být přidáno.

a) きのうこうえんで大きいいぬをみました。Včera jsem viděl velkého psa v parku.



\subsection{の nahrazující podstatné jméno}
\begin{center}
\begin{tabular}{|c|c|}
\hline
しろい& の\\
\hline
I-adj&no\\
\hline
\end{tabular}
\end{center}
ten bílý


\begin{center}  
\begin{tabular}{|c|c|c|}
\hline
きれい &な &の\\
\hline
Na-adj. & na&no\\
\hline
\end{tabular}
\end{center}
ten hezký

の může nahradit podstatné jméno které je modifikováno adjektivem pokud je pochopeno z kontextu. 


Porovnejte:

a) しろいばらはいっぽん300円です。Bíle růže stojí 300 jenů jedna.

b) しろいのは いっぽん300円です。Ty bílé stojí 300 jenů jedna.


\subsection{Spojení でも: Ale}

でも je spojení znamenající "ale" a je použito pro spojení dvou vět.

a) スポーツセンターのプールは新しいです。でもそんなに大きくないです。 Bazén ve sportovním centru je nový. Ale není moc velký.











