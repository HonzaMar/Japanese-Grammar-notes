
\section{33 lekce}
\label{sec:lekce_33}

\subsection{〜ことに する: Rozhodnout že ---}
\begin{center}
\begin{tabular}{||c|c||}
\hline
だいがくいんにいく & ことにしました。\\
\hline
V plain (non-past)&koto ni suru\\
\hline
\end{tabular}
\end{center}
Rozhodl jsem se že půjdu na magisterské studium.

〜ことに する znamená "(někdo) rozhodl se udělat/neudělat --" často je používané v minulé formě jako 〜ことにした Před tím je vždy plain ne-minulá forma slovesa.

a)私は会社をやめて、ブラジルに留学することにしました。  Rozhodl jsem se že dám výpověď a budu studovat v Brazílii


\subsection{〜ことに なる: Bylo rozhodnuto že ---}
\begin{center}
\begin{tabular}{||c|c||}
\hline
六時にあつまる &ことになりました。\\
\hline
V plain (non-past) & koto ni naru\\
\hline
\end{tabular}
\end{center}
Bylo rozhodnuto že se sejdeme v 6 hodin.

〜ことに なる znamená "je rozhodnuto že/ zařízeno tak že". Je to velmi často používáno v minulé formě jako 〜ことに なった "bylo rozhodnuto/ zařízeno že". Před tím je vždy plain ne-minulá forma slovesa.

a) 上田さんは今度てんきんすることになりました。Bylo rozhodnuto že pan Ueda bude brzo přesunut.



\subsection{〜あと: Po --- (nějaké činnosti)}
\begin{center}
\begin{tabular}{||c|c||c||}
\hline
カナダに帰った &あと、& しゅうしょくします。\\
\hline
V ta&ato&\\
Sub-clause&&Main clause (past)\\
\hline
\end{tabular}
\end{center}
Seženu si práci po tom co se vrátím do Kanady.

あと které následuje sloveso v ta-formě znamená "Poté co---". Minulá forma slovesa je před あと i když popisujeme akci která se stane až v budoucnosti. 

Porovnej: 

a) 私は大学をそつぎょうしたあと、高校のきょうしになります。Stanu se středoškolským učitelem až odmaturuji.

b) 私は大学を卒業した後、高校の教師になりました。Stal jsem se středoškolským učitelem po tom co jsem odmaturoval.

\subsection{〜とき: Když --}
\begin{center}
\begin{tabular}{||c|c||c||}
\hline
けさ大学に来る &時(に)& 森さんに会いました。\\
\hline
V dict.&toki(ni)&\\
Sub-clause&&Main clause (past)\\
\hline
\end{tabular}
\end{center}
Viděl jsem Pana Moriho když jsem dnes ráno přicházel na univerzitu.



Porovnej:

a)パリに行く時、このさいふを買った。 Koupil jsem si tuhle peněženku když jsem jel do Paříže (tzn. Koupil jsem ji před tím než jsem odjel do Paříže)

b)パリに行った時、この財布を買った。 Koupil jsem si tuhle peněženku když jsem jel do Paříže (tzn. Koupil jsem ji v Paříži)

\subsection{Partikulární fráze 〜について: Ohledně/o}
〜について znamená "Ohledně/o/týkající se/odkazující se" a používá se se slovesy jako かく"psát" はなす"mluvit" しっている "vědět" しらべる "vyhledávat" べんきょうする "studovat" けんきゅうする"zkoumat"

日米関係(にちべいかんけい)についてレポートを書いた。Psal jsem report o vztazích Japonska a USA.

\subsection{Honorifikace (3): Zdvořilá forma sloves}

Mnoho sloves může být vloženo do zdvořilé formy použitím struktury [ お + V stem + に なる]. Tato forma se používá pro referenci činností nadřízeného. Nevztahuje se na slovesa, která mají svůj speciální tvar

かえる →  おかえりになる vrátit se
出かける →  お出かけになる jít ven
まつ →  お待ちになる čekat

V případě sloves typu suru-nouns je struktura [(ご) + N(+suru) + なさる ]

れんらくする →   (ご)れんらくなさる kontaktovat
きこくする → (ご)きこくなさる vrátit se do vlasti

Existuje pár vyjímek, například (お)でんわなさる kde místo  ご je お. 

