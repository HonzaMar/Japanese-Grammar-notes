\section{40. Lekce}
\label{sec:lekce_40}

\subsection{ ~ように: Tak aby ~}
\begin{center}
\begin{tabular}{|c|c|c|}
\hline
よくわかる&ように&説明しました。\\
\hline
V plain (přítomný)&yoo ni&sloveso\\
\hline
\end{tabular}
\end{center}
Vysvětlil jsem to tak aby to bylo pochopitelné.


~ように  následující za planou přítomnou formou slovesa znamená "(dělat něco) tak aby ---." ~ように je často předcházeno potenciální formou nebo negativní formou sloves, nebo slovesa jako  わかる “rozumět," みえる “viditelné," atd.
a) 2年後に留学(りゅうがく)できるように、貯金(ちょきん)しています。Šetřím abych mohl 2 roky studovat v zahraničí.

b) 道に迷(まよ)わないように、地図を持っていった。Vzal jsem mapu abych se neztratil.

\subsection{ ~ようにする: Snažit se aby ~ /Udělat tak aby -- }
\begin{center}
\begin{tabular}{|c|c|}
\hline
時間(じかん)がかからない& ようにしました。\\
\hline
V plain (přítomný) & yoo ni suru\\
\hline
\end{tabular}
\end{center}
Ujistil jsem se že to nezabere moc času.

~ようにする následující za planou přítomnou formou slovesa indikuje "Snažit se aby ~ /Udělat tak aby --." Když se する objeví ve formě している jako v 〜ようにしている, znamená to "zařídit to aby--."
a) 人と会う時、必( かなら)ず約束( やくそく)の時間(じかん)を守(まも)るようにしています。Dal jsem si pravidlo abych byl vždy dochvilný když se mám s někým potkat.

 
b) 旅行中(りょこうちゅう)はなるべく生水(なまみず)を飲まないようにした。Snažím se nepít nepřevařenou vodu jak jen to jde během výletu.


\subsection{ 〜ことがある : Jsou časy kdy ~}
\begin{center}
\begin{tabular}{|c|c|}
\hline
たまに雪が降る&ことがあります。 \\
\hline
Planá forma (přítomná)&koto ga aru\\
\hline
\end{tabular}
\end{center}
Občas prší.


〜ことがある následující přítomnou planou formu slovesa, i-adjektiva nebo  [na-adj./noun + だ ] znamená “Jsou časy kdy ~” nebo “Někdy se stává že ~.”だ po na-adjektivu a podstatném jménu se změní na な  respektive の. 

a) 帰りの電車はたいていすいているが、 たまに 座(すわ)れないことがある。Čas kterým jezdím domů většinou není vlak moc plný, ale někdy nelze sehnat místo.

b) こづかいはふつう2万円ぐらいだが、月によって3万円以上のこともある。Moje kapesné je většinou 20 000 jenů, ale jsou měsíce kdy je to více 30 000 jenů.



Tento výraz by neměl být zaměňován  s  ~ことがある což následuje ta-formu slovesa.
\subsection{ Zvýraznění části věty (2)}

Věty které zvýrazňují důvod jako v "je to protože --" může být vyjádřeno použitím  [〜のは〜からだ]. Tato forma je v podstatě stejná jako  [〜のは~だ] (Lekce 35). Hlavní klauzule je umístěna před 〜のは, a kara-klauzule, část která je zvýrazňována, je položena před ~だ. ~のは je předcházeno planou formou (Lekce 16) a な by mělo být přidáno po na-adjektivu nebo podstatném jménu.

道がこんでいたから、時間がかかりました –> 時間がかかったのは道がこんでいたからです
zabralo to čas, protože byl provoz. –> protože byl provoz, zabralo to čas


Zvýraznění účelu jako v "je to tak ---aby---" může být vyjádřeno  [〜のは〜ため だ].  に v 〜ために je v tomto případě vynecháno.

車を買うために、お金をためています –> お金をためているのは車を買うためです
šetřím peníze abych si koupil auto  –> abych si koupil auto, tak šetřím


\subsection{ Množstevní slovo modifikující podstatné jméno}
\begin{center}
\begin{tabular}{|c|c|c|}
\hline
多く&の&資料(しりょう)\\
\hline
Množství&no&Podstatné jméno\\
\hline
\end{tabular}
\end{center}
mnoho dokumentů



おおく, což znamená "hodně," vyžaduje použití の pokud se nachází před podstatným jménem. 

日本では、多くの大学生が親に学費を出して もらっているようである。V Japonsku, jak to tak vypadá, je většina školného na univerzitách placená rodiči.


Množstevní slova jako おおぜい “mnoho (lidí),” かなり “docela dost,” ほとんど “skoro vše" a 100にん "sto lidí" nemůže modifikovat podstatné jméno přímo ale musí být přidáno の  hned po množstevní slově pokud předchází podstatné jméno.

a) おおぜいの人が調査( ちょうさ)に協力(きょうりょく)してくれました。Mnoho lidí se účastnilo průzkumu.


b) アジアのほとんどの国がオリンピックに 参加(さんか)した。Skoro všechny země v Asii se účastnilo Olympijských her.

\subsection{ 〜だけでなく (~も): Nejen}


~だけでなく(~も) je používané pro vyjádření "nejen--." 〜も se často objevuje ve frázi za 〜だけでなく.
a) 図書館(としょかん)だけでなく、市役所(しやくしょ )にも調(しら)べに行きました。Nejen že jsem šel do knihovny ale i na městský úřad se podívat.


b) アンケート用紙(ようし)を配(くば)っただけでなく、多くの人にインタビュー調査(ちょうさ)をした。Nejen že jsme rozdali dotazníky ale také jsme měli interview s mnoha lidmi. 


~だけでなく je hlavně používané ve formální řeči či psaném projevu a jeho hovorový ekvivalent používané jen v mluvené řeči je, 〜だけじゃなくて, 
\subsection{ Příslovečné fráze}
Následuje několik příkladů příslovečných frází.

~と  違(ちが)って    jiný

~と 比(くら)べて v porovnání s

~を  対象(たいしょう)に   jako objekt


2、30年前と違(ちが)って、最近(さいきん)は日本で働( はたら)く外国人(がいこくじん)が増えている。Narozdíl od doby před 20 až 30 lety se počet cizinců pracujících v Japonsku zvedá.
























