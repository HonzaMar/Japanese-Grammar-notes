V případě nalezení jakýchkoliv chyb mi prosím napište na janmarjanko@seznam.cz

\paragraph{Poznámky pro čtenáře}
Tento dokument je amatérský překlad gramatických poznámek učebnice Total Japanese (Okano Kimiko a spol.) do Češtiny. Nejsem profesionální překladatel a proto nemůžu ručit za úplnou správnost překladu. Pokud naleznete chybu v překladu, prosím obraťte se na výše zmíněný email. Během překladu jsem některé anglické výrazy zanechal buď kvůli nejistotě v překladu a nebo z praktičnosti ("noun" je kratší než "podstatné jméno.") V prvních kapitolách se skoro vůbec nevyskytují složité  znaky kanji. To ale není konzistentí. Furigana je v některých případech napsaná v závorce ale to také není úplně všude. Tyto všechny věci budu postupně opravovat a vylepšovat, viz. dašlí odstavec o stavu dokumentu.

\paragraph{Aktuální stav dokumentu}

Hotovo:
\begin{enumerate}
\item vycentrované tabulky
\item kontrola spellingu pomocí editoru
\item připravené odkazy na jednotlivé lekce
\end{enumerate}

To Do:
\begin{enumerate}
\item Doplnit odstavce "Ostatní" (v Aj "Miscellaneous") do některých lekcí
\item Doplnit odkazy na ostatní lekce v textu
\item Přidat furiganu (závorky)
\item Furigana (nad/pod kanji)
\item Přepsat co nejvíce slov na kanji s furiganou
\end{enumerate}

Dále bych chtěl v budoucnosti sepsat konverzační poznámky, seznam slovíček, přehled všech kanji v učebnici (nebojte, napíšu si na to program), kanji z RTK (Remembering the Kanji, James W. Heisig). Navíc se poohlížím po možném audio zpracování poznámek. Toto vše bych samozřejmě dal ke mě na github.

Všechny dokumenty jsou psané pomocí \LaTeX  a dávány na github. Pokud vládnete obojím, není problém dané dokumenty rovnou upravovat a jen požádat o Push. 