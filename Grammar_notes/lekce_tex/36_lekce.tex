\section{36. lekce}
\label{sec:lekce_36}

\subsection{〜たら: Časové (když--)}

\begin{center}
\begin{tabular}{||c||c||}
\hline
家に帰ったら、&でんわします。\\
\hline
V tara&Hlavní klauzule\\
\hline
Sub-klauzule&\\
\hline
\end{tabular}
\end{center}

Až přijdu domů, tak ti zavolám.

Jak bylo představeno ve 29 lekci, tara-klauzule vyjadřuje kondicionální význam  "pokud". Ale pokud tara-klauzule se slovesem vyjadřuje podmínku která definitivně bude splněna, znamená to "když/až" indikující "až tato akce bude hotová tak nastane nová akce"

a) 大学をそつぎょうしたら、中国にりゅうがくすることになりました。Bylo zařízeno že pojedu studovat do číny až odpromuji na vysoké škole.


\subsection{〜のだったら/〜んだったら  Pokud se jedná o tohle tak --}

〜のだったら/〜んだったら znamená "pokud se jedná o tohle --" To co často toto předchází je věc kterou mluvčí právě zaslechl a může to být použito jako základ mluvčího názoru/ radě/ návrhu nebo požadavku. což je vyjádřeno v hlavní klauzuli.


A: こんどバソコンを買おうと思っているんです。Přemýšlím nad koupí počítače brzo.

B:パソコンを買うんだったら、どれがいいかいろんな人に聞いたほうがいいですよ。Pokud chceš kupovat nový počítač, měl by ses zeptat více lidí který je nejlepší.

\subsection{Imperativní forma}
Slovesa v imperativní formě mohou být použity v citačních frázích, nebo mužskými mluvčími na konci věty jako obyčejná úroveň mluvy. 

\subsubsection{kladná}
1) 1 だん slovesa
drop 〜る a přidej  〜ろ
\begin{center}
\begin{tabular}{cc}
みる → みろ & ねる → ねろ\\
おきる → おきろ &  食べる → たべろ\\
\end{tabular}
\end{center}


2) 5だん slovesa

Vyměň konečné うーだん hiraganu ze slovníkové formy za  えーだん hiraganu.
\begin{center}
\begin{tabular}{ccc}
かく → かけ &まつ → まて &のむ → のめ\\
およぐ → およげ & しぬ → しね &とる → とれ\\
はなす → はなせ & あそぶ → あそべ &かう → かえ\\
\end{tabular}
\end{center}


3) Nepravidelná slovesa

する → しろ  くる → こい



\subsubsection{záporná}
Negativní imperativní forma je vytvořena přidáním な ke slovníkové formě. Tato forma je často používána v přísných zákazových cedulích.
はいるなnevstupujte
捨てるな nezanechávejte odpadky

\subsection{〜しろ/なという: Rozkaz/říct někomu aby udělal ---/neudělal ---}
\begin{center}
\begin{tabular}{||c|c||c||}
\hline
会社に来い&と&言いました。\\
\hline
V imp&to&iu\\
\hline
\end{tabular}
\end{center}

Imperativní forma slovesa je často používána v citovacích frázích a v tomto případě nemusí nutně nést imperativní význam. V nepřímých citacích 〜しろ a 〜するな může být použito ve významu 〜してください/〜しなさい a 〜しないでください/〜してはいけない 

a) 安田さんはやくそくをまもれと言いました。 Paní Jasuda mi řekla abych dodržel slib.

\subsection{〜てもらいたい: Chtít aby někdo udělal ---}

když 〜てもらう je použité v --たい formě jako v 〜てもらいたい to znamená že mluvčí chce po někom aby něco udělal nebo aby bylo něco uděláno. místo て může být použito  で

a) 私はいしゃに元気のけっかを知らせてもらいたいと言いました。Řekl jsem doktorovi že chci aby mi dal vědět výsledky testů.

\subsection{Honorifikace}

\subsubsection{お〜ください: Rozkaz s respektem}
(お+V stem+ください) je formální ekvivalent 〜てください a je používán při dávání instrukcí nebo při slušném naléhání na někoho. Výrazy s tímto vzorem se často používají v oznámeních v obchodech, vlacích atd. Také mohou být na cedulích a značkách.

まつ  →   おまちください.    prosím počkejte

はいる  →  お入りください.  prosím vstupte

pro slovesa typu noun+suru se místo お používá ご, kromě několika vyjímek, například でんわする → お電話ください prosím zavolejte mu/jí


れんらくする → ごれんらくください prosím kontaktuje ho/jí

\subsubsection{お〜いただけますか: Pokorný požadavek}
( お + V stem + いただけますか) je více formální ekvivalent k 〜ていただけますか a je používán k více slušenému požadavku.

待っていただけますか →  お待ちいただけますか počkáte prosím?

伝えていただけますか → お伝えいただけますか Řekl byste mu/jí to prosím?

pro slovesa typu noun + suru se používá místo お  přepona ご
れんらくしていてだけますか →   ごれんらくいただけますか kontaktuješ ho/jí prosím?
