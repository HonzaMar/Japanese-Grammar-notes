
\section{32. lekce}
\label{sec:lekce_32}

\subsection{と: pokud/když}
\begin{center}
\begin{tabular}{||c|c||c||}
\hline
かどを曲がる&と、&銀行があります。\\
\hline
V dict.& to& Hlavní klauzule\\
\hline
Sub-klauzule&&\\
\hline
\end{tabular}
\end{center}
Pokud támhle zahneš za roh, tam je banka.

Klauzule končící na 〜と znamená "pokud --" "když--" nebo "kdykoli když --" a celá věta indikuje že "Když akce nebo událost nastane, něco dalšího nevyhnutelného/přirozeného nastane" 〜と následuje slovníkovou formu slovesa.

a) 4月になると、さくらがさきはじめます。Až přijde duben, začnou kvést sakury.

\subsection{Podstatné jméno +  だったら :Pokud se jedné o --- }

Jako (Podstatné jméno + なら), tak i (Podstatné jméno + だったら) může být použito na zvýraznění tématu. "Pokud je to o tomto ---"  (Podstatné jméno + なら) a (Podstatné jméno + だったら) jsou navzájem záměnné.

a) 
A:このへんにゆうびんきょく、ありますか。 Je tu někde okolo pošta?

B:ゆうびんきょくだったら、駅の近くにありますよ。  Pokud jde o poštu, tak ta je blízko stanice.






\subsection{Partikule を: Podél}
\begin{center}
\begin{tabular}{||c|c||c||}
\hline
とおり&を&あるきました。\\
\hline
Místo& o& V pohybu\\
\hline
\end{tabular}
\end{center}
Šel jsem podél této ulice.

を označuje místo kudy někdo/něco prochází, používá se společně se slovesy jako  あるく "jít" とおる "projít" わたる "přejít" はしる "běžet". Je potřeba si zapamatovat že で které se používá k indikaci místa akce nemůže být v tomto případě použito.

\subsection{〜てくる/〜ていく: Ke/od (něčeho)}
Když くる"přijít" nebo  いく"jít" následuje te-formu sloves jako 出る "jít ven" はいる "vstoupit" はしる "běžet" あるく "jít" atd. Přidává to směrovou informaci k akci vyjádřenou předcházejícím slovesem. 〜てくる indikuje že pohyb je směřován směrem k nějakému místu, což je často pozice mluvčího. Na druhou stranu  〜ていく indikuje že pohyb je směřován pryč z daného místa. 

a) その男の人は駅の方に走って行きました。 Muž běžel směrem ke stanici.

