
\section{30. lekce}
\label{sec:lekce_30}

\subsection{úmyslná forma sloves}


Slovesa mají formu zvanou "Volitional" (úmyslná) 〜(よ)う, která je plain ekvivalent k 〜ましょう. Tato forma se může kombinovat s とおもっている (níže uvedeno) nebo se vyskytuje volně na konci věty. Vytváření této formy podléhá následujícím pravidlům.

\subsubsection{1だん slovesa}
Odtrhněte 〜る a přidejte 〜よう
きる nosit/nandat si →きよう
おきる probudit se →起きよう
食べる jíst → 食べよう
止める odejít →やめよう

\subsubsection{5だん slovesa}
místo うーだん dejte お
\begin{center}
\begin{tabular}{c c c}
かく psát → かこう& まつ čekat → まとう& のむ pít → のもう\\
泳ぐ plavat → およごう&死ぬ zemřít → 死のう&きる řezat → きろう\\
話す mluvit → 話そう&あそぶ hrát si → あそぼう&買う koupit → かおう\\
\end{tabular}
\end{center}

\subsubsection{Nepravidelná slovesa}
\begin{center}
\begin{tabular}{c c c}
する dělat → しよう& くる přijít → こよう\\
\end{tabular}
\end{center}


\subsection{〜(よ)うとおもっている: Úmysl udělat ---}
\begin{center}
\begin{tabular}{|| c | c | c ||}
\hline
まつりを見よう &と&おもっています。\\
\hline
Vvol. &to&omotte iru\\
\hline
\end{tabular}
\end{center}
Přemýšlím nad sledováním festivalu.

Pokud za Volitional formou slovesa následuje 〜(よ)うとおもっている, znamená to že dotyčný přemýšlí na děláním něčeho nebo plánuje něco dělat.

a) 私は留学(りゅうがく)のことを先生に相談(そうだん)しようとおもっています。Přemýšlím nad tím že bych zkonzultoval s učitelem studium v zahraničí.

\subsection{〜つもりだ: Plánovat--/zamýšlet ---}
\begin{center}
\begin{tabular}{|| c | c ||}
\hline
京都へ行く &つもりです。\\
\hline
Vplain (non-past) &tsumori da\\
\hline
\end{tabular}
\end{center}
Plánuji že pojedu do Kjóta.

Pokud za plain ne-minulou formou slovesa je 〜つもりだ znamená to že dotyčný plánuje něco udělat. Pokud chceme udělat negativní výrok, spona ani 〜つもりだ se neneguje ale negace se provádí v plain formě před tím.

今年の夏休みはアルバイトをするので、国へは帰らないつもりです。Jelikož budu o letních prázdninách na brigádě, tak se neplánuji vracet do vlasti.

〜つもりだった/でした se používá pokud mluvčí něco plánoval ale plán neuskutečnil.





\subsection{ Myšlenka/plán/přání třetí osoby}

〜(よ)うと思っている a 〜つもりだ  jsou používány pro popis toho co mluvčí zamýšlí nebo přemýšlí že udělá. 〜たい a ほしい jsou také používány pouze pro popis myšlenek a přání mluvčího. Aby bylo možné vyjádřit plán/myšlenku/přání třetí osoby, dotyčný musí použít výrazy jako 〜らしい、〜そうだ、〜ようだ、〜みたいだ、〜といっている

a) 村田さんはだいがくいんでけんきゅうを続けようと思っているらしい。Vypadá to že Paní Murata  přemýšlí na pokračováním ve výzkumu na magisterském studiu.

\subsection{〜し: A co víc---}

\begin{center}
\begin{tabular}{|| c | c || c ||}
\hline
温泉(おんせん)がある&し、&海もきれいです。\\
\hline
Plain form&shi&\\
Sub-klauzule&&Hlavní klauzule\\
\hline
\end{tabular}
\end{center}
Jsou tam horké prameny a (co víc) oceán je tam krásný.
し na konci klauzule vyjadřuje "a co víc". Partikule も "také"  je používaná v hlavní klauzuli stejně jako v sub-klauzuli pokud "také" má být zvýrazněno.

a) あきおくんはハンサムだし、いろいろなスポーツもできます。Akio je hezký, a co víc, je také dobrý v mnoha sportech.

Více než jedna shi-klauzule může být použita v jedné větě.


Shi-klauzule také může být použito pro vyjmenovávání důvodů jako příkladů což indikuje že je více nezmíněných důvodů. 

a) お金がないし、今新しいアパートに引っ越すことはできません。Nemám žádné peníze (, z dalších důvodů) takže se teď nemůžu stěhovat do nového bytu. 


\subsection{Klauzule upravující podstatná jména (3)}

\begin{center}
\begin{tabular}{|| c || c ||}
\hline
私たちがとまる&ところ\\
\hline
Vplain&Noun\\
Modifying clause&\\
\hline
\end{tabular}
\end{center} 
Místo kdu zůstaneme


Jak vysvětleno v Lekci 18 a Lekci 23, předmět (nebo téma) nebo objekt věty se může stát upraveným podstatným jménem, se zbytkem věty změněným na upravující klauzuli. Partikule které následují dané podstatné jméno se nepoužívají pokud je takové podstatné jméno upraveno klauzulí.


\begin{center}
\begin{tabular}{c c}
私はアパートに住んでいます。   → &[私が住んでいる]アパート\\
Bydlím v bytě.&byt ve kterém bydlím\\
母はびょういんへ行きました。→ &[母が行った]びょういん\\
Moje matka šla do nemocnice.&nemocnice kam moje matka šla\\
ともだちは会社ではたらいています。→ &[ともだちがはたらいている]会社\\
Můj kamarád pracuje pro firmu. & firma pro kterou můj kamarád pracuje\\
\end{tabular}
\end{center}


\subsection{〜のではないか/〜んじゃないか: Řekl bych že --}

〜のではないか/〜んじゃないか znamená "Řekl bych že--". Přestože se jedná o negativní výraz, nezpůsobí že celá věta je negativní.

しょうがくきんがもらえる学生がすくなすぎるのではないだろうか。Řekl bych že je jen málo studentů kteří dostanou stipendium.

\subsection{Partikule でも: nebo něco}

でも je partikule která může nahradit を což znamená "nebo něco" nebo "něco jako --" でも může také nahradit が a může nasledovat po partikulích jako に、へ、で、と

a) 何かいい仕事でもあったら、しょうかいしてください。Prosím dejte mi vědět pokud je nějaká dobrá práce nebo tak něco.



