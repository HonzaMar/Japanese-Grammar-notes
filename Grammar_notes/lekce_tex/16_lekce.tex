\section{16. lekce}
\label{sec:lekce_16}

\subsection{Planá forma: Minulost}
(1) Slovesa
Planá minulá kladná forma sloves se jednoduše získá změnou て v te-formě na た. Planá minulá a kladná forma se bude nazývat ta-forma. Planá minulá a záporná forma sloves se získá změnou 〜ない v ne-minulé záporné formě na 〜なかった.

2) I-adjektiva
Planá minulá kladná  a záporná forma je vytvářena stejně jako kladná a záporná forma i-adjektiv je vytvářena.

3) Na-adjektiva a Podstatná slova
Planá minulá kladná a záporná  forma (na-adj/noun +です) je vytvořena přidáním  〜だった a 〜ではなかった/〜じゃなかった k na-adjektivu nebo podstatnému jménu

\begin{center}
\begin{tabular}{cccccc}
\hline
&&\multicolumn{2}{c}{Ne-minulost}&\multicolumn{2}{c}{Minulost}\\
&&Kladná & záporná & Kladná&záporná\\
\hline
\multirow{ 12}{*}{Slovesa}&1だん& みる& みない& みた& 見なかった\\
&\multirow{9}{*}{5だん}& かく& かかない& 書いた& かかなかった\\
&&	およぐ& およがない& およいだ &およがなかった\\
&&	はなす& はなさない& はなした &はなさなかった\\
&&	まつ &またない& まった &またなかった\\
&&	しぬ &しなない& しんだ &しななかった\\
&&	あそぶ& あそばない& あそんだ& あそばなかった\\
&&	よむ &よまない& よんだ &よまなかった\\
&&	つくる& つくらない& つくった& つくらなかった\\
&&	かう &かわない& かった &かわなった\\
&\multirow{2}{*}{Nepravidelná slovesa}	&する& しない &した &しなかった\\
	&&くる& こない &きた& こなかった\\
\hline
\end{tabular}
\end{center}
	

\begin{center}	
\begin{tabular}{ccccc}
\hline
&\multicolumn{2}{c}{Ne-minulost}&\multicolumn{2}{c}{Minulost}\\
&Kladná&záporná&Kladná & záporná\\
\hline
I-adjektiva &たかい& たかくない& たかかった& たかくなかった\\
\hline
Na-adjektiva &きれいだ &きれいではない/じゃない& きれいだった& きれいではなかった/じゃなかった\\
\hline
Podstatná jména &ほんだ& 本ではない/じゃない& ほんだった& 本ではなかったじゃなかった\\
\hline
\end{tabular}
\end{center}


\subsection{〜と おもう:  Myslím že --}

Planá minulá forma stejně jako ne-minulá může být využita v citační klauzuli následována とおもう. Příslovce  jako  たぶん "pravděpodobně" a たしか "pokud si dobře pamatuji" jsou často používané společně s 〜とおもう.

a) 西川さんは たぶんたいいんしたと思います。Myslím že paní Nishikawa byla (asi) propuštěna z nemocnice.


\subsection{〜のです/〜んです: Vysvětlující predikát}
\begin{center}
\begin{tabular}{|c|c|}
\hline
行かない &のです/んです。\\
\hline
Planá forma & no/n da\\
\end{tabular}
\end{center}


Věty končící s 〜ます nebo 〜ました indikuje mluvčího vůli nebo faktická tvrzení. Na druhou stranu, pokud věty končí na  〜のです/〜んです, indikuje to že mluvčí nejen něco tvrdí, ale i něco vysvětluje. Otázky končící na  〜のですか/んですか jsou používané když chce mluvčí vysvětlit co posluchač právě řekl.


のです/んです následuje za planou formu. Narozdíl od plané formy se だ な

〜んです jsou ekvivalentní s 〜のです ale první se nikdy nesmí používat v psaném projevu.

a) 私は冬休みに九州へいきます。長崎に友達がいるのです。Během zimních prázdnin jedu na Kjúšú. (protože) Kamarád bydlí v Nagasaki.



\subsection{〜ないでください: Zdvořilý negativní rozkaz}
\begin{center}
\begin{tabular}{|c|c|}
\hline
授業を休まないで &ください。\\
\hline
Vnaide&kudasai\\
\hline
\end{tabular}
\end{center}

Když se nai-forma slovesa zkombinuje s で a je následována ください, znamená to "prosím nedělejte ---." 〜ないでください jako 〜てください je používáno jen v určitých situacích.

a) 辞書を使わないでくさい。Prosím nepoužívejte slovník.

\subsection{Partikule が: Subjekt}
は identifikuje subjekt věty jako téma když nahradí が. Ale,  が nemůže být nahrazeno は když celá věta popisuje co mluvčí rozeznává jako nový fakt nebo událost. Tedy が je používáno na popis kdo dělá co, kde a kdy je nová informace. 

a) きのう両親が名古屋から来ました。Včera přijeli moji rodiče z Nagoji.


\subsection{Spojení それに: Navíc} 
それに je spojení používané ke spojení dvou vět, znamenající "nejen to" "navíc". Partikule も je často používána v druhé větě.

a) 私はロンドンでかぶきをみました。それに、すもうもみました。Viděl jsem kabuki v Londýně, Navíc jsem viděl i Sumó.

それに je podobné  それから "navíc", ale první je více empatické. Ale, narozdíl od それに, それから může také znamenat "poté" indikující sekvenci událostí.
























