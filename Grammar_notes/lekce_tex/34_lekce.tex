
\section{34. lekce}
\label{sec:lekce_34}

\subsection{Ba-forma}
Když vyjadřujeme podmínku/hypotetickou situaci znamenající "(pouze) pokud...", slovesa, i-adjektiva, (na-adj/podst. + だ) musí být změněno do ba-formy. Ba-forma je získána podle následujících pravidel.

\subsubsection{Slovesa}

1) 1 だん slovesa
drop 〜る a přidej  〜れば
\begin{center}
\begin{tabular}{cc}
いる → いれば & かえる → かえれば\\
おきる → おきれば &  食べる → たべろ\\
\end{tabular}
\end{center}


2) 5だん slovesa

Vyměň konečné うーだん hiraganu ze slovníkové formy za  えーだん hiraganu a přidej 〜ば
\begin{center}
\begin{tabular}{ccc}
かく → かけば &まつ → まてば &たのむ → たのめば\\
いそぐ → いそげば & しぬ → しねば &かえる → かえれば\\
さがす → さがせば & あそぶ → あそべば &かう → かえば\\
\end{tabular}
\end{center}


3) Nepravidelná slovesa

する → すれば  くる → くれば



\subsubsection{I-adjektiva}

Ba-forma i-djektiv se vytváří změnou い na 〜ければ

あつい → あつければ

いい → よければ

たかい → たかければ

\subsubsection{Na-adjektiva/podstatná jména}


Ba-forma (na-adjektiv/podstatných+だ)  se vytváří přidáním 〜なら(ば) k na-adjektivu nebo podstatnému jménu. Většinou se ば  v 〜ならば vynechává.
べんりだ → べんりなら

日本人だ → 日本人なら



\subsubsection{Ba-forma negativních }
Ba-forma negativních forem sloves nebo i-adjektiv se získává změnou い v 〜ない na 〜ければ jako níže.
\begin{center}
\begin{tabular}{cc}
おきない → おきなければ     & わからない → わからなければ \\
 べんきょうしない → べんきょうしなければ & こない → こなければ \\
さむくない → さむくなければ  & よくない → よくなければ  \\
\end{tabular}
\end{center}


Ba-forma negativa z (na-adj/podstatného jména + だ) je vytvořena změnou z い na  〜ければ Například ではない se změní na でなければ



\subsection{〜ば: Kondicionál/hypotetické (Pouze pokud--)}

\begin{center}
\begin{tabular}{||c||c||}
\hline
このボタンをおせば &うこきます。\\
\hline
V/A/N ba & Hlavní klauzule\\
\hline
Sub-klauzule&\\
\hline
\end{tabular}
\end{center}
Pokud zmáčkneš tohle tlačítko, nastartuje se to.


Sub-klauzule končící ba-formou vyjadřuje hypotetickou podmínku "(pouze) pokud...". Když ba-forma akčního slovesa je použita, hlavní klauzule musí být faktický výrok a nemůže být návrh, pozvánka nebo instrukce tedy 〜ましょう、 〜ませんか、〜てもらえませんか、〜てください

a) ラジオがこわれてしまったが、ぶひんをかえればなおる、と店の人はいった。Rádio se rozbilo, ale prodejce říkal že se dá opravit pokud vyměním pár součástek.


Když ba-forma i-adjektiva, slovesa vyjadřující stav jako いる "existovat/být", ある"existovat/být" nebo  わかる "rozumět", nebo negativní slovesa se objeví v sub-klauzuli tak hlavní klauzule může být jakéhokoli typu. 

今日つごうがわるければ、明日にしましょう。 Pokud se ti to dnes nehodí, provedeme to zítra.


ba-klauzule je porovnatelná s tara-klauzulí. Ba-klauzule dává potřebnou podmínku pro stav/akci v hlavní klauzuli, která se objeví jako nevyhnutelný následek akce/stavu v ba-klauzuli. Na druhou stranu, tara-klauzule pouze vyjadřuje "pokud nastane akce/stav, něco se stane" a výsledek vyjádřený v hlavní klauzuli může být nevyhnutelné nebo něco neočekávané.





\subsection{Tázací slovo + 〜ば いいですか: Hledání rady}

\begin{center}
\begin{tabular}{||c||c|c||}
\hline
どうやって& しらべれば &いいですか。\\
\hline
Tazací slovo &V ba& ii desu ka\\
\hline
\end{tabular}
\end{center}
Jak bych to měl ověřit?

Když je tázací slovo jako どこ "kde" どう/どうやって "jak" atd. použito s ba-formou sloves následováno  いいですか znamená to "Jak/Kam bych měl ---?"

a)
A: しょるいはどこにおけばいいですか。Kam bych měl dát dokumenty?

B: つくえのうえにおいてもらえますか。Položil bys je na stůl?


(Tázací slovo + 〜ば いいですか) je ekvivalentní s (Tázací slovo + 〜たら いいですか)





\subsection{Tranzitivní slovesa a Intranzitivní slovesa}
Existují páry tranzitivních/intranzitivních sloves jako (〜を)こわす "(někdo něco) rozbít" a  (〜が) こわれる "(něco) se rozbít". Slovesa jako こわす která pracují s přímým objektem označeným partikulí を jsou nazvané tranzitivní slovesa (たどうし). Slovesa která nepracují s přímým objektem jakoこわれる jsou intranzitivní slovesa (じどうし). Tranzitinví slovesa jsou používaná pro popis co osoba něčemu udělá, zatímco intranzitivní slovesa jsou používaná na popis co se někomu/něčemu stalo.


\begin{center}
\begin{tabular}{cc}
(電気を)つける (světlo) zapnout     &  (電気が)つく (světlo) zapnout se\\
(電気を)けす (světlo) zhasnout      &  (電気が)きえる (světlo) zhaslo se\\
(ドアを)しめる  zavřít (dveře)    &  (ドアが)しまる   (dveře) zavřít se\\
\end{tabular}
\end{center}
a) 昨日でがける時、ヒーターを消すのを忘れたが、家に帰ったときには消えていた。Zapomněl jsem vypnout topení když jsem včera odcházel, ale ono to se to samo vyplo když jsem došel domů.




\subsection{Suffix 〜てき(な)}

〜てき(な)je suffix který mění podstatná jména (většinou čínského původu) na na-adjektiva. Také, jako u normálních na-adjektiv lze za pomocí に vytvořit příslovce

じどうてき(な)automatické    じどうてきに automaticky

a) このドアはじどうてきにひらいたりしまったりします。Tyhle dveře se zavírají a otevírají automaticky. 

