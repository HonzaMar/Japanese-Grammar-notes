\section{39. lekce}
\label{sec:lekce_39}

\subsection{ ~ために/~ための: účel}
\subsubsection{ ~ために: Za účele~}
\begin{center}
\begin{tabular}{|c|c|c|}
学費をかせぐ& ために &アルバイトをしています。\\
\hline
V dict. &tame ni& Verb\\
\hline
\end{tabular}
\end{center}
Mám brigádu abych vydělal peníze na školní výdaje.



\begin{center}
\begin{tabular}{|c|c|c|c|}
\hline
調查&の &ために&アンケートを行っています。\\
\hline
N (+ suru) &no&tame ni&Verb\\
\hline
\end{tabular}
\end{center}
Pro náš průzkum používáme dotazník.

〜ために je používané na modifikaci slovesa na vyjádření za jakým účel se něco dělá, znamenající "za účelem ---." ~ために je předcházeno slovníkovou formou slovesa. V případě suru-nouns je potřeba の jako v  〜のために.  に v ~ (の)た めに je občas vynecháváno ve formálním písemném projevu. 
a) 資料(しりょう)を集( あつ)めるために図書館(としょかん)に行きました。Šel jsem do knihovny abych sesbíral nějaké podklady.



b) ロングさんは社会学(しゃかいがく)の研究( けんきゅう)のために日本へ 来た。Pan Long přišel do Japonska aby dělal výzkum v sociologii.

\subsubsection{~ための: pro/na}
\begin{center}
\begin{tabular}{|c|c|c|}
\hline
学費をかせぐ&ための&アルバイト\\
\hline
V dict.&tame no&Noun\\
\hline
\end{tabular}
\end{center}
Brigáda na vydělání peněz  pro/na školní výdaje.



\begin{center}
\begin{tabular}{|c|c|c|c|}
\hline
調查(ちょうさ)&の&ための& アンケート\\
\hline
N(+suru)&no&tame no&Noun\\
\hline
\end{tabular}
\end{center}
Když je 〜ため použito pro modifikaci podstatného の je přidáno po ため jako v 〜ための.  Forma předcházející  ~ための je stejná jako forma předcházející  〜ために. 
a) 先週 (せんしゅう)、漢字を調べるための辞書を買いました。 Minulý týden jsem si koupil slovník pro vyhledávání kanji.


b) 私は環境保護のための運動に参加しようと 思っている。Přemýšlím o přidání se do hnutí za ochranu prostředí.


\subsection{ 〜たり〜たりだする: Někdy ~, a jindy --}

Vたり Vたり する “dělat věci jako ~” bylo představeno v Lekci 22. ~たり 〜たり následované  だ nebo する může být využito pro vyjádření “někdy ~, a jindy ~." 〜たり je vytvořeno přidáním りk plané minulé formě slovesa, i-adjektiva nebo  [na-adj/noun + だ ]. Tento vzor vět se často objevuje s ~によって " záviset na ~." 

a) 家賃(やちん)は場所( ばしょ)によって高かったり安かったり です。Nájem je vysoký nebo nízký podle lokace.

b) A: バイト代(だい)は毎月だいたい同じですか。Je tvoje výplata z brigády skoro vždy stejná?


B: いえ、3万円ぐらいだったり6万円ぐらいだったりです。 Ne, někdy to je okolo 30 000 jenů, někdy zase 60 000 jenů.

Jak je ukázáno v příkladech výše, 〜たり〜たり je často používáno pro vyjmenovávání opačných situací. 〜たり se většinou objevuje dvakrát, může být ale užito vícekrát. Jedno 〜たり může být využito pro indikaci jedné z možných situací.


私は車で通勤(つうきん)しているが、帰りに友だちと酒を飲む時などは電車で行ったりする。Většinou jezdím autem, ale někdy jedu vlakem když jdu pít s kamarády.





\subsection{ Klauzule modifikující podstatné jméno(4)}
Abstraktní podstatná jména jako  もくてき “účel," りゆう “důvod," げんいん “příčina," どうき
“motivace," atd. můžou být modifikována klauzulí jak ukázáno níže za účelem vysvětlení toho co daná podstatná jména indikují.
ボランティア活動(かつどう)を始( はじ)める目的(もくてき) účel/cíl konání dobrovolnických aktivit



仕事を続けなかった理由(りゆう) důvod nesetrvání v mé práci




事故が起こった原因 (げんいん) příčina nehody


友だちにクラブをやめた理由を聞かれました。Kamarád se ptal na důvod proč jsem opustil klub.

\subsection{ 〜のなら / 〜んなら : Pokud je to tak,  tak---}

〜のなら/~んなら vyjadřuje stejnou věc jako  〜のだったら/~んだったら(Lekce 36), "Pokud jde o to, tak -." Každopádně, 〜のだったら/~ん だったら je o trošičku hovorovější než  〜のなら/~んなら、~のなら~んなら následuje za planou formou slovesa nebo i-adjektiva. Na-adjektiva a podstatná jména přímo předcházejí なら bez の nebo ん.

グループで発表するのなら、よく話し合った ほうがいい。Pokud budeme mít skupinovou prezentaci, tak bychom to měli pořádně prodiskutovat.

\subsection{Partikulární fráze}
\subsubsection{ ~によって: Záleží na/podle toho--}
〜によって znamená "záleží/záležet na~ ” a je často používané s  ちがう “lišit se," nebo 〜たり
〜たりだする.

パートの時給(じきゅう)は仕事の種類(しゅるい)によって違(ちが)います。Hodinová mzda na brigádě je různá podle toho co to je za práci.


\subsubsection{ 〜として: jako}
〜として je používané na vyjádření použití/účelu/funkce. Také to může být použito pro indikaci funkcí (společenské) a rolí.


a) かせいだお金は小遣いとして全部使(ぜんぶつか)って しまいました。 Všechny peníze co vydělám používám na denní výdaje.
b) 橋本さんは今回(こんかい)コーチとしてオリンピックに出ることになった。Bylo rozhodnuto že paní Hashimoto pojede na olympiádu jako trenér.





































