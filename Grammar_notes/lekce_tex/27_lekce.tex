\section{27. lekce}
 \label{sec:lekce_27}
\subsection{〜かもしれない Možná (Být / Dělat)}

\begin{center}
\begin{tabular}{|| c | c ||}
中山さんは今日来る & かもしれません\\
\hline
Plain form & kamo shirenai \\
\end{tabular}
\end{center}

Pan Nakayama možná dnes přijde.

\paragraph{}〜かもしれない znamená "možná (být/dělat), doslovně "Nemůžu vědět jestli --- je pravda." Toto se používá když mluvčí nemůže říci něco s jistotou ale ta možnosti určitě existuje. Toto 〜かもしれない nemůže být použito v tázacích větách. 

a) このテープレコーダーはもしかしたらこわれているかもしれません.   Tento záznamník je asi rozbitý. 

b)しけんはむずかしいかもしれませんが、がんばります. Ta zkouška je možná těžká, ale budu se snažit.

\subsection{〜だろう/〜でしょう: Předpokládám}

\begin{center}
\begin{tabular}{|| c | c ||}
中山さんは今日来る & でしょう。\\
\hline
Plain form & daroo \\
\end{tabular}
\end{center}

Předpokládám že pan Nakayama dnes přijde.

\paragraph{} 〜だろう/〜でしょう  "Předpokládám" indikuje mluvčího dohad na základě jeho subjektivního pohledu a je použito pokud mluvčí nemůže udělat závěr nebo si něčím nemůže být naprosto jistý.

a) リーさんはたぶんしゅうでんにまにあわなかっただろうと思います。Paní Lee pravděpodobně nestihla poslední vlak.

\subsection{〜てみる: zkus a uvidíš}

\begin{center}
\begin{tabular}{|| c | c ||}
としょかんで聞いて & 見ました\\
\hline
V te & miru \\
\end{tabular}
\end{center}
Zeptal jsem v knihovně (abych věděl).

Když se za te-formu slovesa dá 見る, "vidět", znamená to "zkusit --- (a vidět)" nebo "udělat --- a zjistit (co se stalo)"

a) カラオケには行ったことがないので、一度行ってみたいと思っています。Nikdy jsem nebyl v karaoke baru, takže si myslím že bych rád jednou šel (a zjistil jaké to je).


\subsection{Podstatné jméno + なら: Pokud jde o  -- tak}

Podstatné slovo  + なら, může být použito na zvýraznění tématu což indikuje o čem se mluvčí chystá mluvit . Znamená to "Pokud je to o  ---", "Pokud je--- tento případ" nebo " Pokud mluvíš o --"  Slovo vybrané jako téma spojením なら je většinou předem naznačeno v předchozích částech konverzace od konverzačního partnera.

A: 大村さん、見ませんでした? Neviděl jsi pana Omuru (že)?


B: 大村さんなら、事務所にいましたよ。Ah, Pan Omura, byl v kanceláři.

\subsection{Partikule も[+ neg.]: Ani}
 Když je partikule も použitá v negativní větě, znamená to "ani---." Toto použití も se často obejvuje když následuje za množstevním slovem "jeden."

a) 試験に出た漢字が一つも読めませんでした。V testu jsem nemohl přečíst ani jedno kanji.


\subsection{Příslovce まだ [+ aff.] a もう [+ neg.]}

 まだ je používané v pozitivních větách jako příslovce korespondující s  "stále" a もう je používané v negativních větách korespondující s  "už ne."

A: 北海道はまだ寒いですか。Je na Hokkaidu stále zima?

B:いえ、もう寒くないと思います。 Ne, myslím že tam už není zima.


\subsection{Zdvořilosti (2): Pokorná forma sloves}

Mnoho slovesa se může změnit na pokornou formu s použitím [お+  V stem+ する]. tato forma se používá pro referenci vlastních akcí vůči nadřízeným. 

待つ →    お待ちする čekat na (nadřízeného)

持つ →    お持ちする odnést něco někomu (nadřízenému)

聞く →    お聞きする zeptat se (nadřízeného)

Pokorná forma sloves která se vytvářejí pomocí suru-nouns je [ご+  N (+suru)+する]


相談する →    ご相談する konzultovat

連絡する →    ご連絡する kontaktovat

電話する →    ご電話する zavolat
