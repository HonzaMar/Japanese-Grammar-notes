\section{2. lekce}
\label{sec:lekce_2}

\subsection{Ukazovací zájmena: Kore, Sore a Are}
Tři ukazovací zájmena jsou používané na referenci věcí.
\begin{enumerate}
\item Kore "toto" je použito když se odkazuje na něco blízkého k mluvčímu
\item Sore "tamto" je použito při odkazu na něco co je blízko k posluchači, nebo je trošku dál od obou
\item Are "támhle toto" je použito když se odkazuje na něco co je mimo dosah obou
\end{enumerate}

\subsection{Tázací věty s tázacím zájmenem}
\begin{center}
\begin{tabular}{||c|c||c|c|c||}
\hline
それ&は&なん&です&か。\\
\hline
Noun&wa&tázací zájmeno&desu&ka\\
\hline
\end{tabular}
\end{center}
Co to je?

Pokud se ptáme co něco je, tázací zájmeno nan "co" je použito. Toto slovo je na stejném místě ve slovosledu v otázce tak jako odpověď bude.

A: それは なんですか。Co to je?

B:これはおもちです。 To je rýžová koláč.


\subsection{Partikule no: Modifikace podstatných jmen (2)}

(N1 no N2) (Lekce 1) může indikovat vlastnictví jako v watashi no kaban "můj batoh" "Čí --" je vyjádřeno jako dare no. (N1 no N2) může být také zkráceno na N1 no. 

A:これは だれの おはし ですか。 Čí jsou to hůlky?

B: カーンさんのです。Jsou pana Kahna.

Poznamenejte si že (N1 no N2) může být zkráceno pouze pokud N2 indikuje vlastněnou věc.




