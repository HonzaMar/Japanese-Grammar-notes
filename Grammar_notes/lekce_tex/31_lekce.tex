\section{31. lekce}
\label{sec:lekce_31}

\subsection{〜か: vložená otázka}

"(nevím) kdy ..." "(nevzpomínám si) jestli ... nebo ne" atd. může být vyjádřeno skrze vložené otázky. Vložené otázky se často používají společně se slovesy jako わかる "vědět/rozumět" a しっている "vědět", おばえている "pamatovat si", わすれる "zapomenout" nebo  きく/しつもんする "zeptat se".

\subsubsection{〜か (わかりません): (nevím) Kdy/kde, atd.}

\begin{center}
\begin{tabular}{||c||c|c||c||}
\hline
どこで&なくした&か&わかりません。\\
\hline
Q word&Plain form&ka&Verb\\
\hline
\end{tabular}
\end{center}
Nevím kde jsem to ztratil.

Pokud se věta Tázací zájmena je vložena do další věty, sloveso, i-adjektivum nebo na-adjektivum/podstatné jméno po kterém následuje か musí být v plané formě.

いつ合いますか    →          いつあうかわかりません
kdy se sejdeme.    				nevím kdy se sejdeme

どんなくるまですか    →        どんな車か忘れました
co je to za typ auta				zapomněl jsem co je to za typ auta



a) 石田さんがいつ出発するかおしえてください。Prosím řekni mi kdy Paní Ishida odejde.


Na vyjádření otázky  “zda ano ~ nebo ne”, lze vytvořit z otázky předcházející  か planou formu

見つかりましたか  →  見つかったかどうか聞きませんでした
bylo to nalezeno?      → Neptal jsem se jestli se to našlo nebo ne.


おもしろいですか →  おもしろいかどうか知りません
Je to zajímavé? → Nevím jestli je to zajímavé nebo ne.



a) 今度日本りょうりを作ろうとおもっていますが、うまくできるかどうかじしんがありません。
Přemýšlím o tom že si občas uvařím Japonské jídlo, ale nejsem si jistý jak se mi to povede.

b) 早川さんはよやくがひつようかどうか店の人に聞きました。Pan Hajakawa se zeptal zaměstnance zda potřebuje rezervaci nebo ne.

\subsubsection{Alternativní otázka}

Alternatvní otázka může být vložena do věty ve formě  ーか ー(ない)か

行きますか、行きませんか →    行くか行かないかわかりません
jdeš nebo nejdeš.          ->.        Nevím jestli jdu nebo nejdu


がっしゅくに行けるか行けないかへんじをしてない。Neřekl jsem jestli můžu jet na tréninkový tábor nebo ne.


\subsubsection{ーのか/ーのか どうか}
Tázací věta ve formě ーのですか/んですか by měla být změněná na  ーのか nebo ーのかどうか aby se vytvořila vložená otázka. Poznámka že ん nemže být použito místo  の v tomto případě.


a)西村さんがどうして会社をやめたのかわかりません。 Nevím proč paní Nishimura dala výpověď.

b)小山さんに本当にてんきんするのかどうか聞いた。  Zeptal jsem se pana Koyai jestli bude převelen nebo ne.


\subsection{Tázací slovo + 〜たらいいですか: žádání o radu}

\begin{center}
\begin{tabular}{|| c || c | c ||}
どこへ & 行ったら& いいですか。\\
Tázací slovo &V tara&ii desu ka\\ 
\end{tabular}
\end{center}
Kam bych měl jít?

Pokud se tázací zájmeno/slovo jako どこ "kde" , どう "jak" použije společně s Tara formou slovesa kterou následuje いいですか, znamená to "Kam/jak mám ...?"

a) 
A:ていきをおとしちゃったんですか。 Ztratil jsi lítačku? 
B: ええ。どうしたらいいですか。 Jo. Co mám dělat?


\subsection{〜ておく/〜とく. Příprava na budoucí použití/účel}

\begin{center}
\begin{tabular}{||c|c||}
\hline
けいさつにとどけて & おきます。\\
\hline
V te & oku\\
\hline
\end{tabular}
\end{center}
 Půjdu napřed a nahlásím to policii.
 
 Když te-forma slovesa je následována おく "položit", znamená to  "dělat něco dopředu/ připravit si něco do budoucna" 〜ておく nemůže být použito se slovesy které vyjadřují nekontrolovatelný stav/podmínku jako například できる "být připraven/být schopen udělat - " ある "být/existovat" nebo よろこぶ "být potěšen". まえもって "dopředu" může být použito s  〜ておく když význam "dělat něco dopředu" je potřeba zdůraznit
 
a)  ともだちが来るので、ワインを買っておこうと思っています。Protože přijede kamarád, přemýšlím o nákupu nějakého vína (dopředu).

V mluveném jazyku 〜ておく je zkráceno na 〜とく. 〜でおく je zkráceno na 〜どく.




\subsection{Podstatné jméno + のよう(な) /Podstatné jméno +みたい(な): Podobnost}

\begin{center}
\begin{tabular}{||c|c|c||}
\hline
富士山&の&ようです。\\
\hline
Podstatné jméno&no&yoo da\\
\hline
\end{tabular}
\end{center}
Vypadá to jako hora Fuji.


\begin{center}
\begin{tabular}{||c|c||}
\hline
富士山&見たいです。\\
\hline
Podstatné jméno&mitai da\\
\hline
\end{tabular}
\end{center}
Vypadá to jako hora Fuji.

〜の よう(な)nebo 〜みたい(な)které následují podstatné jméno vyjadřuje kvalitu nebo stav podobnosti s daným podstatným jménem.  〜の よう(な)a 〜みたい(な) se chovají jako na-adjektiva a mění své formy jako  〜の ように a  〜みたいに před slovesy.  〜みたい(な)je výraz používaný převážně v mluvené řeči. 〜よう(な) se používají jak v mluvené řeči tak i v psané podobě.

a) 今日はあたたかくて、春のようです。Dnes je teplo, je to jako jaro.

\subsection{Sufix}
1. 

〜さ je suffix přidaný k stem formě i-adjektiv, měnící adjektivum na podstatné jméno. 

高さ výška  
ながさ délka
おもさ hmotnost
ひろさ rozloha

2.

〜っぽい je ekvivalent k "-ish" v angličtině a chová se jako i-adjektiv

ちゃいろっぽい nahnědlý 
こどもっぽい dětinský
くろっぽい načernalé

3. 〜さき  může být přidáno k stem-formě sloves nebo suru-podstaných jmen aby indikovalo "místo (konání)"

行き先 cíl
れんらくさき destinace
しゅうしょくさき místo zaměstnání

