\section{10. lekce}
\label{sec:lekce_10}
\subsection{Věty vyjadřující lokaci}
\begin{center}
\begin{tabular}{|c|c|c|c|c|}
\hline
カレン& は &あそこ &に& います。\\
\hline
Noun (životné) &wa&místo&ni&imasu\\
\hline
\end{tabular}
\end{center}


\begin{center}
\begin{tabular}{|c|c|c|c|c|}
\hline
レモン &は& ここ& に &あります。\\
\hline
Noun (neživotný)&wa&místo&ni&arimasu\\
\hline
\end{tabular}
\end{center}

Obojí います a  あります jsou slovesa vyjadřující lokaci, znamenající "(někdo/něco) je v (místo)." います je používáno pro životné věci a  あります pro neživotné (i rostliny). Partikule に označuje lokaci.

a)高田さんはじむしょにいます。Pan Takada je v kanceláři.


Tázací věta "Kde je〜?", může být vytvořena nahrazením místa slovem  どこ a přidáním か za 〜ます.
a)
A: 山中さんはどこにいますか。Kde je paní Yamanaka?
B:にかいにいます。Je o patro výše.

で, partikule indikující lokaci ve které se něco děje nemůže být použitá s います/あります.
Porovnejte:
a) スミスさんはとしょしつにいます。Pan Smith je v knihovně.
b) スミスさんはいつもとしょしつでべんきょうします。Pan Smith vždy studuje v knihovně.

2. (Místo +です) může být použito jako zkrácená věta pro (místo にいます/あります). Například 2かいにいます "Je o patro výše" je zkrácená na 2かいです "Je o patro výše" どこにいますか a どこにありますか může být zkráceno na どこですか.

a) キムさんはだいどころにいます。カーンさんはそとです。Pan Kim je v kuchyni. Paní Kahn je venku.

\subsection{Fráze lokace}
\subsubsection{Fráze lokace: N1 の N2}
\begin{center}
\begin{tabular}{|c|c|c|}
\hline
れいぞうこ& の& 中\\
\hline
Noun1 & no& Noun2\\
\hline
\end{tabular}
\end{center}

(N1のN2) může indikovat lokace když N2 je podstatné jméno vyjadřující pozici nebo prostor jako  中 "vevnitř," うえ"nad,"  した "pod," atd.

〜のうえ nad   〜の下 pod
〜の中 vevnitř		〜のそと venku
〜の前 vepředu   	〜のうしろ za
〜の右 napravo		〜の左 nalevo
〜よこ po boku 		〜のそば poblíž
〜のとなり vedle

\subsection{〜そう(な)}

Když je 〜そう(な) přidáno k stem-formě i-adjektiv nebo k na-adjektivům, znamená to "vypadá to---". Výsledek se následně chová jako na-adjektivum. Tento výraz se nemůže používat u zjevných faktů jako třeba barva. 

おいしい chutné おいしそう(な) chutně vypadající
まずい nechutné まずそう(な)nechutně vypadající
たかい drahé たかそう(な)draze vypadající
いい dobré よさそう(な)dobře vypadající

a)
A: おいしそうですね。Vypadá to chutně, že?
B: そうですね。Ano.

\subsection{Ostatní}
こっち、そっち、 あっち a どっち jsou demonstrativa indikující směr.

a) 電話はあっちです。Telefon je tímto směrem.



