\section{25 lekce}
\label{sec:lekce_25}

\subsection{Dělání a přijímání laskavostí}
Slovesa dávání a přijímání あげる くれる もらう\footnote{lekce 24} společně s te-formou sloves vyjadřují myšlenku dávání a přijímání laskavostí činnosti. 

\subsubsection{〜てあげる}
\begin{center}
\begin{tabular}{||c|c||c|c||}
\hline
リーさん& は/が& えいごをおしえて& あげまあした。\\
\hline
Dávající&wa/ga&V te&ageru\\
\hline
\end{tabular}
\end{center}
Pan Lee ho/jí učil anglicky.

〜てあげる znamená "udělat něco pro někoho (kdo není mluvčí nebo v jeho skupině)" Obdarován může být ve skupině mluvčího, pouze pokud mluvčí je dávající. Narozdíl od slovesa あげる které označuje obdarovaného partikulí に , tak v tomto případě je obdarovaný označen na základě partikule, která patří ke slovesu v te-formě.

上田さんはたかしくんにうたを教えます → 上田さんはたかしくんにうたをおしえてあげます。
Pan Ueda učí Takašiho písničky 	->	Pan Ueda dělá Takashimu laskavost že ho učí písničky



\subsubsection{〜てくれる}
\begin{center}
\begin{tabular}{||c|c||c|c||}
\hline
リーさん& は/が & えいごを& おしえて くれました。\\
\hline
Dárce &wa/ga&V te&kureru\\
\hline
\end{tabular}
\end{center}
Pan Lee mě učil anglicky.

〜てくれる znamená "(někdo) udělá něco pro mě (mluvčího, neboj jeho skupinu)". Stejně jako  〜てあげる, příjemce laskavosti je indikován mnoha různými způsoby

母が私にようふくをおくります   →   母が私にようふくを送ってくれます
Moje matka mi poslala oblečení.    -> 	Moje matka (je dostatečně hodná že) mi poslala oblečení.




\subsubsection{〜てもらう}
\begin{center}
\begin{tabular}{||c|c||c|c||c|c||}
\hline
わたし& は/が& ともだち &に& えいごをおしえて& もらいました。\\
\hline
Příjemce&wa/ga&Dárce&ni&V te&morau\\
\hline
\end{tabular}
\end{center}
Můj kamarád mě učil anglicky.


〜てもらう znamená "nechat někoho (kdo není mluvčí) něco udělat" doslova "obdržet laskavost od někoho děláním něčeho". Partikule に označuje osobu která dává laskavost.

a) キースさんはたかおさんにひっこしを手伝ってもらいました。 Keith obdržela laskavost od Takaa že jí pomohl se stěhováním.



\subsubsection{〜てさしあげる、〜てくださる、〜ていただく}

〜てあげる、〜てくれる a     〜てもらう mají svoje zdvořile ekvivalenty 〜てさしあげる、〜てくださる a 〜ていただく. 〜てさしあげる je pokorný ekvivalent  〜てあげる a je použit pro vyjádření dělání něčeho pro nadřízené.   〜てくださる je respektující ekvivalent  〜てくれる a je použit v případě že nadřízený udělal něco pro mluvčího.

A: だれが説明してくれたんですか。Kdo ti to vysvětlil?
B: 中村先生が説明してくださったんです。Profesor Nakamura mi to vysvětlil.

〜ていただくje pokorný ekvivalent  〜てもらう a je použit pokud někoho (většinou mluvčího) nadřízený udělal něco pro něj/ní.

A: だれにすいせんじょうをかいてもらったんですか。Koho jsi poprosil aby ti napsal doporučující dopis?
B: 大川先生に書いていただきました。Profesor Ókawa mi ho napsal.

\subsubsection{〜てやる}
〜てやる je používáno místo  〜てあげる když laskavost je směřována směrem ke zvířatům, sobě rovným, podřízeným nebo mladším členům rodiny

私は毎朝犬をさんぽにつれて行ってやります。Beru svého psa na procházku každé ráno.


\subsection{〜の: Nominalizér (2)}

〜の následující za klauzulí která končí slovníkovou formou sloves dělá z (klauzule+の) ekvivalent podstatnému jménu \footnote{Lekce 22}. V následujících případech (klauzule+の) je následovaná partikulí を a funguje jako objekt (předmět) slovesa, jako je てつだう "pomoct" a   みる "vidět". Subjekt uvnitř klauzule je vždy označen partikulí が


a) 私はともだちがにもつをはこぶのを手伝いました。Pomohl jsem kamarádce s jejími taškami.

(klauzule+の také může být použito jako v 〜のをやめる "přestat dělat.." 〜のをわすれる "zapomenout udělat ..."


a) 北川さんはカナダに留学するのをやめました。Paní Kitagawa vzdala studium v Kanadě.



\subsection{〜てしまう/〜ちゃう: Kompletně/Konečně}
\begin{center}
\begin{tabular}{||c|c||}
\hline
私は道にまよって &しまいました。\\
\hline
V te& shimau\\
\hline
\end{tabular}
\end{center}

Když je te-forma slovesa následována しまう znamená to že akce proběhla "kompletně" nebo "konečně" 〜てしまう také může implikovat mluvčího lítost, naštvanost nebo frustraci.

a) 日本語のしゅくだいはもうやってしまいました。Konečně jsem dokončil domácí úkol z Japonštiny.

〜ちゃう je zkrácená forma 〜てしまう a  〜じゃう je zkrácená forma 〜でしまう. Tyto formy se v psaní nepoužívají.



\subsection{〜やすい/にくい: Jednoduché udělat --/ Obtížně udělat --}
Kořen slovesa následovaný 〜やすい nebo  〜にくい znamená "jednoduché udělat--" respektive "těžké udělat--". Jakmile je sloveso zkombinováno s 〜やすい nebo 〜にくい, stává se z něho i-adjektivum.

a) 道がわかりにくかったので、こうばんで教えてもらいました。Jelikož bylo težké najít cestu, zeptal jsem se v policejní budce.















