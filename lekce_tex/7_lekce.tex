\section{7. lekce}
\label{sec:lekce_7}
\subsection{Počítání věcí}
\subsubsection{Počítadla}
Při počítání, se používají různá počítadla podle toho co se počítá. Počítadla jsou vždy spojená s čísly a samostatně se nepoužívají.

\begin{enumerate}
\item 〜まい ploché a tenké objekty jako papír nebo talíře
\item 〜さつ spojené stránky nebo listy papíru jako knihy nebo časopisy
\item 〜こ přibližně kulaté věci jako jablka
\item 〜ほん dlouhé a úzké věci jako květiny, tužky, flašky atd.
\item 〜はい hrnečky, misky skleničky pití, rýže. nudlí v polévce atd.
\end{enumerate}

Existují výjimky vyjadřující se ve změně výslovnosti jak při kombinaci s čísly tak i s tázacím slovem なん. 

\subsubsection{Čísla japonského původu}
Slova japonského původu pro čísla os jedné do deseti můžou být použité pro počítání stejné věci jalo čínská čísla s pomocí počítadlem 〜こ. Od jedenácti výše se používají pouze čínská čísla s počítadlem 〜こ

\subsection{Množstevní slova}
Když se množstevní slovo (jako například (číslo + počítadlo nebo slova jako はんぶん "polovina",  ぜんぶ "vše" atd.) použije ve větě, mělo by se dávat pozor na pořadí slov ve větě. 

\subsubsection{Nominální věty}
\begin{center}
\begin{tabular}{|c|c|c|c|c|}
\hline
ばら& は& いっぽん& 300円& です。\\
\hline
Noun1&wa&množství&Noun2&desu\\
\hline
\end{tabular}
\end{center}
Růže stojí 300 jenů jedna.

V nominální větě N1 は N2 です je množstevní slovo položeno vždy před  N2 です a ne před N1 は. 


a) はがきは1まい50円です。Pohled stojí 50 jenů každy.

\subsubsection{Slovesné věty}
\begin{center}
\begin{tabular}{|c|c|c|c|}
\hline
ばら&を&3本&買いました。\\
\hline
Objekt&o&množství&sloveso\\
\hline
\end{tabular}
\end{center}
Koupil jsem 3 růže.

Ve slovesné větě je množstevní slovo dává mezi (podstatné jméno+partikule) a sloveso. Množstevní slova neberou partikuli を.

a) りんごを2つ買いました。Koupil jsem dvě jablka.


\subsection{Demonstrativa: この その あの どの}

この その あの a どの jsou demonstrativa která jsou vždy následována podstatným jménem jako v  このほん "tahle kniha" そのりんご "tamto jablko" あの店 "támhle ten obchod" a  どのばら "která růže".

porovnej:
a) この本は千円です。Takhle kniha stojí 1000 jenů.

b) これは千円です。Tohle stojí 1000 jenů.




