\section{18. lekce}

\subsection{〜てもいい: Povolení}
\begin{tabular}{|c|c|}
\hline
ここにおいても &いいです。\\
\hline
V/A/N te mo&ii\\
\hline
\end{tabular}
Můžeš to položit sem.

Když se te-forma sloves, i-adjektiv nebo  [na-adj./noun + だ] je následována も a いい “dobrý," znamená to "je v pořádku i když ---"

a) 辞書を使ってもいいです。Můžete používat slovníky. (doslova: Je v pořádku pokud budete používat slovníky)

b) 練習の時間は少なくてもいいです。Je to v pořádku i když trénuješ jen krátce.

c) 少し不便でもいいです。Je to v pořádku i když te není úplně vhodné.

d) 申し込みは来週でもいいです。Je to v pořádku i když podáte přihlášku až příští týden.

\subsection{〜てから: Až bude/udělám}
\begin{tabular}{|c|c|c|}
\hline
じゅぎょうが 終わって & から、&成田へ 行きます。\\
\hline
V te &kara&\multirow{2}{*}{Hlavní klauzule}\\
\multicolumn{2}{c}{Sub-klauzule}&\\
\hline
\end{tabular}
Až skončí hodina, pojedu do Narity.

Když za te-formou slovesa následuje 〜から , indikuje to  "poté co (někdo) udělá něco"
Jelikož te-forma nevyjadřuje čas sama o sobě, čas tekara-klauzule je určen časem hlavní klauzule.
a) 私は部屋を かたづけてから、べんきょうします。 Budu se učit až si uklídím pokoj.

b) うちに帰ってから、友だちに電話しました。Zavolal jsem kamarádovi poté co jsem přišel domů.


Jak bylo vysvětleno v 17. lekci, když subjekt sub-klauzule je jiný než ten hlavní klauzule, měl by být označem partikulí が, a ne は. Subjekt hlavní klauzule je změněn na téma celé věty když je označen  は, a může být umístěn buď na začátku věty nebo na začátku hlavní klauzule.
Porovnejte :
a) 友だちが 帰ってから、私は手紙を書きました。Poté co kamarád odešel, napsal jsem dopis.

b) 私は帰ってから、手紙を書きました。Poté co jsem se vrátil domů, napsal jsem dopis.

\subsection{ 〜てくる (1) : Udělat ~ a Přijít}
\begin{tabular}{|c|c|}
\hline
お弁当を買って&きました。\\
\hline
Vte&kuru\\
\hline
\end{tabular}
Koupil jsem si oběd a přišel jsem.


Když くる “přijít"  je předcházeno te-formou slovesa, může to znamenat "udělat -- a přijít sem." Když 〜てくる je použito v minulém čase jako v 〜てきました, často to indikuje "udělal -- než jsem přišel sem"


a) 国では日本語をべんきょうして きませんでした。Do Japonska jsem nepřišel poté co jsem studoval Japonštinu.

b) 
A: 奨学金はもう申し込みましたか。Už jsi se přihlásil na stipendium?

B: ええ、事務所で申し込んできました。Ano, přihlásil jsem se v kanceláři (než jsem přišel sem).

c) 
先生: しゅくだいはやってきましたか。Udělali jste domácí úkol (než jste přišli sem)?

学生: ええと、来週までにやってきます。No... udělám to příští týden (než sem přijdu).

\subsection{ Klauzule modifikující podstatná jména (1)}
\begin{tabular}{|c|c|}
アメリカから来る&友だち\\
\hline
V plain (Modifikujííc klauzule)&Podstatné jméno\\
\hline
\end{tabular}
kamarád pocházející z USA

\subsubsection{1}
Stejně jako [noun + の] a adjectiva modifikují podstatná jména, klauzule končící slovesem také může modifikovat podstatné jméno. Sloveso v modifikující klauzuli musí být v plané formě. 

(1) Modifikující klauzule musí být před podstatným jménem.

(2) V japonšitně nejsou ekvivalenty pro relativní zájmena jako "který" atd. 

\subsubsection{2}.
Mezi různými typy modifikikujících klauzulí, jeden typ je tento právě představený. Subjekt (nebo téma) věty se stane mofikovaným podstatným jménem a zbytek věty se stane modifikujícíc klauzulí.

友だちがメキシコから来ました。---> [メキシコから来た] 友だち
Kamarád přijel z Mexika.  ––> kamarád který přijel z Mexika

ケーキはれいぞうこの中にあります。 ––>[れいぞうこの中にある] ケーキ
Koláč je v lednici. ––> koláč, který je v lednici.


先生は 日本史を教えています。–––> [日本史を教えている] 先生
Učitel učí Japonskou historii ––> učitel, který učí Japonskou historii



ひと “osoba/lidi" a もの "(hmotné) věci" také mohou být modifikované klauzulíy

毎日 ジョギングをする人 lidé co běhají každá den

私の部屋にあるもの Věci, které jsou u mě v pokoji


\subsubsection{3.}
 [Modifikující kaluzule + noun] funguje jako (noun) fráze ve věte a může být použitá jako téma, objekt atd. pokud je následována partikulí は、を atd. nebo jako predikát pokud je následováno  ~です.
a) きのう アメリカから来た友だちは今シカゴではたらいています。Kamardá který přijel ze států pracuje v Chicagu.


b) 中国に留学する 友だちを成田へ 送りに行きます。Jedu do Narity vidět se s kamarádem kdo bude studovat v Číně.

c) こちらは 法学部で教えていらっしゃる村田先生です。Tohle je profesor Murata, který účí na právech.

の “one(s)” může nahradit modifikované podstatné jméno pokud je to pochopitelné z kontextu (neplatí to ale pro osoby)

そこにあるのを見せてください。 Prosím ukaž mi tento támhle.



\subsection{Partikule に: Statická lokace (2)}
Slovesa jako  おく “strčit,” とまる “přespat," たつ “stát," すわる “sedět," とまる
“zastavit,” かく "psát,” つとめる “být zaměstnán," atd. mají partikuli に což indikuje statickou lokaci.
a) 私はいすの上にかばんをおきました。Položil jsem můj batoh na židli.


b) 両親は先週 箱根のホテルに泊まりました。 Moji rodiče zůstali přes noc v hotelu v Hakone minulý týden.


\subsection{Konjunkce: Tedy/tak}
それで "tedy,” “proto” spojuje dvě věty. Je to používané když první věta vyjadřuje důvod nebo příčiny pro druhou větu.

a) きのうは休みでした。 それで、私たちは映画を見に行きました。Včera bylo volno. Tak jsme šli do kina.

b) A: ケーキを買ったんですか。Koupil jsi koláč?


B: ええ。 きょうは友だちのたんじょうび なんです。 それで、パーティーをするんです。Ano, dneska měl kamarád narozeniny. Tedy jsme měli párty.


それで je také používané na začátku tvrzení, pokud se navazuje na předchozí tvrzení.

A: あのう、今時間がないんです。Víte, teď nemám čas.

B: そうですか。 Jasný.
A: それで、 あしたでもいいですか。Ale, bylo by to zítra v pohodě?



\subsection{ Tázací slovo + か / Tázací slovo + も}
1. Tázací slovo následované  か znamená  “nějaké" jako v だれか “někdo," なにか“něco," a どこか “někde." Když  [Tázací slovo + か ] je používané jako subjekt nebo objekt, が a を může být přidáno. Ostatní partikule, jako に と, で, atd. jsou přidané po [Tázací slovo+か] když používané ve větě.
a) 台所でだれか(が) 料理しています。Někdo vaří v kuchyni.

b)アミンさんはだれかと食事に行きました。Pan Amin šel s někým na oběd.


c) 私は前どこかで山下さんに会いました。Paní Yamashitu jsem už někde viděl.


2.Tázací slovo následované も je používané v negativních větách a znamená "žádné" jako だれも〜ない “nikdo,”  なにも〜ない “nic” a どこも〜ない “nikde”. Když je [Tázací slovo + も ] použito jako subjekt nebo objekt,  が a を nemůžou být přidané. Ostatní partikule jako  に, と, で, へ, atd. jsou vloženy mezi Tázací slovo a  も jako v だれとも “s nikým” a どこに も“nikde." へ v どこへも je často vynecháno v mluvě.
a) 気分がわるかったので、何も食べませんでした。Necítil jsem se dobře, takže jsem nic nejedl.

b) 私はだれとも話しませんでした。S nikým jsem nemluvil.

c)
 A: カメラはどこかにありましたか。 Našel si tu kameru někde?
B: いえ、 どこにもありませんでした。Ne, nikde jsem ji nemohl nejít

\subsection{Různé}


おおい “mnoho" a すくない “trochu" jsou oboje i-adjektiva, ale obě se chovají jinak než ostatní i-adjektiva. Když se vyjadřuje množství おおい a すくない jsou často dávány na místo  たくさんいる/ある "je jich hodně ~" a あまりいな いない “není jich hodně ~ ". おおい a すくない jsou používané pouze jako predikáty a nemůžou být použité pro přímou modifikaci podstatných jmen jako  おおい がくせい.
このごろは車を持っている学生が多いです。Touhle dobou je hodně studentů co mají auta.






