\section{20. lekce}
\label{sec:lekce_20}

\subsection{Porovnávající věty}
\subsubsection{Tvrzení}

\begin{center}
\begin{tabular}{|c|c|c|c|c|}
\hline
バス&より&電車&のほうが&はやいです。\\
\hline
Noun1&yori&Noun2&no hoo ga&Predikát\\
\hline
\end{tabular}
\end{center}

 Vlak je rychlejší než autobus.


“N2 je více než N1" je vyjádřeno jako  N1 より N2 のほうが~.  より, což koresponduje s "než" je partikule používaná na porovnávání. Pořadí N1 a N2 může být obrácen jako v N2のほうがN1 より ~. Japonská adjektiva nemají komparativ. Je možné vynechat  N1 より nebo N2のほうが když je to pochopitelné z kontextu.

a) タイプライターより ワープロのほうが べんりです。Word procesor je více vhodný než psací stroj.

b) 公園は土曜日も日曜日もこんでいます。でも、土曜のほうがすこしすいています。V parku je plno lidí jak v sobotu tak i v neděli. V sobotu je však trošku méně lidí než v neděli.

c) A: 会話の試験はむずかしかったですね。Test z konverzace byl těžký, že?

B: ええ、漢字の試験よりむずかしかった ですね。Ano, byl těžší než test z kanji.

\subsubsection{Tázací věta}
\begin{center}
\begin{tabular}{|c|c|c|c|c|c|c|}
\hline
電車&と& バス&と、&どちら/ どっち&が& はやいですか。\\
\hline
Noun1 &to&Noun2&to&dochira/dotchi&ga&Predikát (otázka)\\
\hline
\end{tabular}
\end{center}
Co je rychlejší, vlak nebo autobus?

"Co je víc ~, N1 or N2?" je vyjádřeno s použitím N1 とN1とどちら/ どっちが
~ですか/ますか. Partikule  と  je přidána po každé ze dvou věcí které se porovnávají. どっち je méně formální než どちら.


a) A: 上野と 新宿と、 どっちが近いですか。Co je blíže, Ueno nebo Shinjuku?

B:新宿のほうが近いと思います。Myslím že Shinjuku je blíže.

b) A: 地下鉄とバスと、どっちがこんで(い) ますか。 Co je více zaplněné, metro nebo autobus?

B:地下鉄のほうがこんで(い) ます。Více lidí je v metru.

\subsection{〜たことがある: Zažil jsem}
\begin{center}
\begin{tabular}{|c|c|}
\hline
そこへ行った &ことがあります。\\
\hline
Vta&koto ga aru\\
\hline
\end{tabular}
\end{center}
Byl jsem tam.

ことがある předcházeno ta-formou slovesa je používáno pro vyjádření něčí zkušenost což koresponduje s "(někdo) udělal--." Negativní tvar  ~たことがある/あります  je 〜た ことがない/ありません.

a) 私は国で日本の映画を見たことが あります。Viděl jsem u nás Japonský film.

b) 寄田さんに会ったことはありません。 Nikdy jsem nepotkal pana Yasudu.

\subsection{ 〜ている (3): Výsledný stav (2)}
Existují skupiny sloves které vyjadřují měnící se stav. Když je te-forma slovesa následována いる, indikuje to současný stav, který je výsledkem ukončené změny.

こむ stat se plným ––> こんでいるbýt zaplněný

すく vyprázdnit se ––> すいている vyprázdněno

a) 地下鉄はいつもこんでいますが、 けさはすいていました。V metru je vždy plno, ale dnes ráno ne tolik.
b) A: どうしたんですか。 気分がわるいんですか。Co je? Není ti dobře?

B: ええ。 かぜをひいて(い) るんです。Ne, mám rýmu.


\subsection{ 〜て: způsoby}
Te-forma sloves může být použita způsob jakým někdo něco udělal. “Jak" jako v  “Jak uděláš --?" je vyjádřeno pomocí  どうやって.

a) 私はテープを聞いて勉強しました。Studoval jsem tím že jsem poslouchal nahrávku.
b) A: 駅までどうやって行くんですか。Jak se dostáváš na stanici.

B:歩いて行きます。Chodím pěšky.

c) A: これ、どうやって食べるんですか。 Jak se to jí?

B:ソースをかけて食べるんです。Dáš na to omáčku a sníš to.

Toto použití te-formy je spojeno s partikulí で indikující způsoby. Slovní fráze končící te-formou, jako  ~にのって “nastoupení/řízení -" a 〜をつかって "používaje ~," může být občas nahrazeno s [noun + で].
Porovnejte:
a) バスに乗って 学校へ行きます。Jezdím autobusem do školy.
b) バスで学校へ行きます。Jezdím do školy autobusem.



\subsection{ 〜ないで: Místo toho aby ~/Bez toho aby}
\begin{center}
\begin{tabular}{|c|c|}
\hline
バスに乗らないで、&電車で行きます。\\
\hline
V naide&Verb\\
\hline
\end{tabular}
\end{center}

Místo toho abych jel autobusem, jel jsem vlakem


Když je nai-forma sloves následována  で, má to jedno ze dvou možných významů a použití v závislosti na kontextu
\begin{enumerate}
\item “místo toho abys---"
\item “bez toho abys udělal---"
\end{enumerate}

a) 北川さんは授業に出ないで、友だちとコーヒーを飲みに行きました。Místo toho by šla na hodinu tak měla paní Kitagawa kafe s kamarádkou.

b) 私はげんこうを見ないでスピーチをしました。Měl jsem řeč bez toho abych se podíval na text.



\subsection{Partikule}

\subsection{ で: Množství}
で, když je použité s podstatným jménem indikujícím množství (čas, peníze atd.) tak koresponduje s "v jasném čase" nebo "pro (nějaké množství)."

a) 30分ぐらいで渋谷に着くと思います。Myslím že do Shibuyi dorazíme přibližně do 30 minut.

b) 私は2万円でこの本だなを買いました。Tuhle poličku jsem koupil za 20 000 jenů.


