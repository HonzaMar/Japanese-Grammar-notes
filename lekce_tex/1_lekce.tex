\section{1. lekce}
\label{sec:lekce_1}

\subsection{Nominální věty: $N_1$ wa $N_2$ desu}

\subsubsection{Pozitivní}

\begin{center}
\begin{tabular}{||c|c||c|c||}
\hline
ジムハリスさん&は&がくせい&です。\\
\hline
Noun 1&wa&Noun 2&desu\\
\hline
\end{tabular}
\end{center}
Jim Harris je student.

Wa je partikule používaná jako zvýrazňovač tématu a indikuje že předcházející podstatné jméno je téma věty. Signalizuje o čem se chystá mluvčí mluvit. Desu které předchází podstatné jméno poskytuje informaci ohledně tématu. Desu jde vždy na konec věty.

a) カレン ロペズさん は アメリカンじんです。Karen Lopézová je Američanka.

\subsubsection{Negativní}
\begin{center}
\begin{tabular}{||c|c||c|c||}
\hline
やまださん&は&がくせい&では/じゃ ありません。\\
\hline
Noun 1&wa&Noun 2&dewa/ja arimasen\\
\hline
\end{tabular}
\end{center}
Paní Jamada není student.

Negativ  $N_1$ wa $N_2$ desu je $N_1$ wa $N_2$ dewa arimasen, znamenající "$N_1$ není $N_2$". Dewa je zkráceno na "ja" jako v ja arimasen, které je ekvivalentní dewa arimasen. V psané formě se používá pouze dewa arimasen.


a)キムさんはせんせいではありません。がくせいです。 Pan Kim není učitel. Je student.




\subsection{Otazník: Ka}
\begin{center}
\begin{tabular}{||c|c||c|c||c||}
\hline
ジムさん&は&がくせい&です&か\\
\hline
Noun 1&wa&Noun 2&desu&ka\\
\hline
\end{tabular}
\end{center}
Je Jim student?

Ano/ne otázka jako "Je Jim student?" v japonštině může být vytvořena jednoduše přidáním otazníku ka k desu. Ka je větná partikule která může být položena na konec věty.

a)もりさんはせんせいですか。 Je pan Mori učitel?


\subsection{Partikule}

\subsubsection{Wa} Zmíněno v lekci 1 (budou odkazy na gramatické poznámky v různých lekcích)
\subsubsection{Mo: Také}

\begin{center}
\begin{tabular}{||c|c||c|c||}
\hline
わたし&も&がくせい&です。\\
\hline
Noun 1&mo&Noun 2&desu\\
\hline
\end{tabular}
\end{center}
Také jsem student.

\begin{center}
\begin{tabular}{||c|c||c|c||}
\hline
わたし&も&がくせい&では/じゃ ありません。\\
\hline
Noun 1&mo&Noun1 &dewa/ja arimasen\\
\hline
\end{tabular}
\end{center}
Také nejsem student.

Když je partikule wa nahrazena partikulí mo v pozitivní větě, je přidán význam "také". Mo nahrazující wa v negativní větě přidává význam "také ne". 


a) わたしはがくせいです。ハリスさんもがくせいです。Jsem student. Pan Harris je také student.


\subsubsection{No: modifikace podstatného jména}
\begin{center}
\begin{tabular}{||c|c||c||}
\hline
わせだ &の &がくせい\\
\hline
Noun 1&no&Noun 2\\
\hline
\end{tabular}
\end{center}
Student Wasedy

Podstatné jméno následované partikulí no může modifikovat následující podstatné jméno, předchozí podstatné jméno popisující pozdější následující podstatné jméno. 


ほうがくぶのせんせい učitel na právnické fakultě

\subsection{Vynechání jasného témata}
Téma je často vynecháno pokud je jasné z kontextu. (V češtině zrovna docela přirozené)

\subsubsection{Předem oznámené téma}
Jakmile je téma představeno, je vynecháváno v následujících větách. Pokud ovšem není potřeba dané téma zvýraznit.

ロペスさんだいがくのともだちです。こくさいぶのがくせいです。Paní Lopezové je kamarádka z vysoké školy. Je studentkou mezinárodní divize.

\subsubsection{Zřejmé téma}

Pokud je zřejmé ze situace o čem mluvčí mluví, téma následované wa jako watashi wa je často vynecháno, i přes to že nebylo zmíněné před tím.


ジムハリスです。はじめまして。 Jsem Jim Harris. Jak se máte?




























