\section{38. lekce}
\label{sec:lekce_38}

\subsection{Kauzativní forma sloves}

Když vyjadřujeme "nechat někoho něco udělat" tak se používá kauzativní forma sloves. Kauzativní forma se získává za pomocí daných pravidel.

1) 1だん
místo 〜る dáme 〜させる

みる → 見させる

かえる → 帰させる


2) 5だん

Přidáme 〜せる k jádru nai-formy  (část před ない v nai-formě) slovesa

\begin{center}
\begin{tabular}{ccc}
かく → かかせる&まつ → またせる&のむ → のませる\\
いそぐ → いそがせる&しぬ → しなせる&かえる → かえらせる\\
はなす → はなさせる&あそぶ → あそばせる&かう→かわせる\\
\end{tabular}
\end{center}



3) Nepravidelná slovesa
する → させる

くる → こさせる

\subsection{Kauzativní věty}
Kauzativní forma slovesa prakticky znamená "nechat někoho něco udělat". V závislosti na kontextu, může to také indikovat povolení, "povolit někomu něco udělat". Tedy じゅくにいかせる může znamenat jak "donutit jít děcka na doučování" nebo "nechat děti jít na doučování"

\subsubsection{Kauzativní věty s tranzitivními slovesy}

\begin{center}
\begin{tabular}{||c|c||c|c||c|c||c||}
\hline
先生&は/が&学生&に&じしょ&を&もって来させました。\\
\hline
Původce&wa/ga&účinkující&ni&objekt&o&Vt klauzative\\
\hline
\end{tabular}
\end{center}

Učitel nechal studenty přinést jejich slovníky.

Když věta s tranzitivním slovesem je změněná do kauzativní věty, osoba která je nechána aby něco udělala je označená に

みなみさんがちょうさのけっかをほうこくした →  ぶちょうはみなみさんにちょうさのけっかをほうこくさせた

Pan minami ohlásil výsledky výzkumu.         šéf oddělení nechal pana Minamiho ohlásit výsledky výzkumu


\subsubsection{Kauzativní věty s intranzitivními slovesy}

\begin{center}
\begin{tabular}{||c|c||c|c||c||}
\hline
父&は/が&弟&を&ぎんこうへいかせました。\\
\hline
Původce&wa/ga&účinkující&o&V int - kauzativní\\
\hline
\end{tabular}
\end{center}
Můj táta poslal mladšího bratra do banky. 

Když se věta s intranzitivním slovesem jako いく"go" はしる"běžet" nebo かえる"vrátit se domů"  je změněna do kauzativní věty, osoba která je donucena něco udělat je označena を


まいあさせんしゅが走っている   →  コーチがまいあさせんしゅを走らせている

Hráči běhají každé ráno 				každé ráno je trenér nutí běhat 


\subsubsection{〜(さ)せる a   〜てもらう}
Obojí jak (さ)せる  (kauzativní forma sloves) a 〜てもらう by mohlo bát přeloženo "donutit někoho něco udělat". Ale, kauzativní forma slovesa jsou používaná pouze rodiče donutí své děti něco udělat, neboli nadřízení donutili podřízené něco udělat. Na druhou stranou 〜てもらう indikuje že jeden něco požaduje bez ohledu na vztah mezi žadatelem a účinkujícím. 


\subsection{〜(き)せてもらう/〜(さ)せていただく: Mít povoleno udělat ---}

Když se slovesa v kauzativní formě zkombinují s 〜てもらう/〜ていただく (Lekce 25), doslova to znamená "obdržet laskavost povolení udělat ---". Často je to používáno když zdvořile žádám o povolení. 


a) きゅうようができたので、明日のかいぎを休ませてもらうことにしました。Rozhodl jsem se neúčastnit zítřejšího meetingu jelikož jiná naléhavá záležitost se objevila.


\subsection{Zdvořilost: přehled}

\subsubsection{Výrazy s respektem: そんけいご}

\begin{enumerate}
\item Slovesa s respektem いらっしゃる、なさる、おっしゃる、くださる
\item Respektující forma sloves, která nemají ekvivalent (お+Vstem + になる ),(  ご/お+N(+する)+なさる) v Lekci 33
\item Respektující rozkaz (お+Vstem +ください) (ご/お+N(+する)+ください) v Lekci 36
\item Honorující prefixy/sufixy indikující respekt ごかぞく、おなまえ (lekce 11) a 〜さん、〜の方 (lekce 1)
\item 〜ていらっしゃる/〜ていらっしゃいます jsou používány na  〜ている/〜ています(lekce 14)
\item 〜でいらっしゃる/〜でいらっしゃいます jsou používány pro 〜だ/〜です za na-adjektivy a podstatnými jmény které referují osoby
\end{enumerate}

a)高山さんでいらっしゃいますか Je tam pan Takajama? / Jste pan Takajama?

b)の本の方でいらっしゃいますか。 Jste z Japonska?

\subsubsection{Pokorné výrazy けんじょうご}
Pokorné výrazy jsou používané pro vyjádření mluvčího (případně jeho skupiny) akce a stav.
\begin{enumerate}
\item pokorná slovesa おる、まいる、いたす、うかがう、おめにかかる
\item pokorná forma sloves která nemají pokorný ekvivalent( お+Vstem+する)( ご/お+N(+する)+する) lekce 27
\item Pokorný požadavek (お+Vstem+いただけますか ),( ご/お+N(+する)+いただけますか) Lekce 36
\item 〜ております je používáno pro  〜ています (lekce 36)
\item 〜でございます je pokorná ekvivalent  〜です následující na-adjektiva a podstatná jména
\end{enumerate}
じんじかでございます。Toto je personální oddělení.


\subsubsection{Kultivované výrazy ていねいご}
Kultivované výrazy nemusí nutně referovat na nadřízené, nebo něco s nimi spojené. Kultivované výrazy mohou způsobit že vaše mluva zní více kultivovaně a dovolí vám zároveň projevovat respekt k ostatním.
\begin{enumerate}
\item uctivý prefix お používaný pro kultivovanost おはし おみず  (lekce 12)
\item Slova používaná v kultivované řeči こちら、そちら、あちら、どちら (lekce 7)
\item 〜でございます je používaný pro  〜です následující na-adjektiva a podstatná jména
\item 〜ございます、まいります a 〜ております jsou používané pro  ある、くる/いく respektive 〜ている. Tyto výrazy mohou být slyšeny v hlášení zákazníkům, cestujícím atd.
\item 〜です/〜ます může být použito v sub-klauzulích a klauzulích modifikující podstatná jména jako v  〜でして、〜でしたら、〜まして、〜ましたら、〜ます/〜ました (+noun)
\end{enumerate}
a) そういうしりょうでしたら、としょうかんにあります。Pokud mluvíš o těchto materiálech, tak jsou v knihovně.





















