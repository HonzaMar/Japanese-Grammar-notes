\section{22 lekce}

\subsection{Věty s přídavnými jmény vyjadřující preferenci/schopnost}

\begin{tabular}{||c|c||c|c||c||}
\hline
私&は&音楽&が&好きです。\\
\hline
Osoba&wa&podtstané jméno&ga&přídavné jméno\\
\hline
\end{tabular}

\subsubsection{} 好き (な)a na-adjektivum znamenající "oblíbený/mít rád" bere partikuli が  na vyznačení co má někdo rád. Seznam všech adjektiv vyjadřující něčí preferenci, schopnost nebo přání je níže, všechny berou が

\begin{tabular}{cccc}
好き(な)&oblíbený/mít rád&きらい(な)&nemít rád\\
大好きな&nejoblíbenější&だいきらい(な)&nesnášet\\
上手(な)&dobrý/šikovný&へた (な)&nešikovný\\
とくい(な)&oblíbený&にがて (な)&nešikovný/slabý v\\
ほしい&chtěný&&\\
\end{tabular}

a)ホセさんはフランス語が上手です。 Pan José je dobrý ve Francoužštině.

Když se 〜たい  "chtít udělat" skombinuje se slovesem který potřebuje objekt jako například よむ  "číst", partikule が  se normálně používa na označení toho co chce mluvčí udělat, ale を je také možné použít.


\subsubsection{} Věty s adjektivy které berou が  moho být změněny na klauzule modifikující podstatná jména jako níže. Pozn. že je přídáno za na-adjektivum když jde před podstatné jméno. 

犬がだいきらいな川村さん Paní Kawamura, která nesnáší psy

\subsection{Superlativní věty "nej"}
\begin{tabular}{||c|c||c|c||c||c||}
\hline
音楽& のなかで& ジャズ &が &いちばん& 好きです。\\
\hline
Podstatné jméno&(no naka) de& Podstatné jméno& ga& ichiban&predikát\\
\hline
\end{tabular}
Ze všech typů hydby se mi nejvíce líbí jazz.

1。 いちばん "nejlepší (číslo jedna)" je používané pro vyjádření superlativu "nej--" když předchází adjektivu. Japonská adjektiva nemají tyto fotmy .で je použito na indikaci hranic ve kterých se porovnání koná. -のなかで znamená "mezi--" používané pro vyjádření skupiny lidí/věcí ve které se porovnání koná.

a) 私はスポーツの中でサッカーいちばん好きです、よくしあいを見に行きます。Ze všech sportů se mi nejvíce líbí fotbal, často chodím na zápasy.

2。"Co/kdo/kdy/který- je nejlepší?" lze vyjádřit なに/だれ/逸/ドレ/どの/ Nが いちばん 〜(ですか. どれ "který" nebo  どの N "který (z nich)" je používáno když je speicifkováno koho budou provnávat なに se používá když není specifikováno. Pozn. どっち se nemůže použít, proto že to vyjadřuje "který (ze dvou)", musí se použít どれ

a)
 A: 日本語の勉強でなにがいちばん大変ですか。  Co je nejtěší část na studium Japonštiny?
 
B: そうですねー 感じがいちばん大変ですね。 Řekněme....... Kanji je nejtěžší.


3。 いちばん může být použito pro modifikaci příslovce

クラスで英語がいちばん上手に話せる人は上田さんです。 V této třídě, osoba která umí nejlépe anglicky je pan Ueda.

\subsection*{〜の: Nominalyzér (1)}

\begin{tabular}{||c|c|c||c||}
\hline
うたをうたう& の& が& 好きです。\\
\hline
Klauzule (V dict)&no&&\\
\hline
\end{tabular}

の následující klauzuli která končí slovníkovou formou slovesa změní (klauzule+ の) na ekvivalent podstatného jména. (Klauzule+  の) může být použito ve větě jako níže. 

a) 弟はしゃしんをとるのが上手で、高校の写真部に入っています。Můj mladší bratr je dobrý fotograf a je členem fotografického kroužku ve škole.



\subsection{〜とき: Když --}
\begin{tabular}{||c|c||c||}
\hline
小さい & 時(に)& 始めました。\\
\hline
Planá forma & toki (ni)& Hlavní kaluzule\\
Sub& klauzule&\\
\hline
\end{tabular}

とき je podstatné jméno znamenající "čas" a  〜とき znamená "(v čase) kdy --." 〜とき následuje planou formu.  だ za na-adjektivem se změní na な a  だ po podstatném jménu se změní na の. Partikule に indikující specifický čas  v  〜とき není povinný používat.

V této lekci, toki-klauzule vyjadřující stav (tedy, 〜とき následující sloveso jako いる "existovat/být") bude představen. Toki-klauzule vyjadřující akci bude v pozdějších lekcích. Když toki-klauzule vyjadřuje stav, non-past forma je často používaná v toki-kaluzuli, a to i tehdy když hlavní klauzule končí v minuléí formě.

a) うちにいるときに、そのニュースを聞きました。Slyšel jsem zprávy když jsem byl doma.

Jak bylo vysvětleno v Lekci 17, když je subjekt sub-klauzule jiný než subjekt v hlavní klauzuli, je vždy označen partikulí  が	místo 	は. Subjekt hlavní klauzule je často změněn na téme celé věty a označen 	は.

Porovnej 

a)父がいそがしいとき、私はよく父の仕事を手伝います。 Když má můj otec hodně práce, tak mu často pomáhám s jeho prací.

b)父はいそがしいとき、日曜日にも会社へ行きます。 Když má můj otec hodně práce, chodí do práce i v něděli.


V otázce "Kdy jsi ---" obojí	どんな とき	"v jakou příležitost" a  いつ	"kdy" jsou používané.

a) A: 	どんなときに、ともだち電話するんですか。	Kdy voláš svým přátelům.

B: 友達の声が聞きたいとき、します。	Volám jim když chci slyšet jejich hlasy.

\subsection{Vたり Vたり する Dělat věci jako --}

\begin{tabular}{||c|c|c||}
\hline
歌を歌ったり&踊ったり&しました。\\
\hline
V tari & V tari& suru\\
\hline
\end{tabular}
Zpíval jsem a tancoval. (Doslova: Například jsem zpíval a tancoval"

Když jsou slovesa použita ve formě 〜たり 〜たり する		, znamená to "dšlat věci jako-- a ---". Je to používané když se vyjmenovávají akce jako příklady, implikujíce že byly dělány i jiné věci. 〜たり	je vytvořeno přídáním り	 k ta-formě slovesa.

休みの日はテニスをしたり、家で映画のビデオを見たりします。Ve svém volnu hraju tenis, koukám na filmy doma a další věci.

Nejčastěji je tato forma využívaná ve dvojicích, ovšem může být použita i víckrát.


\subsection{〜ながら Během ---}

\begin{tabular}{||c|c||c||}
\hline
音楽を聴き&ながら&勉強しました。\\
\hline
V stem&nagara&Verb\\
\hline
\end{tabular}

Studoval jsem během poslechu hudby.

Když stem-formu slovesa následuje 	〜ながら	, znamená to že zatím co někdo něco dělá, dělá během toho i něco jiného. Sloveso, které je ke konci věty, indikuje hlavní aktivitu. 	〜ながら	je používané se slovesy indikujícíc pokračující akci, nemůže být tedy používané se slovesy s cílovým pohybem, slovesy indikující stav, potenciální formou atd.


京子さんは昼間働きながら夜高校に通っています。	Během pracovních dnů, Kjóko chodí na střední v noci.






















