\section{23 lekce}
\label{sec:lekce_23}




\subsection{〜なくてはいけない/〜なくちゃいけない: Muset udělat --}


\begin{center}
\begin{tabular}{||c|c||}
\hline
明日までに書かなくては& いけません。\\
\hline
V/A/N nakute wa& ikenai\\
\hline
\end{tabular}
\end{center}

Musím to do zítra napsat.

〜なくてはいけない vyjadřuje nutnost nebo závazek, znamenajíce "muset --", je zformováno změnou nai-formy slovesa, nebo plain nezáporné formy i-adjektiv nebo (na-adj/noun + だ) na  〜なくて, které je následováno  は a zkombinováno s いけない. Důležité vědět že  ではない které následuje po na-adjektivech nebo podstatných jménech se transformuje na でなくてがいけない

〜なくちゃ いけない zkrácená forma, většinou používaná v mluvené řeči.

\begin{center}
\begin{tabular}{|c|c|c|}
\hline
Sloveso &書かなくては&\\
I-adjektivum&新しくなくては		&			いけない\\
Na-adjektivum&きれいでなくては&\\
Podstatné jméno&今日でなくては  &\\
\hline  
\end{tabular}
\end{center}


a) 毎日新し言葉をたくさん覚えなくてはいけないので、大変です。Je to náročné, protože si musím zapamatovat mnoho nových slov každý den.

\subsection{ーてくる: Jít a udělat (a vrátit se zpět)}
\begin{center}
\begin{tabular}{||c|c||}
\hline
生協で買って& きます。\\
\hline
&\\
\hline
\end{tabular}
\end{center}
Půjdu a koupím to v obchodě. (a vrátím se)

〜てくる může znamenat "jít a udělat --- a vrátit se zpět" a může být použito když někdo jde někam něco udělat a vrátí se zpět na originální místo kde se momentálně nachází.

a)
A: バスの時刻表は事務所にありますよ。Jízdní řád autobusu je v kanceláři.

B: そうですか。じゃあ、見てきます。Opravdu? Dobře, tak já půjdu a podívám se.

\subsection{Klauzule modifikující podstatné jméno (2)}
\begin{center}
\begin{tabular}{|c|c|}
書いている &レポート\\
\hline
V plain&Podstatné jméno\\
Modifikující klauzule&\\
\hline
\end{tabular}
\end{center}

1. Stejně jako subjekt (téma) se mohou stát modifikovaným podstatným jménem (Lekce 18), objekt se tím také může stát, zbytek věty se transformuje do modifikující klauzule.

いつもワープロを使います。 →         (いつも使う)ワープロ
Vždy používám word procesor			Word procesor který vždy používám


もの "věc" a  こと "věc (nehmatatelná)" mohou být také modifikovány klauzulemi jako 

今集めているもの to co teď sbírám
図書館で調べたこと  Co jsem hledal v knihovně

2. Subjekt klauzule modifikující podstatné jméno by měl být vyznačen が nikoli は

私が授業で習ったかんじ kanji které jsem se naučil na hodině

Subjekt klauzule který je stejný jako subjekt celé věty většinou není zmiňován jako v případě a) níže. Když jsou dané subjekty jiné, jsou zmíněné jako v příkladě b)

Porovnej:

a) 妹は描いた絵を友達に見せました。Moje mladší sestra ukázala jejím kamarádům obrázek, který namalovala.

b) 妹は私が描いた絵を友達に見せました。Moje mladší sestra ukázala kamarádům obrázek který jsem namaloval.

が které označuje subjekt klauzule se může změnit na の

私のかいた絵 obrázek který jsem namaloval

3. S větami které používají adjektiva jako 好き(な)"oblíbený, líbit se, milovaný" きらい(な) "nemít rád"  mohou se stát modifikovanými podstatnými jmény jako níže. Důležité že な se objevuje když na-adjektivum předchází podstatné jméno.

子供が好きなまんが Komiksy které mají děti rády



\subsection{Suffix: 〜じゅうに/〜ちゅうに: V}
〜じゅうに/〜ちゅうに je suffix který znamená "v (specifická perioda)"  
今日中に ke konci dne

\subsection{Tázací slovo+でも: jakýkoli}

Když se tázací zájmeno/slovo jako なん、どれ、どこ je následováno partikulí でも, znamená to jakýkoli

いつでも kdykoli
どこでも kdekoli
だれでも kdokoli
なんでも cokoli
どれでも jakýkoli
どちらでも kterýkoli


a) 兄はよく本を読むので、なんでも知っています。Jelikož můj starší bratr hodně čte, víc skoro cokoli.

\subsection{Tázací zájmeno +か / Tázací zájmeno + も}
(Tázací zájmeno +か) jako だれか "někdo"  nebo なにか "něco" a   (Tázací zájmeno+も) spojené s negativem jako だれも"nikdo" nebo なにも "nic" můžou být před podstatným jménem jako níže

a) 私はカメラをもっていないので、だれかともだちに借ります。Jelikož nemám kameru, půjčím si jednu od kamaráda.































