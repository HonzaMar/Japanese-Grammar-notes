\section{29. lekce}
\label{sec:lekce_29}


\subsection{Tara-forma}
Když se vyjadřuje podmínka jako v "Pokud ..., tak ...." tak je použita tara-forma v sub-klauzuli. Tara-forma může být vytvořena přidáním ら k plané minulé formě slovesa, i-adjektiva nebo  [na-adj./noun + だ/です].


\begin{center}
\begin{tabular}{|| c || c ||}
\hline
1だん&みる  → みたら \\
 &みない → みなかったら \\
\hline
5だん&よむ → よんだら\\
&よまない → よまなかったら\\
\hline
Nepravidelná &する → したら \\
&しない → しなかったら\\
&くる → きたら\\
&こない → こなかったら\\
\hline
I Adjectiva&あつい → あつかったら \\
&あつくない → あつくなかったら\\
\hline
Na Adjectiva&きれいだ → きれいだったら \\
&きれいではない → きれいでなかったら\\
&きれいじゃない → きれいじゃなかったら\\
\hline
Podstatná jména&びょうきだ → びょうきだったら\\
&びょうきではない → びょうきでなかったら\\
&びょうきじゃない → びょうきじゃなかったら\\
\hline
\end{tabular}
\end{center}

\subsection{〜たら: Kondicionál (pokud --)}

\begin{center}
\begin{tabular}{|| c || c ||}
\hline
熱(ねつ)が下がったら、&学校へ行きます。\\
\hline
V/A/N tara,& Hlavní klauzule\\
Sub-klauzule&\\
\hline
\end{tabular}
\end{center}
Pokud se mi sníží teplota, půjdu do školy.

\subsubsection{Tara-klauzule}
Když se tara-forma slovesa, adjektiva nebo podstatného jména použije v sub-klauzuli, může to znamenat "pokud je určitá podmínka splněna, pak ..."


a) 母は、もし熱(ねつ)が下がらなかったら、医者(いしゃ)にみてもらったほうがいいと言いました。Moje máma řekla že pokud se mi nesníží teplota, měl bych jít k doktorovi.


\subsubsection{Subjekty Tara-klauzule a hlavní klauzule}
Ja je zmíněno v lekci \ref{sec:lekce_17} a lekci \ref{sec:lekce_18}, když se subjekty sub-klaiuzule a hlavní klauzule neshodují, subjekt sub-klauzule je vždy označen partikulí が. Zatím co když se shodují, většinou se z něho stane téma celé věty označené は.

\subsection{〜ても: I když}

\begin{center}
\begin{tabular}{|| c || c ||}
\hline
熱(ねつ)が下がっても& 学校へ行きません。\\
\hline
V/A/N te mo,& hlavní klauzule\\
sub-klauzule&\\
\hline
\end{tabular}
\end{center}
I když mi klesne teplota, nepůjdu do školy.


〜ても fomuje klauzuli která vyjadřuje podmíněný význam "i kdyby/když ..." narozdíl od 〜たら což znamená "pokud." ~ても je vytvořeno přidáním も k te-formě slovesa, i-adjektiva nebo  [na-adj/noun + だ]
a) 今から急いで準備(じゅんび)しても、飛行機(ひこうき)の出発( しゅっぱつ)の時間には間に合わないかもしれません。I kdybych si teď začal balit, nemusel bych odlet stihnout.

b) 山内さんは、給料(きゅうりょう)がよくても、銀行(ぎんこう)にはつとめたくないと言っていました。Paní Yamauchi řekla že nechce pracovat v bance i kdyby byla mzda dobrá.




c) A: ただでも、コンサートに行かないんで すか。Nešel bys na ten koncert ani kdyby byl zadarmo?

B: あ、ただだったら、行きます。Ah, pokud by byl zdarma, tak bych šel.


\subsection{ ~ようだ/~みたいだ: Vypadá to jako  (jako kdyby)}
Jak 〜ようだ tak 〜みたいだ znamenají "vypadá to jako  ~," a indikují domněnku který je založená na mluvčího subjektivního soudu.

\subsubsection{ ~ようだ}
\begin{center}
\begin{tabular}{|c|c|}
\hline
かぜをひいた&ようです。\\
\hline
Planá forma&yoo da\\
\hline
\end{tabular}
\end{center}
Vypadá to že mám rýmu.




 ~ようだ následuje za planou formou.  だ následující na-adjektivum by mělo být změněno na な a だ za podstatnými jmény by mělo být změněno na  の.
a) スミスさんはしばらく授業(じゅぎょう)に来られない ようです。Vypadá to že pan Smith teď chvilku nebude chodit na hodinu.

b) A: あしたは天気(てんき)が悪(わる)そうですね。Očividně nebude zítra vůbec hezký počasí.


B: ええ、 一日中雨(いちにちじゅうあめ)のようですね。 Ano, vypadá to že asi bude pršet celý den.




\subsubsection{ ~みたいだ}
\begin{center}
\begin{tabular}{|c|c|}
\hline
かぜをひいた&みたいです。 \\
\hline
Planá forma&mitai da\\
\hline
\end{tabular}
\end{center}
Vypadá to že mám rýmu.


〜みたいだ následuje planou formu.  だ, な nebo の nejsou přidávány po na-adjektivech a podstatných jménech. ~みたいだ je hlavně používané v mluvě, zatím co  ~ようだ  se používají jak v mluvě tak i psaném projevu.

a) A: どうしたんですか。Co se děje?

B: ねつがあるみたいなんです。Myslím že mám horečku.


\subsubsection{〜ようだ/〜みたいだ a 〜らしい}

〜ようだ/〜みたいだ a 〜らしい (Lekce 28) indikují mluvčího názor nebo úsudek.  〜ようだ/〜みたいだ jsou používané pokud mluvčího názor je založen na tom co opravdu pozoroval. Zatím co   〜らしい je využívané pokud je mluvčího názor získán nepřímo.



\subsection{〜ようになる/〜なくなる: Dosáhnout stavu ---/ dosáhnout toho že (ne)}

\subsubsection{〜ようになる}
\begin{center}
\begin{tabular}{|c|c|}
\hline
あるける &ように なりました。\\
\hline
V dict.&yoo ni naru\\
\hline
\end{tabular}
\end{center}
Naučil jsem se chodit.


Když je slovníková forma slovesa následována 〜ように なる, znamená to "dosáhnout stavu že ---" nebo "stal jsem se ---". Často je používáno s potenciální formou. 

a) 早く日本語のしんぶんが読めるようになりたいと思っています。Myslím že brzo budu schopný číst Japonské noviny.

Slovesa která indikují změnu stavu jako ふとる "ztloustnout," やせる "zhubnout" a  つかれる "unavit se," nepotřebují  〜ように なる aby znamenali že jsem "ztloustnout" nebo "zhubnout."

さいきんぜんぜんうんどうをしていないので、少し太ってしまいました。Jelikož jsem poslední dobou necvičil, bohužel jsem přibral.


\subsubsection{ 〜なくなる}
\begin{center}
\begin{tabular}{|c|c|}
\hline
歩(ある)けなく&なりました。\\
\hline
V naku&naru\\
\hline
\end{tabular}
\end{center}
Nemůžu chodit.


Když se nai-forma slovesa změní na 〜なく a je následována なる "stát se", znamená to "nemoci nadále ---." 〜なくなる  je často používané s potenciální formou slovesa. 

a) 父はこのごろお酒をあまり飲まなくなりましたが、また飲むようになるかもしれません。Můj otec už moc nepije, ale zase by mohl začít.



b) しばらく中国語を使わなかったので、話せなく なってしまいました。Jelikož jsem Čínštinu dlouho nepoužil, už nejsem schopen s ní mluvit.


\subsection{ Ostatní}
1にちに3かい znamená  “třikrát za jeden den” nebo “třikrát denně.” に je partikule znamenající  “v."

a) 1日に3回(かい)この薬(くすり)をのんでください。Prosím berte tyhle líky třikrát denně.


b) 2週間(しゅうかん)に1回家族(かい かぞく)に手紙(てがみ)を書きます。Rodině píšu dopis jednou za dva týdny.






 〜ずつ znamená "každý" a je používané pouze po množstevním slově. Indikuje opakování množství nebo rovnoměrnou distribuci.

a)1日に5つずつ漢字をおぼえていますが、前におぼえたのをすぐわすれてしまいます。 Učím se 5 kanji denně, ale snadno zapomenu kanji které jsem se učil před tím.

b) 一人ずつ名前と専攻(せんこうを言ってください。Chtěl bych aby každý z vás postupně řekli svoje jméno a obor.

