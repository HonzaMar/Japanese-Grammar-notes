\section{19. lekce}
\label{sec:lekce_19}

\subsection{Sloveso + ほうがいい: Bylo by lepší udělat--/neudělat}
\begin{center}
\begin{tabular}{|c|c|}
\hline
はやく帰った& ほうがいいです。\\
\hline
Vta&hoo ga ii\\
\hline
\end{tabular}
\end{center}
Bylo by lepší jít brzo domů.


\begin{center}
\begin{tabular}{|c|c|}
\hline
言わない&ほうがいいです。\\
\hline
Vnai&hoo ga ii\\
\hline
\end{tabular}
\end{center}
Bylo by lepší nemluvit.


Když  --ほうがいい"~ je lepší" následuje ta-formu nebo nai-formu slovesa, znamená to "někdo by měl/neměl".
a) 先生に聞いたほうがいいと思います。Myslím že by ses měl zeptat učitele.

b) くすりはのまないほうがいいでしょうか。Neměl bych si vzít léky?



\subsection{〜なくてもいい:Nemuset dělat  ---}
\begin{center}
\begin{tabular}{|c|c|}
\hline
辞書は持ってこなくても&いいです。\\
\hline
V/A/N nakute mo&ii\\
\hline
\end{tabular}
\end{center}
Nemusíte si nosit slovníky.

Když se nai-forma slovesa změní na 〜なくて a je následována もいい znamená to "nemuset dělat --". Když se planá neminulá negativní forma i-adjektiva a  [na- adj./noun+だ] změní na 〜なくて a je následována も  a いい, znamená to "nemusí to být--." ではない přidáno po na-adjektivech a podstatných jménech se změní na でなくてもいい.

a)おかずは ぜんぶ 食べなくてもいいです。Nemusíš sníst všechno.


b) 水はそんなにつめたくなくてもいいです。Voda nemusí být až tak studená.

c) 飲み物はコーヒーでなくてもいいと思います。Myslím že nápoj nemusí být káva.


\subsection{~から: protože /jelikož}
\begin{center}
\begin{tabular}{|c|c|c|}
\hline
おれいの手紙です& から、 &ていねいに書きました。\\
\hline
Sub-klauzule&kara&Hlavní klauzule\\
\hline
\end{tabular}
\end{center}
Jelikož je to děkovný dopis, dal jsem si záležet (při psaní).

1. Kaluzle končící na から vyjadřuje důvod nebo příčinu toho co je řečeno v hlavní klauzuli.
~です/~ます stejně jako planá forma je použita před から. だ by mělo být přidáno za na-adjektiva nebo podstatná jména když planá forma předchází から.
a) いそがしそうでしたから、ホワイトさんには 仕事をたのみませんでした。Jelikož pan White vypadal zaneprázdněný, nepožádal jsem ho aby to udělal.

b) A: 会議に出たほうがいいですか。Měl bych přijít na schůzku?

B: ええ、みんなも来ますから、出たほうが いいと思います。Ano, myslím že bys měl, jelikož jdou všichni ostatní.


2. Obojí ~から a 〜ので/〜んで ( Lekce 17) jsou používané pro uvádění důvodů. V podstatě から je používáno pro indikaci příčiny pro podporu mluvčího osobního názoru/úsudku a je je často používáno pokud hlavní klauzule vyjadřuje mluvčího názor nebo rada. ~ので/~んで  je často používáno pro vysvětlování věcí které nejsou čistě položeny na mluvčího názoru nebo úsudku.


Porovnejte:
a) A: 上着を持って(い)ったほうがいいですか。Mám si s sebou vzít bundu?

B:ええ、さむそうですから、持って(い)ったほうがいいと思います。Ano, vypadá že je zima takže si myslím že bys ji měl vzít.

b) A: おそかったですね。Jdeš pozdě.

B: ええ、バスが来なかったんで、 おそくなったんです。ANo, jdu pozdě protože nepřijel autobus.

\subsection{〜かた: Jak udělat --}
\begin{center}
\begin{tabular}{|c|c|}
\hline
書き&方\\
\hline
Vstem&kata\\
\hline
\end{tabular}
\end{center}
jak psát/ způsob psaní

Když 〜かた “cesta" následuje stem formu slovesa, znamená "jak udělat" nebo "postup činnosti." Když je sloveso co bere objekt použito s  〜かた, を což označuje objekt se změní na の.
手紙を書く psát dopis ––> 手紙の書き方 jak psát dopis

ビデオを使う pustit video ––> ビデオの使い方 jak pustit video


a) 高校の時、ワープロの使い方を習いました。Jak používat word procesor jsem se naučil na střední.

b) 教師:この漢字の読み方を知って(い)ますか。Víš jak se čte tohle kanji.

 学生:ええと...。 わすれました。Hmm.... Zapomněl jsem.

\subsection{ Sloveso + だけだ: Stačí udělat/ pouze udělám ~}
\begin{center}
\begin{tabular}{|c|c|}
\hline
このボタンをおす &だけです。\\
\hline
V dict. &dake da\\
\hline
\end{tabular}
\end{center}
Stačí zmáčknout tohle tlačítko.


Když je slovníková forma slovesa následována だけだ, znamená to "Stačí udělat --" nebo "pouze udělat--."

a) 私は今晩どこへも行きません。 うちに帰って寝るだけです。Dneska nikam nepůjdu. Dnes pouze půjdu domů a spát.

b) A: ここにサインしてもらえませんか。Podepíšeš se sem?
 B:サインするだけでいいですか。Stačí abych se podepsal?
 
 
\subsection{Adjectiv / Podstatné jméno + する:udělat tak aby}

1. Když je ku-forma i-adjektiva následováno slovesem する "dělat", znamená to "udělat něco (tak aby)." Když za ku-formou i-adjektiva následuje sloveso "dělat", znamená to "udělat něco tak aby ..."  する  se může také objevit s na-adjektivem nebo podstatným jménem následovaným  に.
\begin{center}
\begin{tabular}{|c|c|c|}
\hline
I-adj.&〜を大きく&\\

Na-adj.&〜をきれいに&\multirow{3}{*}{する}\\

Noun&〜を 休講に&\\
\hline
\end{tabular}
\end{center}


a)
 A: 音を大きくしてもらえませんか。Dal bys to prosím nahlas?

B: はい わかりました。Jasně.

b) 私はつくえの上をきれいにしました。Uklidil jsem si stůl.


c) 先生は 授業(じゅぎょう)を休講(きゅうこう)にしました。Učitel zrušil hodinu. (Udělal ji zrušenou)



2.~を~く/にする popisuje něčí akci, zatímco  ~く/になる popisuje výsledek změny způsobené akcí.
Porovnejte :
a) 私は部屋をくらくしました。Zatemnil jsem pokoj.

b) 部屋がくらくなりました。Pokoj ztmavnul

\subsection{ Partikule が : Objekt}
Slovesa vyjadřující schopnost jako  わかる "rozumět", touhu いる "potřebovat", slovesa smyslového vnímání jako みえる "být vidět" a きこえる "být slyšet" berou partikuli  が na označení co je rozuměno, potřeba atd. Partikule を nelze použít u těchto sloves.


a) 私はビデオの使い方がわかりません。Nevím jak použít video.

b) A: 申し込みの時、何かいりますか。Potřebuji něco když se přihlásím?

B:学生証 (がくせいしょう) がいると思います。Myslím že potřebujete ISIC.

c) A: 何か聞こえましたか。Slyšel si něco?

B:人のこえや車の音が聞こえました。 Slyšel jsem hlasy a zvuky auta.



が což označuje co je rozuměno/ známo atd. může být nahrazeno  は nebo も jako v následujících příkladech

A:西山さんの住所(じゅうしょ)はわかりますか。Znáš adresu paní Nishiyami?

B: いえ、 わかりません。Ne, neznám.

A: 電話番号(でんわ ばんごう)はわかりますか。Znáš její telefonní číslo?

B: 電話番号もわかりません。Také neznám její telefonní číslo.


\subsection{Ostatní}
かく “psát" může brát partikuli と což označuje citaci. (っ)て, což je ekvivalentní s  と, může být použito místo  と v mluveném jazyce.
A:はがきに何て書いたんですか。Co jsi napsal na ten pohled?

B:「お世話になりました」 って書きました。Napsal jsem: "Jsem ti vděčný za tvojí laskavou pomoc."
I

かく také bere を na označení objektu. Následující ukazuje rozdíl mezi  かく a ~をかく.
Porovnejte:
a) 手紙に「ありがとうございました」と 書きました。Napsal jsem "děkuji" v dopise.
b) 手紙にお礼のことばを書きました。Vyjádřil jsem díky v dopise.
























