\section{8. lekce}
\label{sec:lekce_8}

\subsection{Demonstrativa: ここ そこ あそこ どこ}

ここ そこ a あそこ jsou demonstrativa indikující polohu.  どこ "kde" je tázací slovo z této skupiny. 

A: ここは北口ですか。Je tohle severní východ?

B: いえ、南口ですよ。Ne, tohle je jížní východ.

\subsection{Spojování nominálních vět}
\subsubsection{Spojování kladných vět}
\begin{center}
\begin{tabular}{|c|c|c|c|}
\hline
これは中央線& で、 &かいそく& です。\\
\hline
Noun& de& Noun& desu\\
\hline
\end{tabular}
\end{center}
Toto je Chuo linka a rychlík.

Když se spojují dvě nominální věty, です v první větě se změní na で, čímž se ze dvou vět stane jedna. (Tahle spojovací forma , 〜で, se bude nazývat v této učebnici te-forma (Noun+です) ). Partikule と "a" která spojuje dvě podstatná jména nemůže v tomto případě být použita.

Porovnej:
a) 森さんは山本さんの友だちです。法学部の学生です。Pan Mori je přítel paní Jamamoto. Je studentem práv.

b) 森さんは山本さんのともだちで、法学部の学生です。Pan Mori je přítel paní Jamamoto a student práv.

\subsubsection{Spojování negativních vět}
\begin{center}
\begin{tabular}{|c|c|c|c|}
\hline
ここは 2番線 &では/じゃ なくて& 3番線 &です。\\
\hline
Noun&dewa/ja nakute & Noun & desu\\
\hline
\end{tabular}
\end{center}
Tohle není kolej 2 ale kolej 3.
Když se spojují negativní věta s jinou, では/じゃありません v první větě se změní na では/じゃなくて.

Porovnej: 
a) このバスは渋谷ゆきではありません。新宿行きです。Toto není autobus do Shibuji. Je to autobus do Shinjuku.
 
b) このバスは渋谷行きではなくて、新宿行きです。Tohle není autobus do Shibuji ale do Shinjuku.


\subsection{Partikule で: způsob}

Partikule で je používaná s významem "způsob" 

a) 私はきのう車で箱根へ行きました。Včera jsem jel do Hakone autem.