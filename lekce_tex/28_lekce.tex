\section{28. lekce}
\label{sec:lekce_28}
\subsection{〜と/〜って いう: Citace}

Jsou dva způsoby jak citovat co ostatní řekli, nepřímá citace a přímá citace . と je používané pro označení citace って také ale jen v mluvě.

\subsubsection{Nepřímá citace}

\begin{center}
\begin{tabular}{|| c | c || c | c || c ||}
南さん&は&行く&と&言いました。 \\
\hline
Person & wa/ga & Plain form\footnote{tabulka v lekci 16} & to & iu \\
\hline
\end{tabular}
\end{center}

Pan Minami řekl že půjde.


a) ブラウンさんは部長に相談したほうがいいと言いました。Pan Brown řekl že by bylo lepší mít nejdříve konzultaci s vedoucím oddělení.


\subsubsection{Přímá citace}

Co někdo říká nebo řekl může být přímo citováno jako v  「、、、、」と/っていう. Přímo citované věty jsou často dávány do  「」, což funguje jako uvozovky v Japonštině.

早川さんは「わかりませんね」と言いました。 Pan Hayakawa řekl, "Nevím." 


\subsubsection{Formy いう}

Když se reportuje obsah toho co někdo řekl,  いっています a いっていました jsou normálně používané. 言いました je používané při reportu faktu že někdo něco řekl. いっています nebo いっていました nejsou používané při popisu toho co mluvčí sám řekl. V takovém případě by se mělo použít 言いました.

Porovnejte:

a) 木下さん は用事(ようじ)があるといっていました。 Paní Kinoshita řekla že je zaneprázdněna.


b)  私は用事があると言いました。 Řekl jsem že jsem zaneprázdněn.


\subsubsection{おっしゃる}
おっしゃる je respektující ekvivalent  いう, a je používaný na ukázku respektu k osobě která je citována. おっしゃる je 5だん slovesa, ale masu-forma je おっしゃいます.

A:お父さんはなんておっしゃいました? Co řekl tvůj otec?

B:わからないっていっていました。 Řekl že neví.


\subsection{〜そうだ: Slyšel jsem že--}


\begin{center}
\begin{tabular}{|| c | c ||}
明日の朝、台風が来る & そうです。\\
\hline
Plain form\footnote{tabulka v lekci 16} & soo da \\
\hline
\end{tabular}
\end{center}
Slyšel jsem že zítra ráno dorazí tajfun.


\subsubsection{〜そうだ}

〜そうだ "slyšel jsem že--," "pochopil jsem že ---" nebo "řekli že---" indikuje že mluvčí reportuje nebo předává informaci kterou obdržel tím že něco četl nebo slyšel 〜そうだ následuje za planou formou s だ přidaném po na-adjektivech a podstatných jménech.

a) 川島先生の授業(じゅぎょう)は休講(きゅうこう)だそうです。Slyšel jsem že hodina profesora Kawashimi byla zrušena.

〜そうだ by se nemělo plést 〜そうな "vypadat"\footnote{lekce 10}

\subsubsection{Zdroj informací}

Zdroj informací, který může být vyjádřen jako 〜の話では "podle toho co-- (někdo) řekl" nebo 〜によると "podle ---," se často objevuje s  〜そうだ/〜そうです。


a) 今日の新聞によると、メキシコで大きな地(じしん)があったそうです。Podle dnešních novin, bylo velké zemětřesení v  Mexiku.



\subsection{〜らしい: Vypadá to}

\begin{center}
\begin{tabular}{|| c | c ||}
黒田さんは仕事を止める(やめる) & らしいです。\\
\hline
Plain form\footnote{tabulka v lekci 27} & rashii \\
\hline
\end{tabular}
\end{center}
Vypadá to že paní Kuroda dá výpověď.


〜らしい znamená "vypadá to --" nebo  "Slyšel jsem-- " a je používáno pro vyjádření úsudku který je založen na tom co mluvčí slyšel nebo četl. Toto je, narozdíl od předchozího případu, používáno pokud mluvčí něco usoudí na základě nějaké informace.


a) 世界(せかい)で一番高い山はチョモランマはないらしいです。 Vypadá to že Mt. Chomolungma (Mt. Everest) není ta nejvyšší hora na světě.



Jelikož 〜らしい s sebou nese pocit nejistoty, tak fráze jako よくわかりませんが "Nevím to přesně, ale ..." nebo よく知らないんですけど "Nejsem si jistý, ale ..."  jsou často používané.


\subsection{〜まえ(に): Před tím než---} 

\begin{center}
\begin{tabular}{|| c | c || c ||}
日本へ来る&まえ(に) & 結婚(けっこん)しました。\\
\hline
V dict. & mae (ni) & Main clause \\
Sub & clause&\\
\end{tabular}
\end{center}
Oženil jsem se před tím než jsem přišel do Japonska.

〜まえ(に)následující za slovníkovou formou slovesa znamená "před tím než ..." Ta-forma nemůže předcházet  〜まえ(に)i kdyby se mluvilo o akci která se stala v minulosti.
