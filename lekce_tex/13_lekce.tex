\section{13. lekce}
\label{sec:lekce_13}

\subsection{Te-forma sloves}
\subsubsection{1 だん Slovesa}

Odstraníme 〜ます a přidáme 〜て 

みます vidět --> みて			ねます spát --> ねて
おきます vstát --> おきて    たべます jíst --> たべて

\subsubsection{5 だん Slovesa}

1) Odstraníme 〜ます a nahradíme 〜い、〜ち、〜り za 〜って
かいます koupit --> かって まちます čekat --> まって つくります vytvořit --> つくって
いいます říct --> いって もちます držet --> もって とります vzít --> とって

2) Odstraníme 〜ます a nahradíme 〜み、〜に、〜び  za 〜んで

よみます číst --> よんで しにます zemřít --> しんで あそびます hrát si --> あそんで
のみます. pít --> のんで

3) Odstraníme 〜ます a nahradíme 〜き za 〜いて
かきます psát -->  かいて
ききます poslouchat --> きいて

いきます "jít" je jediná výjimka , te-forma je いって

4) Odstraníme 〜ます a nahradíme 〜ぎ za 〜いで
およぎます plavat --> およいで
いそぎます pospíchat --> いそいで

5 ) Odstraníme 〜ます a nahradíme 〜し za 〜して
はなします mluvit --> はなして
かします půjčit --> かして

\subsubsection{Nepravidelná slovesa}
Jsou jen dvě nepravidelná slovesa します "dělat" a きます "přijít", mají nepravidelné skloňování.

します dělat --> して
きます přijít --> きて


\subsection{〜てください: Zdvořilý rozkaz}
\begin{center}
\begin{tabular}{||c|c||}
\hline
名前をかいて& ください\\
\hline
V te & kudasai\\
\hline
\end{tabular}
\end{center}
Prosím napište své jméno.

Když je te-forma následována ください, znamená to "prosím udělejte --" nebo "Udělejte--- prosím". Je potřeba si uvědomit že 〜てください je vhodné jen v omezených situacích, například v dávání instrukcí, nikoli však ve zdvořilém požadavku.

a) テープを聞いてください。Poslouchejte prosím nahrávku.

\subsection{〜てもらえませんか:  Zdvořilý požadavek}
\begin{center}
\begin{tabular}{||c|c||}
\hline
ペンをかして& もらえませんか。\\
\hline
V te & moraemasen ka\\
\hline
\end{tabular}
\end{center}
Půjčil byste mi prosím pero?

Když もらえませんか následuje te-formu slovesa, indikuje to zdvořilý požadavek, znamenající "--- byste prosím---"

a) キムさんの住所をおしえてもらえませんか。Řekl byste mi prosím adresu pana Kima.

\subsection{Spojování slovesných vět}
\begin{center}
\begin{tabular}{||c|c||}
\hline
名前を書いて、&出しました。\\
\hline
Vte& sloveso\\
\hline
\end{tabular}
\end{center}
Napsal jsem svoje jméno a odevzdal to.

Při spojování dvou a více slovesných vět jednu po druhé, všechny kromě posledního slovesa se změní na te-formu. Partikule と v tomto případě nemůže být použita-

a) 私はときどき英字新聞をかって、うちで読みます。Občas si koupím anglické noviny a doma si je přečtu.

Te-forma sloves neindikuje čas sama o sobě, čas 〜て je určen časem posledního slovesa.

Porovnejte:

a) あした渋谷にいって、友だちに会います。Zítra jedu do Shibuji a setkám se s kamarádem.

b) きのう友だちとうっしょにしょくじをして、えいがをみました。Včera jsme s kamarádem měli večeři a viděli film.

\subsection{Partikule}
\subsubsection{に: Nepřímý objekt}
\begin{center}
\begin{tabular}{ccccccc}
\hline
私 &は& ともだち& に& ペン& を& かしました。\\
\hline
Osoba&wa&Nepřímý objekt&ni&Přímý objekt&o&Sloveso\\
\hline
\end{tabular}
\end{center}
Půjčil jsem kamarádovi pero.

Slovesa jako かします "půjčit"、かえします "vrátit"、だします "vrátit", おしえます "učit", いいます "říct", ききます "zeptat se",  たのみます "požádat" berou nepřímý objekt stejně jako přímý objekt. Nepřímý objekt indikuje příjemce akce. Partikule に označuje nepřímý objekt, přímý objekt je označen partikulí を.

a) きょう図書室にテープをかえします。Dnes vrátím pásku do knihovny. (knihovně)

\subsubsection{までに: Do (nějaký čas)}

までに koresponduje s "do (nějakého indikovaného času)" nebo "ne později než (nějaký čas)". Ukazuje časový limit (termín) pro ukončení nějaké akce nebo události. までに se objevuje se slovesy vyjadřující akci jako  かえします "vrátit",  だします "odevzdat", します "dělat" atd. nebo se slovesy se směrovým pohybem jako いきます "jít",  きます "přijít" かえります "vrátit se". 

までに se nesmí plést s  まで 

Porovnejte:

a) いつも9時までにねます。Vždy do 9 hodin jdu spát. (Jdu nějaký čas před 9 nebo přímo v 9)

b) いつも9時までねます。Vždy spím do devíti. (spím před tím a nepřerušovaně až do 9)




























