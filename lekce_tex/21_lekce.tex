\section{21. lekce}

\subsection{Potenciální forma sloves}
\subsubsection{} Většina sloves má formu zvanou potenciální はなせる		"být schopen mluvit/ umět mluvit" je potenciální forma slovesa 	はなす	"mluvit". Slovesa jako ある	"existovat/být", 	いる	"potřebovat", 	わかる	"rozumět" a 	しる "vědět". Nemají potenciální formu. Potenciální forma se dělá podle následujících pravidel 

\begin{enumerate}
\item  いちだん	slovesa
Odděl 〜る	 ze slovníkové formy a přidej 	〜られる	

\begin{tabular}{cc}
みる →  見られる moci vidět& ねる → 寝られる moci spát\\
おきる → 起きられる moct vstát& 食べる → 食べられる moci jíst\\
\end{tabular}

\item 	Slovesa
Vyměn konečné 	うーだん	hiraganu ze slovníkové formy za 	えーだん hiraganu ve stejném sloupci tabulky níže a pak přidej  〜る

\begin{tabular}{|c|c|c|c|c|c|c|c|c|c||c}
\hline
あーだん&か&が&さ&た&な&ば&ま&ら&わ& nai forma\\
\hline
いーだん&き&ぎ&し&さ&に&び&み&り&い& masu-forma\\
\hline
うーだん&く&ぐ&す&つ&ぬ&ぶ&む&る&う& slovníková forma\\
\hline
えーだん&け&げ&せ&て&ね&べ&め&れ&え& potenciální forma\\
\hline
おーだん&こ&ご&そ&と&の&ぼ&も&ろ&お& později\\
\hline
\end{tabular}

\begin{tabular}{ccc}
かく → かける moci psát&まつ → まてる moci počkat&よむ → よめる moct číst\\
およぐ → およげる umět plavat&しぬ → しねる moci umřít&つくる → つくれる moci udělat\\
はなす → はなせる umět mluvit&あそぶ → あそべる moci hrát&かう → かえる moci koupit\\
\end{tabular}

\item Nepravidelná slovesa

Pro 	する	neexistuje odvozený tvar, ale 	できる	je používané na vyjádření "moct udělat". Potenciální forma 	くる	"přijít" je 	こられる	"moct přijít".

\end{enumerate}

\subsubsection{} Jakmile je sloveso v potenciální formě, chová se jako 1だん	. Tedy všechny potenciální formy sloves následují pravidla skloňování 	1だん sloves.
\subsubsection{} V mluvené forma jsou různé variace v potenciálních formách podle regionu, věku nebo generaci. 

\subsection{Věty s potenciální formou sloves}
\subsubsection{Použítí partikule が}

Když slovesa, která používají objekt jako はなす "mluvit" atd. jsou změněna do potenciální formy, to co je někdo schopný udělat je označeno partikulí が. Patrikule を se nepoužívá.

Porovnej:

a) 母は中国語と英語を話します。Moje maminka mluví čínsky a anglicky.

b) 母は中国語と英語が話せます。Moje maminka umí čínsky a anglicky.

\subsubsection{Schopnost/Možnost/Přístupnost}
Potenciální forma sloves může indikovat něčí schopnost něco udšlat.

a) 私は子供のとき、ぜんぜん泳げませんでした。Když jsem byl dítě, vůbec jsem neuměl plavat.

Potenciální forma sloves také může indikovat přístupnost nebo možnost, znamenajíce "(něco) je přístupné" nebo "je možné udělat..." V tomto případě, jelikož potenciální forma neindikuje schopnost určité osoby, osoba v subjektu se většinou neobjevje.

a) ワープロは事務所で借りられます。Můžeš si půjčit word procesor v kanceláři.


\subsection{〜ことが できる}
\begin{tabular}{||c|c||}
\hline
タイ語を話す& ことができます。\\
\hline
V dict & koto ga dekiru\\
\hline
\end{tabular}
Umím thajsky.

Když je slovníková forma nsáledovaní ことが できる, znamená to "umět --" nebo "být schopen udělat --". Jako potenciální forma sloves, může to být použito na vyjádření něčí schopnosti, možnosti  nebo přístupnosti. Narozdíl od potenciální formy, když sloveso skombinováno s 〜ことが  できる bere objekt, je ten objekt označem を.

a) 私は一人で着物を着ることができません。Neumím si sám nandat kimono.

Obojí 〜ことが できる a potenciální forma vyjadřuje "umět--" nebo "být schopen udělat-- ", ale použití 〜ことが できる je lehce více formálnější.


\subsection{〜て: Příčina}

Sub-klauzule končící v te-formě slovesa, i-adjektivu nebo (na-adj+だ) může indikovat důvod nebo příčinu toho co vyjdřuje hlavní klauzule. Jelikož hlavní použití te-formy je pro spojení dvou klauzulí, nevyjařuje to nutně jasné důvody/příčiny stejně jako  〜から nebo 〜ので.

a) この問題はふざつで、よくわかりません。Tento problém je komplikovaný, tedy moc dobře mu nerozumím.


\subsection{〜すぎる: Přehnat}

〜すぎる znamená "(dělat --) až moc" nebo "být moc ---". Následuje to stem formu slovesa, stem formu i-adjektiva nebo na-adjektivum- 〜すぎる se chová a skloňuje jako 1だん sloveso.
\begin{tabular}{cc}
飲みすぎる	&		働くすぎる\\
強すぎる 	&		難しすぎる\\
かんたんすぎる &		ふくざつすぎる\\
\end{tabular}

a) パーテイーで少しワインを飲みすぎました。Včera jsem na párty vypil až moc vína.

\subsection{Partikule}

\subsubsection{に Zdroj}
Slovesa která vyjadřují nebo implikují "obdržení akce od někoho" jako ならう "brát lekce" a  かりる "půjčit" berou partikuli  に na vyznačení osoby která je zdrojem "akce"

a) 私は友達に韓国語を習っています。Učím se korejštinu od kamaráda.

\subsubsection{でも I (když) /dokonce}
でも přidává význam "i /dokonce" kde nahrazuje  が nebo を, nebo když je přidáno za partikule に、と、へ、から atd.

a) この言葉は子供でも知っています。I děti znají tohle slovo.

でも také může být přidáno za kvantitativní slovo nebo čas jako  すこし "trochu" nebo  にちようび "neděle" atd

a) 毎日すこしでもスポーツをした方がいいと思います。Myslím si že je lepší sportovat každý den, i když jenom trochu.

\subsubsection{しか pouze}

しか(ない)které se objevuje v negativních větách znamená "pouze pro", "pouze" しか nahrazuje が nebo を, nebo je přídáno po partikulích で、と、に、から、まで atd.  しか může být také přidáno za kvantitativní slovo nebo časové slovo.

a) 授業には貴田さんしか来ませんでした。Nikdo kromě (pouze) paní Kida přišla na hodinu.






































