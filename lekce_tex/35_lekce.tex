


\section{35. lekce}
\label{sec:lekce_35}

\subsection{〜ことになっている: Bylo rozhodnuto/zařízeno že --}
\begin{center}
\begin{tabular}{||c|c||}
\hline
いっしょにべんきょうする &ことになっています。\\
\hline
Planá forma&koto ni natte iru\\
\hline
\end{tabular}
\end{center}
Bylo zařízeno že budeme studovat dohromady.

〜ことになっている která následuje planou, přítomnou formu slovesa znamená "bylo zařízeno/naplánováno/rozhodnuto že ---" Indikuje to že existuje plán na něco udělat.

a) あしたアルバイトのめんせつを受けることになっています。Byl mi zítra zařízen pracovní pohovor na brigádu.

〜ことになっている také může znamenat "je pravidlem/regulací že--" indikující že to co bylo zařízeno/rozhodnuto se stalo pravidlem. きそくで "podle pravidel" nebo ほうりつ "podle zákona" může být přidáno pokud je potřeba.

a) 私が住んでいるまちではゴミは分けて出すことになています。 Ve městě kde bydlím je pravidlem že, odpad musí být vytříděný před vynesením.




\subsection{〜はずだ: Očekávám/Věřím že --}

\begin{center}
\begin{tabular}{||c|c||}
\hline
3時間ぐらいでおわる &はずです。\\
\hline
Planá forma&hazu da\\
\hline
\end{tabular}
\end{center}
Během asi tří hodin by to mělo skončit.

〜はずだ znamená "předpokládám/věřím že --" nebo "mělo by--", indikující mluvčího silná očekávání nebo víra založená na spolehlivých informacích. Zajímavé že 〜はずだ nemůže být použito na vyjádření toho co by mluvčí měl udělat. 〜はずだ předchází ho Planá forma, ale だ po na-adjektivu nebo podstatném jménu by mělo být změněno na な respektive の

a) 父は10じまでに帰ってくるはずです。Otec by měl v 10 dorazit domů.

〜はずだ  je často použito v kara-klauzulí "protože/jelikož ..." nebo ba-klauzule "pokud ..." indikující důvod nebo fakta na kterých je založena víra mluvčího.

a) 池山さんはタクシーで行ったから、四時のきゅうこうにまにあったはずです。Jelikož jel pan Ikejama taxíkem, měl by stihnout expres ve 4 hodiny.

\subsection{〜てある: Stav vzniklý činností někoho jiného}
\begin{center}
\begin{tabular}{||c|c||}
\hline
たなにおいて &あります。\\
\hline
V te & aru\\
\hline
\end{tabular}
\end{center}
Bylo to dáno na poličku.

\subsubsection{〜てある}
Když je te-forma tranzitivního slovesa (slovesa která mají přímý objekt) je následováno ある, tak to indikuje současný stav, který je výsledkem akcí někoho jiného.

a) アルバムはひきだしにしまってある。 Album je v šuplíku. (někdo ho tam dal)

\subsubsection{〜てある a 〜ている}



Když slovesa která mají tranzitivní a intranzitivní páry jsou zkombinovány s 〜てある a 〜ている jako v しめてある "bylo (to) zavřeno" a しまっている "je (to) zavřeno" obojí vyjadřují momentální stav který vznikl z již dokončené akce. Ale, jelikož intranzitivní slovesa nevyjadřují co kdo dělá. しまっている indikuje že něco je zavřené, ale ne nutně implikuje že někdo to zavřel. Na druhé straně しめてある implikuje že to zavírání bylo uděláno někým z nějakého důvodu.

Porovnejte:

a) 今日は風が強いので、まどはしめてある。Jelikož je dnes větrno, okno bylo zavřeno.

b) 今日は日曜日なので、ぎんこうはしまっている。Jelikož je Neděle, banky jsou zavřené.



\subsection{Zdůraznění části věty}

\begin{center}
\begin{tabular}{||c|c||c|c||}
\hline
ナイターがはじまる& のは& 6じから& です。\\
\hline
Plain forma & nowa&&da\\
\hline
\end{tabular}
\end{center}
Večerní hra začíná v 6 hodin.

Abychom zdůraznili nějakou část věty jinou než nějaký výrok., použijeme vzor 〜のは〜だ. Část kterou chceme zdůraznit dáme před  〜だ a zbytek věty změníme do klauzule končící planou formou a položíme před 〜のは. V této klauzuli by subjekt měl být vždy označen partikulí が

\textbf{ブラウンさんは}土曜日にしゅっぱつしました → 土曜日にしゅっぱつしたのは\textbf{ブラウンさんです}
Paní Brownová odešla v sobotu.        Byla to paní Brownová která odešla v sobotu

ブラウンさんは\textbf{土曜日に}しゅっぱつしました →  ブラウンさんがしゅっぱつしたのは\textbf{土曜日です}
Paní Brownová odešla v sobotu.                  bylo to v sobotu kdy odešla paní Brownová


