\section{5 lekce}
\label{sec:lekce_5}

\subsection{ーます :Zvyk}

Navíc k vyjádření budoucnosti nebo vůle se 〜ます  může také použít na indikaci zvyku. Rozdíl mezi těmito dvěma významy může být rozeznán za pomocí slov jako いつも "(skoro) vždy" nebo あした "zítra" ve větě. Pokud takováto slova nejsou použita, dá se rozdíl poznat z kontextu.

Porovnej:

a)いつもとしょしつでべんきょうします。 Vždy studuji v knihovně.

b)あしたとしょしつでべんきょうします。 Zítra jdu studovat do knihovny.

Následující slova mohou být použita s indikující zvyk. 

\begin{center}
\begin{tabular}{cccc}
いつも&(skoro) vždy&まいにち&každý den\\
よく&často&まいあさ&každé ráno\\
たいてい&většinou&まいばん&každý večer\\
ときどき&občas&まいしゅう&každý týden\\
\end{tabular}
\end{center}

わたしはまいあさうちでしんぶんをよみます。Každé ráno čtu doma noviny.

\subsection{Výrazy indikující čas (2)}

\subsubsection{Dny v týdnu}
Dny v týdnu jsou vyjádřeny jako 〜ようび jako například v どようび  "sobota" "Který den v týdnu" vyjádříme jako なんよいび. 〜ようび bere partikuli に indikující specifický bod v čase (Lekce 3) 

わたしはまいしゅうどようびにおよぎます。Každou sobotu chodím plavat.


\subsubsection{Hodiny}

〜じかん "-hodin" přidané po číslu indikuje délku času jako například v にじかん "dvě hodiny". Pokud vyjadřujeme přibližnou délku použijeme ぐらい "přibližně" jako například v にじかんぐらい "asi dvě hodiny". ぐらいje občas vyslovováno jako くらい . ごろ nesmí být použito v tomto případě.


\subsection{Partikule}
\subsubsection{は a も nahrazující を}
\begin{center}
\begin{tabular}{||c|c||c||}
\hline
バスケットボール&は&します。\\
\hline
Téma/objekt&wa&sloveso\\
\hline
\end{tabular} 
\end{center}
Hraju basketbal. (Pokud jde o basketbal, tak ho hraju.)

\begin{center}
\begin{tabular}{||c|c||c||}
\hline
バレーボール&も&します。\\
\hline
Téma/objekt&mo&sloveso\\
\hline
\end{tabular}
\end{center}
Také hraju volejbal.

Partikule, která značí téma věty (lekce 1,3) může nahradit partikuli, která označuje přímý objekt slovesa čímž objekt změní na téma. Použití  místo často implikuje že mluvčí speciálně vybral dané téma na rozdíl od ostatních možností. 


a)わたしはクラシックをききます。ロックはききません。 Poslouchám klasickou hudbu. (Ale) neposlouchám rock.

を také může být nahrazeno partikulí  も. Tady partikule  も přidává k předešlému podstatnému jménu význam "také".

a) わたしはよくうどんをたべます。ときどきそばもたべます。Často jím udon. Někdy také jím Soba.



\subsubsection{は a も po časových slovech}
\begin{center}
\begin{tabular}{||c|c||c||}
\hline
あした& は& いきます。\\
\hline
Téma/čas & wa & sloveso.\\
\hline
\end{tabular}
\end{center}
Zítra půjdu.

\begin{center}
\begin{tabular}{||c|c||c||}
\hline
あした &も &いきます。\\
\hline
Téma/čas & mo & sloveso\\
\hline
\end{tabular}
\end{center}
Zítra také půjdu.

Časová slova indikující relativní čas jako きょう "dnes" můžou také vzít partikuli  は a も jako v  きょうは "pro dnešek" a  きょうも "také dnes"

a) わたしはきのうおよぎました。きょうもおよぎます。Včera jsem byl plavat. Dnes také půjdu plavat.

\subsubsection{から/まで: Z/do}
\begin{center}
\begin{tabular}{||c|c||c|c||c||}
\hline
はちじ& から& じゅうじ& まで& べんきょうしました。\\
\hline
čas/místo & kara & čas/místo & made & Sloveso\\
\hline
\end{tabular}
\end{center}
Studoval jsem od 8 hodin do 10 hodin.


Partikule から "z" přichází po slovech indikující začátek jak v čase tak i prostoru. Partikule  まで "do" indikující konec v čase nebo prostoru 

a) まいしゅうげつようびからきにょうびまでれんしゅうします。Cvičím od pondělí do pátku každý týden.


\subsubsection{と: "A"}
\begin{center}
\begin{tabular}{||c|c|c||}
\hline
かようび& と& どようび\\
\hline
Podstatné jméno 1 & to & Podstatné jméno 2\\
\hline
\end{tabular}
\end{center}
Úterý a Sobota(y)

と je partikule spojující podstatná jména a N1 と N2  znamená "N1 a N2". と nemůže být použito pro kombinaci slovesa, věty atd.


a) きょうぎんこうとゆうびんきょくへいきました。 Dnes jsem šel do banky a na poštu.

\subsubsection{や: a tak podobně}
\begin{center}
\begin{tabular}{||c|c|c||}
\hline
テニス& や &バレーボール\\
\hline
Noun1 & ya & Noun2\\
\hline
\end{tabular}
\end{center}
Tenis a volejbal (a další)

や je partikule spojující podstatná jména a indikuje že daná podstatná jména jsou příklady z větší množiny podobných věcí. N1 や N2 znamená "N1 a N2 atd"    や spojuje pouze podstatná jména, nikoli slovesa, věty atd.

a) わたしはテニスやバレーボールをします。Dělám sporty jako tenis a volejbal.



\subsection{Konjunkce それから : Také/ Navíc}
\begin{center}
\begin{tabular}{||c|c|c|c||}
\hline
せんたくをします。&& それから、&そうじもします。\\
\hline
Věta 1 &&Sore kara&Věta 2\\
\hline
\end{tabular}
\end{center}
Peru prádlo. (A navíc) Také uklízím pokoj.

それから je konjunkce spojující dvě věty.

\subsubsection{Navíc}

それから může být použitá pro význam "navíc." Partikule も se často objevuje ve Větě2.

a) わたしはまいにちほんをよみます。それから、しんぶんもよみます。Každý den čtu knihy. (Navíc) Také čtu noviny.

\subsubsection{Poté}
それから také znamená "poté" kdy akce ve větě2 nastává po akci ve větě 1

a) わたしはきのうともだちとしんじゅくでえいがみました。それから、デイスコへいきました。S kamarádem jsme včera viděli film v Šinjuku. Poté jsme šli na diskotéku.

\subsection{する slovesa}

Skupina podstatných jmen indikující akci jako べんきょう "studovat" れんしゅう "trénovat" mohou být změněna na slovesa přidáním します "dělat".Tyto podstatná jména budeme nazývat "suru-nouns". Ty následované します jako  べんきょうします "studovat" berou přímý objekt označený partikulí   を  jako v にほんごをべんきょうします。"studovat japonštinu" což může být vyjádřeno i jako   にほんごのべんきょうをします jelikož  します které je samostatné sloveso znamenající "dělat" může použít "suru-noun" jako objekt.

〜をNします 	 

〜のNをします

Porovnej: 

a) わたしはまいにちにほんごをべんきょうします。 Studuji japonštinu každý den.

b) わたしはまいにちにほんごのべんきょうをします。Studuji japonštinu každý den.


\subsection{どんな Jaký (druh)}

どんな znamená "Jaký (druh)" 

a)  ホワイトさんはどんなスポーツをしますか。	Paní White, jaké sporty děláte? (v angličtině: What kind of sports?)

White: サッカーやテニスをします。 Například hraju fotbal a tenis. 


\subsection{Slovosled}

Jelikož partikule jako を、へ、で、に、と označují vztah mezi podstatnými jmény a slovesa ve větě, je pořadí větných členů jako (Podst. + partikule) v japonštině relativně flexibilní. 


Porovnejte: 

a)ともだちとビデオをみました。 	Koukal jsem na video s kamarádem.

b)ビデオをともだちとみました。	Koukal jsem na video s kamarádem.


Ale jsou tu určitá pravidla: 
\begin{enumerate}
\item Predikát vždy jde na konec věty.
\item Většinou je "čas" před "místem"
\item Delší časová jednotka předchází kratší. Tedy "V devět zítra" bude あしたくじに
\end{enumerate}


a) わたしはあしたくじにだいがくへいきます。Zítra v 9 jdu do školy.































