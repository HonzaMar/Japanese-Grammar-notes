\section{6 lekce}
\label{sec:lekce_6}
\subsection{  Účel směrovaného pohybu}
\subsubsection{ Vstem + にいきます/きます/かえります}
\begin{center}
\begin{tabular}{||c|c||c|c||c||}
\hline
しんじゅく&& えいがを見&に&行きます\\
\hline
Place&e&Vstem&ni&Vdm\\ 
\hline
\end{tabular}
\end{center}
Jedu do Shinjuku abych se podíval na film.

Když se sloveso směrovaného pohybu jako  いきます “jít,” きます "přijít” or かえります
"vratit (domů)," zkombinuje s stem formou slovesa což je následování partikulí に  vyjadřuje to účel odchodu, příchodu atd. Stem forma slovesa je část před ます.(→ Lesson 3, GNI). Partikule  に tu označuje účel. Tedy [Vstem + に いきます/きます/かえります] znamená "jít/přijít/vrátit se (domů) (za účelem) udělat něco."

\begin{center}
\begin{tabular}{ccc}
えいがを見 &→& えいがを見に行きます。\\
vidět film&& jít se podívat na film\\
レポートを出します&→&レポートを出しに来ます\\
 odevzdat report &&přijít (za účelem) odevzdat report\\
ひるごはんをたべます &→& ひるごはんをたべにかえります\\
sníst oběd &&jít domu (za účelem) sníst oběd\\
\end{tabular}
\end{center}

\subsection{ N (+ suru) + に いきます / きます / かえります}
\begin{center}
\begin{tabular}{||c|c||c|c||c||}
\hline
しんじゅく &へ& かいもの &に& 行きます。\\
\hline
Place&e&N (+suru)&ni&Vdm\\
\hline
\end{tabular}
\end{center}
Jedu do Shinjuku (za účelem) nakupovat.



Když se suru-noun jako べんきょう “studium,” かいもの “nakupování,"
さんぽ "procházka" or しょくじ “jídlo" je podobně následováno partikulí  に a je použito se slovesem s směrovým pohybem, indikuje to účel odchodu, příchodu atd. Tedy
[N ( + suru) + に いきます/きます/かえります] znamená "jít/přijít/vrátit se (domů) něco udělat."

a) あした としょかんへ べんきょうに行きます。Zítra jdu do knihovny studovat.

b) わたしはよくこうえんへさんぽに来ます。Často chodím do parku se projít.

c) うちへしょくじにかえりました。Šel jsem domů najíst.

\subsubsection{ Použití partikule  へ a slovosled}
Jelikož je hlavní sloveso [Vstem/N(+suru) にいきます / きます/かえります]  sloveso směrového pohybu tak partikule へ se používá pro indikaci destinace. で nelze v tomto formátu použít.

Porovnejte:
a) たいてい うちへひるごはんをたべにかえります。 Většinou chodím domů na oběd.

b) たいていうちでひるごはんをたべます。 Většinou jím oběd doma.



Porovnejte:
a) リムさんはデパートへかいものに行きました。Paní Lim jela do obchodního centra nakupovat.

b) リムさんはデパートでかいものをしました。Paní Lim nakupoval v obchodním centru.

Slovosled některých větných členů může být flexibilní jako níže.

Porovnejte:
a) しんじゅくへ えいがを見に行きます。Jedu do Shinjuku abych viděl film.

b) えいがを見にしんじゅくへ行きます。Jedu do Shinjuku abych viděl film.

\subsection{〜ませんか: Pozvání/vybídnutí}
\begin{center}
\begin{tabular}{||c|c||}
\hline
行き&ませんか。\\
\hline
Vstem&masen ka\\
\hline
\end{tabular}
\end{center}
Chtěl/a by jít (se mnou)? (doslova: Nepůjdeš?)


〜ませんか následující kořen slovesa znamená "Chtěl/a bys 
~?" or “Proč ne~?" Je to používané pro pozvání někoho někam nebo k vybídnutí něčeho.


a) いっしょに しょくじに行きませんか。Nepůjdeme někam na jídlo?

b) バスケットをやりませんか。Nezahrajeme se basketbal?

c) 南口で会いませんか。Nesetkáme se u jižního východu?


\subsection{ ~ましょう: Návrh}

\begin{center}
\begin{tabular}{||c|c||}
\hline
行き&ましょう。\\
\hline
Vstem&mashou\\
\hline
\end{tabular}
\end{center}
Pojďme.


Když je kořen slovesa následována ~ましょう znamená to "Pojďme udělat---". Toto nazýváno volitional forma 〜ます a je používána pro navrhování něčeho někomu.
a) いっしょに行きましょう。 Pojďme společně.

b) しんじゅくえきの西口で会いましょう。Setkejme se u západního východu Shinjuku.


Když ~ましょう je následováno か ukazuje to mluvčího respekt k názoru posluchače. Tedy 〜ましょうか indikuje nabídku znamenající "můžeme--". Tázací slova jako  どこ "kde", なに "co" atd můžou bát použita s 〜ましょうか.

a) テニスをしましょうか。Nepůjdeme hrát tenis?

b) どこへ行きましょうか。Kam půjdeme?

c) 何を食べましょうか。Co budeme jíst?

\subsection{Partikule}
\subsubsection{に: Účel}
Lekce 6

\subsubsection{か: Nebo}
\begin{center}
\begin{tabular}{|c|c|c|}
\hline
2時& か& 2時半\\
\hline
Noun1&ka&Noun 2\\
\hline
\end{tabular}
\end{center}

か je partikule spojující dvě podstatná jména (včetně časových slov jako あした "zítra") a (N1 か N2) znamená "N1 nebo N2".  か nemůže být ale použito pro spojení dvou sloves nebo vět.

a) わたしはまいあさコーヒーかこうちゃを飲みますか。Každé ráno piju kafe nebo čaj.l

\subsubsection{の: Modifikace podstatných jmen (4)}

(N1 の N2) může indikovat že N2 je (místně) v N1

上野のびじゅつかん muzeum v Uenu


\subsubsection{でも: Nebo něco (1)}

partikule でも může být použita místo を po podstatném jménu jako v  おちゃでものみに行きませんか když se někdo zve někam. 〜でも znamená "-nebo něco" a je často používané s 〜ませんか nebo 〜ましょうか. Je to pro nabídnutí možností jiných než je předcházející podstatné jméno.

a) コーヒーでも飲みませんか。Něchtěl/a bys mít se mnou kafe nebo něco?






















