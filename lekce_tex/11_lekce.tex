\section{11. lekce}
\label{sec:lekce_11}

\subsection{Věty vyjadřující "bytí"}

\begin{center}
\begin{tabular}{|c|c|c|c|c|}
\hline
あそこ& に& スミスさん& が& います。\\
\hline
Místo & ni&Podstatné jméno (životné)&ga&imasu\\
\hline
\end{tabular}
\end{center}
Pan Smith je támhle.


\begin{center}
\begin{tabular}{|c|c|c|c|c|}
\hline
ここ& に& しゃしん& が& あります。\\
\hline
Místo & ni & Podstatné jméno (neživotné) & ga&arimasu\\
\hline
\end{tabular}
\end{center}
Obrázek je tady.

〜に 〜が います/あります indikuje osobu/věc "přítomnou" na určitém místě, znamenající "Někde něco je." Partikule が je použita pro zvýraznění subjekt věty a partikule に označuje místo kde někdo/něco je.

a) ベンチのそばにいぬがいます。Blízko lavičky je pes.

Tázací věta s "Kdo/Co?" je vyjádřena pomocí だれ nebo なに. 

A: じむしょにだれがいましたか。Kdo byl v kanceláři?

B: 高田さんがいました。Paní Takada tam byla.


Partikule は označující téma často nahrazuje partikuli  が , identifikující subjekt věty jako téma. が se objevuje pouze pokud je subjekt (podstatné jméno nebo fráze s pdostatným jménem před が) v hlavním zájmu věty. Pokud je použito が jako v  〜に〜がいます/あります indikuje to "kdo/co" je tam, podstatné jméno před  が je tedy v zájmu věty. Pokud ale je použito は jako v  〜は 〜に います/あります, hlavní zájem je v "kde" něco/někdo je.

a) ニュージーランドに兄がいます。\textbf{Můj starší bratr} je na Novém Zélandu.

b) 兄はニュージーランドにいます。Můj starší bratr je \textbf{na Novém Zélandu.}


\subsection{います/あります použité s množstevním slovem}
\begin{center}
\begin{tabular}{|c|c|c|c}
\hline
カナダ人& が& 三人& います。\\
\hline
Noun&ga& množství&imasu\\
\hline
\end{tabular}
\end{center}
Jsou tam tři Kanaďané.


\begin{center}
\begin{tabular}{|c|c|c|c}
\hline
おさら& が& 2まい& あります。\\
\hline
Noun&ga&množství&arimasu\\
\hline
\end{tabular}
\end{center}
Jsou tam dva talíře.

množstevní slova jako 3人 "tři lidi" nebo  2まい "dvě (ploché věci)" se většinou objevují těsně před slovesy います nebo あります.

a) れいぞうこの中にケーキが2つあります。V ledničce jsou dva kousky dortu.

\subsection{Věty vyjadřující vlastnictví}

います a ありますjsou také používané na vyjádření "mít --" は označuje osobu v tématu (vlastníka) a  が  označuje vlastněnou věc. (〜は)〜が 〜います často vyjadřuje složení rodiny.

a) リーさんはお兄さんが2人います。Pan Lee má dva starší bratry.



\subsection{〜なさい: Rozkaz}
\begin{center}
\begin{tabular}{|c|c|}
\hline
漢字で書き& なさい\\
\hline
Vstem&nasai\\
\hline
\end{tabular}
\end{center}
Piš v kanji.

Když se za stem-formu slovesa dá 〜なさい, indikuje to rozkaz. V mluvě se tento výraz používá rodiči vůči dětem. V psané formě se používá jako instrukce v testech atd.

a) 日本語で答えなさい。 Odpověz japonsky. 