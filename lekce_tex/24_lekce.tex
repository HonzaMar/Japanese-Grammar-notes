\section{24 lekce}
\label{sec:lekce_24}

\subsection{Slovesa dávání a přijímání}


\subsubsection{あげる}
\begin{center}
\begin{tabular}{||c|c||c|c||c|c||c||}
\hline
私 &は/が &みなみさん& に &ほん& を& あげました。\\
\hline
Dárce&wa/ga&Příjemce&ni&objekt&o&ageru\\
\hline
\end{tabular}
\end{center}


あげる je používáno když je příjemce někdo jiný než mluvčí nebo člen jeho skupiny. Když je mluvčí dárce, příjemce může být členem jeho skupiny. Partikule に označuje osobu která dostává.

a) 私は木村さんにネクタイをあげたいと思っています。Myslím že bych rád dal kravatu panu Kitamurovi.


\subsubsection{くれる}
\begin{center}
\begin{tabular}{||c|c||c|c||c|c||c||}
\hline
みなみさん& は/が& 私& に& ほん &を &くれました。\\
\hline
Dárce&wa/ga&Příjemce&ni&objekt&o&kureru\\
\hline
\end{tabular}
\end{center}

 
くれる je používané když příjemce je mluvčí nebo někdo z jeho skupiny. Partikule  に je použivaná pro označení příjemce.

a)毎年リーさんは妹にクリスマスプレゼントをくれます。 Pan Lee dává mé sestře dárek na Vánoce každý rok.


Je potřeba si uvědomit že あげる nemůže být použito když někdo jiný než mluvčí dává něco někomu v mluvčího skupině. くれる by mělo být použito.


Na porovnání:

a) 高木さんは妹にCDをくれました。Pan Takagi dal mé mladší sestře CD.

b) 高木さんは山下さんにCDをあげました。Pan Takagi dal paní Yamashita CD.



\subsubsection{もらう}
\begin{center}
\begin{tabular}{||c|c||c|c||c|c||c||}
\hline
前田さん& は/が &みなみさん &に& ほん& を &もらいました。\\
\hline
Příjemce&wa/ga&Dárce&ni&Objekt&o&morau\\
\hline
\end{tabular}
\end{center}


Na rozdíl od sloves dávání あげる a くれる,  もらう "obdržet" popisuje akci "přijímání". Je důležité vědět že mluvčí nemůže označit sám sebe jako dárce když se použije もらう. Osoba která dává je označena に indikující "zdroj" akce. Toto に by se nemělo plést s に které označuje příjemce při použití あげる a くれる.


a) 姉はボーイフレンドにゆびわをもらいました。Moje starší sestra obdržela prsten od jejího přítele.

Partikule から také může být použitá místo に na označeni dárce. Toto je pravidlem pokud někdo něco obdrží od nějaké instituce.

Partikule で je používaná pro indikaci místa, kde někdo něco dostal

\subsubsection{さしあげる, くださる, いただく}

さしあげる くださる a いただく jsou zdvořilé ekvivalenty あげる respektive くれる respektive もらう. 

さしあげる je skromný ekvivalent あげる a je používaný místo あげる když se popisuje dávání něčeho nadřízeným.

A: そのはな、だれにあげるんですか。Komu dáš tyto květiny?

B: 山中先生のおくさまにさしあげるんです。Dám je manželce profesora Jamanaky.

くださる je respektující ekvivalent くれる a je používán pro popis nadřízeného jak něco dává mluvčímu nebo jeho skupině. Sloveso くださる se "skloňuje" nepravidelně. Tedy masu-forma くださる je くださいます.

北村さんのお母さんがメロンをくださいました。Matka paní Kitamury mi dala meloun.

いただく je skromný ekvivalent もらう a je používán když někdo něco obdrží od nadřízeného.

A: その辞書、買ったんですか。Koupil jsi ten slovník?
いえ、広田先生にいただいたんです。 Ne, dostal jsem ho od profesora Hirota.

\subsubsection{やる}

Když popisujeme dávání něčeho zvířeti, nebo vodu rostlině, ne あげる ale やる by mělo být použito, stejně jako se používám pro dávání něčeho mladším sourozencům nebo členům rodiny.


毎日はなにみずをやります。

\subsection{〜のではなくて/〜んじゃなくて}
\begin{center}
\begin{tabular}{||c|c||c||}
\hline
買った& のではなくて/んじゃなくて &もらったのです/んです。\\
\hline
Planá forma & no dewa/ n ja nakute& \\
\hline
\end{tabular}
\end{center}
Není to tak že bych si to koupil, já to dostal.


〜のではなくて/〜んじゃなくて je te-forma negativního 〜のだ/〜んだ (lekce 16) 〜んじゃなくて je více hovorová fráze používaná v mluvě.  〜のではなくて je používaná ve formální mluvě a psaném projevu.

日本へはあそびに来たのではなくて、勉強に来たのです。Nepřišel jsem si do Japonska hrát, přisel jsem studovat.


\subsection{Příslovce まだ (〜ていない: zatím ne (udělal ...)}

まだ použité v negativních větách znamená "zatím ne" (lekce 21) Zatím ještě neudělal---------" je vyjádřeno jako まだ 〜ていない.

a) プレゼントはもう書いましたが、まだわたしていません。Už jsem mu koupil dárek. Ale ještě jsem mu to neprezentoval.




\subsection{N1 は N2 が + přídavné jméno}
\begin{center}
\begin{tabular}{||c|c||c|c||c||}
\hline
この スカーフ& は& いろ& が &きれいです。 \\
\hline
Podstatné jméno 1 & wa & Podstatné jméno 2 & ga& Adjektivum\\
\hline
\end{tabular}
\end{center}

Tahle šála má krásnou barvu
 
 
 
( N1 は N2 が + přídavné jméno) může být použito když se popisuje N2, které je součástí tématu N1. Tento vzor je často používaný pro popisování vlastností lidí nebo věcí.

a) 兄はせが高くて、バスケットが上手です。 Můj starší bratr je vysoký a dobrý v basketbale. 



















