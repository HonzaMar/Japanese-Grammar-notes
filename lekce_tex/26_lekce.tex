\section{26. Lekce}
\label{sec:lekce_26}
\subsection{〜とき: Když --}

\begin{center}
\begin{tabular}{||c|c||c||}
\hline
家に帰った &時(に)& 「ただいま」と言います。\\
\hline
V ta&toki(ni)&Hlavní klauzule (přítomná)\\
\hline
Sub&klauzule&\\
\hline
\end{tabular}
\end{center}
Když se vrátíš domů, řekneš "Tadaima."
použita
Když se akční sloveso používá v toki-klauzuli (Lekce 22). Měli bychom věnovat speciální pozornost tomu zda je použita přítomná nebo minulá forma slovesa. Přítomná forma slovesa (slovníková forma) se používá když akce v hlavní klauzuli nastává simultánně nebo těsně před akcí v toki-klauzuli.

a) 私は本を読む時、めがねをかけます。Když čtu knihu, mám na sobě brýle.

Na druhou stranu, minulá forma (ta-forma) je používaná když se akce v hlavní klauzuli stane okamžitě po akci v toki-klauzuli, zhruba znamenající "když je daná akce ukončena, další akce se stane"

Porovnejte: 

a) 明日友達に\textbf{あった}とき、そのことを話します。Až uvidím svého kamaráda zítra, řeknu mu o tom.

b) 明日友達に\textbf{会う}とき、カメラをもって行きます。 Až se půjdu setkat s mým kamarádem zítra, vezmu si svoji kameru.


\subsection{〜てはいけない/〜ちゃいけない: Zákaz}

\begin{center}
\begin{tabular}{||c|c||}
\hline
本を見て は&いけません。\\
\hline
V te wa&ikenai\\
\hline
\end{tabular}
\end{center}

Na tu knížku se nesmíš podívat.

Když je te-forma slovesa následována は a zkombinována s  いけない "je špatné" doslova "nemoci jít", vyjadřuje to zákaz jako v pravidlech nebo regulacích, zhruba korespondující s "neměl bys"

としょかんの人は、中で食べたり飲んだりしてはいけないと言いました。Knihovník/ice řekl že uvnitř nesmíme jíst ani pít.

ちゃいけない a じゃいけない je zkrácená forma てはいけない respektive ではいけない tyto formy nejsou používané v psané japonštině.

\subsection{〜こと: Nominalizér}

\begin{center}
\begin{tabular}{||c||c|c|c||}
\hline
私のしゅみは&切手をあつめる&こと&です。\\
\hline
&Klauzule (V dict)&koto&\\
\hline
\end{tabular}
\end{center}
Mým koníčkem je sbírání známek.

Když je 〜こと přidáno ke klauzuli končící ve slovníkové formě slovesa, změní (Klauzule +こと) na ekvivalent podstatného jména.  (Klauzule +こと) funguje jako fráze s podstatným jménem ve větě a může být požitá jako téma.


a) 父の仕事はワインを作ることです。 Prací mého táty je výroba vína. 


\subsection{〜というのは/〜(っ)ていうのは:co se týče}

〜というんは znamená "co se týče..." a je používané když se vysvětluje definice slov nebo výrazů. と může být nahrazeno jeho ekvivalentem (っ)て jako v  〜(っ)ていうのは což je používáno pouze v mluveném jazyce

a)どくしょというのは本を読むことです。 Dokushi znamená "čtení knih".


\subsection{〜はじめる/〜おわる:Začít dělat ---/ Skončit ---}

\begin{center}
\begin{tabular}{||c|c||}
\hline
食べ&はじめる\\
\hline
V stem &hajimeru\\
\hline
\end{tabular}
\end{center}
začít jíst

\begin{center}
\begin{tabular}{||c|c||}
\hline
食べ&おわる\\
\hline
V stem&owaru\\
\hline
\end{tabular}
\end{center}
přestat jíst


〜はじめる a 〜おわる  následující za stem formou slovesa znamenají "začít dělat. .." respektive "skončit ..."

a) 私は5分休んでから、また泳ぎ始めました。Začal jsem znova plavat po tom co jsem si dal pauzu na 5 minut.

\subsection{Partikule と: V porovnání s}

同じだ "být stejný" ちがう "lišit se" a にている"být podobný)" používají partikuli と která označuje osobu/věc která se porovnává. Tedy N1は N2と おなじです/ちがいます/にています znamená N1 je stejná/jiná/ podobná od N2.

a) これはぼくがもっているカメラと同じです。Tohle je stejná kamera jakou mám já.

\subsection{Partikulární fráze 〜にたいして: Cíl akce}

〜にたいして znamená "respektujíce" nebo "vůči někomu" a indikuje cíl na který je akce směřována.


a)目上の人に対して 「ありがとう」と言いてはいけません。  Neměl bys používat "arigatou" když mluvíš s nadřízenými lidmi.

\subsection{など/なんか: a tak podobně}

Jak など tak なんか znamenají "atd" nebo "a tak podobně". なんか je ekvivalent など  a používá se pouze v mluvené řeči.など se používá ve formální řeči  a v psaném projevu. (Noun + など/なんか) se chová jako podstatné jméno a mohou u něho být použité všechny partikule. 

a) 高校の時、よくテニスやサッカーなどがしました。 Často jsem hrál tenis, fotbal a tak podobně, když jsem byl na střední. 




















