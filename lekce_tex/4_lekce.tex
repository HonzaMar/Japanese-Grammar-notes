\section{4. lekce}
\label{sec:lekce_4}

\subsection{Slovesa vyžadující objekt}
\begin{center}
\begin{tabular}{||c|c||c||}
\hline
レポート&を&かきます。\\
\hline
Objekt&o&Sloveso\\
\hline
\end{tabular}
\end{center}
Jdu napsat report.

Slovesa jako たべます "jíst",  のみます "pít" a  よみます "číst (něco)" vždy požadují objekt označený を. 

a) わたしはあしたえいがをみます。Zítra se jdu podívat na film.

\subsection{Slovesné věty: Minulost}
\subsubsection{Pozitivní a negativní}

Minulá forma 〜ます je 〜ました indikující že se něco stalo v minulosti. 〜ませんでした je negativní forma 〜ました.
\begin{center}
\begin{tabular}{|c|c|c|}
\hline
&Pozitivní&negativní\\
\hline
ne-minulá&かきます& かきません\\
\hline
minulost&かきました& かきませんでした\\
\hline
\end{tabular}
\end{center}

Časová slova jako きのう "včera" ゆうべ "minulou noc" jsou často používané se slovesy v minulé formě.

a) やまもとさんはきのうかまくらへいきました。Paní Jamamoto včera jela do Kamakury. 

\subsubsection{Udělal --}

Navíc k "udělal ---" 	〜ました může také znamenat "již/už udělal" (odpovídá to present perfect v angličtině). V takovém případě se často používá もう  "již/už".

a)わたしはもうレポートをだしました。 Report jsem už odevzdal.

\subsection{Partikule}
\subsubsection{O を : objekt}
\subsubsection{ で:Místo aktivity}
\begin{center}
\begin{tabular}{||c|c||c||}
\hline
うち&で&かきました。\\
\hline
místo&de&V (akce)\\
\hline
\end{tabular}
\end{center}
Napsal jsem to doma.

Partikule で  indikuje že předcházející podstatné jméno je lokace kde se děje nějaká akce. Je používaná se slovesy vyjadřující akci よみます "číst"  たべます "jíst" atd.

a)うちでほんをよみます。 Doma budu číst knížku.


\subsubsection{ と: Společně s }
\begin{center}
\begin{tabular}{||c|c||c||}
\hline
ともだち&と&べんきょうしました。\\
\hline
Osoba&to&sloveso\\
\hline
\end{tabular}
\end{center}
Studoval jsem s kamarádem.

Partikule と následující osobu indikuje "společně s (někdo)".

a)わたしはともだちとバスケットをしました。 Hrál jsem basketbal s kamarády.


\subsubsection{の :modifikace podstatného jména (3)}

(N1 の N2) (lekce 1,2) může indikovat "N2 je o N1" nebo "N1 je něco jako N2" 

こくさいかんけいのレポート report o mezinárodních vztazích 

かんじのしゅくだい domácí úkol z kanji

えいごのせんせい učitel angličtiny

なんの znamená "o čem" 

a) A: それ、なんのレポートですか。O čem je ten report?

B:けいざいのレポートです。  Je to report o ekonomice.






