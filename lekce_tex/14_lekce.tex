\section{14. lekce}
\label{sec:lekce_14}

\subsection{〜ています (〜ている): Průběhová akce nebo stav}
\begin{center}
\begin{tabular}{|c|c|}
\hline
電話を かけて& います\\
\hline
V te&imasu (iru)\\
\hline
\end{tabular}
\end{center}
Právě telefonuje.

\paragraph{1.} Když te-formu slovesa následuje います "existovat", v podstatě to znamená průběhovou akci nebo stav. Negativ  〜ています je 〜ていません. 

1) Akce nebo událost která právě teď probíhá
Když je te-forma slovesa infikující akci nebo událost jako みます "vidět"  おしえます "učit" atd.  následována  います, vyjadřuje to akci nebo událost která právě teď probíhá.

a) 弟はいまテレビを見ています。Můj mladší bratr teď sleduje televizi.

2) Opakovaná nebo zvyková akce
Te-forma slovesa indikující akci nebo sloveso směrového pohybu následovaného います může vyjádřit opakovanou nebo zvykovou akci.

a) 弟は毎朝ジョギングをしています。Můj mladší bratr chodí každé ráno běhat.

3) Momentální stav 
すみます "žít/bydlet" もちます "držet" うります "prodat" つとめます "pracovat pro" a かよいます "dojíždět" musí být využity v jejich 〜ています tvaru aby vyjádřily význam "žít (teď)" ...

a) 私はいま横浜に住んでいます。Teď žiju v Yokohamě.




\paragraph{2.} Sloveso indikující akci, nebo událost použité v〜ています formě může vyjádřit buď "bytí (v Aj be-ing)" nebo "have been--." Správný význam může být určen z kontextu a použití příslovcí jako いま "teď" さいきん "poslední dobou" atd.

Porovnejte:
a) 中ださんはいまタイムをよんでいます。Paní Nakada teď čte časopis Time.

b) 私はさいきんタイムよんでいます。Poslední dobou čtu časopis Time.

\paragraph{3.}〜ていました, minulá forma  〜ています indikuje že někdy v minulosti byla nějaká kontinuální akce pro určitou dobu. 

a) 父はさっきまでテレビを見ていました。Ještě před chvílí se můj táta koukal na televizi.


\paragraph{4.}〜ていらっしゃいます je zdvořilá forma 〜ています

A: ごりょうしんは近くに住んでいらっしゃいますか。Žijí tvoji rodiče poblíž?
B: はい、 近所に住んでいます。Ano, žijí v sousedství.

\paragraph{5.} しります "vědět" je vždy používáno v  〜ています formě na vyjádřené "vědět". Negativ 知っています je nepravidelné しりません. 知っていません nemůže být použito na indikaci "nevím"

A: スミスさんを知っていますか。Znáš pana Smitha?
B: いえ、しりません。Ne, neznám.


\subsection{Příslovečné použití adjektiv}
\begin{center}
\begin{tabular}{|c|c|}
\hline
早く& おきます\\
\hline
I-adj.-ku&Sloveso\\
\hline
\end{tabular}
\end{center}
Vstávám brzo.

\vspace{1 cm}
\begin{center}
\begin{tabular}{|c|c|c|}
\hline
しずか& に &はなします。\\
\hline
Na-adj.&ni&Sloveso\\
\hline
\end{tabular}
\end{center}
Mluvím potichu.

Ku-forma i-adjektiva  stejně jako na-adjektiva následovaná に může být použita zvlášť jako příslovce.

a) 事務所の人はいそがしくはたらいていました。Zaměstnanci v kancelářích mají hodně práce.


\subsection{Adjektivum/Podstatné jméno + なります (なる): Stát se}

Když se ku-forma i-adjektiva nebo (na-adj. + に) zkombinuje se slovesem  なります "stát se", indikuje to změnu stavu něčeho. なります také může být kombinováno s podstatnými jmény, která jsou následována partikulí に. Když se なります použije v minulé formě jako  〜く/〜に なりました, znamená to "změnil se/stal se."

\begin{center}
\begin{tabular}{|c|c|c|}
\hline
I-adj. &むずかしく &\\
Na-adj. &しずかに & なります\\
Noun &大学生に&\\
\hline
\end{tabular}
\end{center}

a) 日本語のしけんはさいきんむずかしくなりました。Japonské zkoušky se staly těžšími v poslední době.


\subsection{N1 と/(っ)て いう N2: N2 zvané N1}

\begin{center}
\begin{tabular}{|c|c|c|}
\hline
はやしさん &  という/ っていう & おたく\\
\hline
Noun1&to/(t)te iu& Noun2\\
\hline
\end{tabular}
\end{center}
Rezidence Hayashi


N1 と/(っ)て いう N2 znamená "N2 zvané/známé jako N1". Když se ptáme na jméno místa, používáme なん と/て いう  〜ですか "Jak je --- nazýváno?" 〜(っ)て je ekvivalentní  〜と a je používáno v mluvené řeči. 〜て může být použito místo  〜って když je předchozí slabika ん.  〜という je používáno ve formální mluvě a psaném projevu.

a) 私はともだちと松本というところへいきました。S kamarádem jsme šli na místo zvané Matsumoto.

\subsection{Partikule}
\subsubsection{から: Od (indikovaný čas)} 

から znamená "od (čas)" a je používáno jako v případě さっきから "od nedávné chvíle" きょねんから "od minulého roku".  Když (čas+ から) je použito s  〜ています, znamená že se tak děje od indikovaného času. 


さっきから雨がふっています。Před chvíli začalo pršet. (Od té chvíle prší)



\subsubsection{Dvojité partikule}
は může být přidáno po に označujíce statickou pozici.  は také může být přidáno po jiných partikulích, jako  に s označením specifického bodu v čase. で、と、へ  atd. čímž se předcházející podstatné jméno (fráze) změní na téma.


a) 私は夏休みによく旅行しますが、冬休みにはしません。


Pokud je も dáno po  に、で、と nebo  へ, význam "také" je přidán v pozitivní větě (v záporné také, v češtině)


a) 私は冬休みにアルバイトをしました。春休みにもします。Během zimních prázdnin jsem chodil na brigádu. Během Jarních prázdnin budu také chodit na brigádu.


\subsubsection{だけ: Pouze}

だけ přidává význam "pouze" nebo "jedině" když nahrazuje が  nebo を nebo když je položeno před partikuli に、で、と atd. が je možno přidat を po だけ v psaném projevu.

a) 私はお茶だけ飲みました。Pil jsem pouze čaj.

だけ je také využíváno po kvantitativních slovech jako  ひとり "jeden (člověk)" a časových slovech jako きょう  

a) クラスには男の学生が1人だけいます。Ve třídě je pouze jeden mužský student.


















































