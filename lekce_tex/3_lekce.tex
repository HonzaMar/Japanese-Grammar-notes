\section{3. lekce}
\label{sec:lekce_3}

\subsection{Věty se slovesy}

\subsubsection{ Budoucí akce/ něčí vůle}
\begin{center}
\begin{tabular}{||c|c||c||c||}
\hline
わたし&は&あした&でかけます。\\
\hline
Osoba&wa&čas&Vmasu\\
\hline
\end{tabular}
\end{center}
Zítra půjdu ven.

\begin{center}
\begin{tabular}{||c|c||c||c||}
\hline
わたし&は&あした&でかけません\\
\hline
Osoba&wa&čas&Vmasen\\
\hline
\end{tabular}
\end{center}
Zítra ven nepůjdu.

〜ます je ne-minulá (non-past) slovesná forma která vyjadřuje budoucí akci, nebo vůli mluvčího. Tato forma se nazývá masu-forma. 〜ません  je negativní forma 〜ます . Činitel akce je označen partikulí は.

わたしはきょう出かけません あしたでかけます。Dnes nepůjdu ven. Půjdu zítra.


\subsubsection{Tázací věty}

Ano/ne otázka se vytvoří přidáním か za 〜ます .

A:こんばんでかけますか Půjdeš dnes ven?

B:いえ、でかけません Ne, nepůjdu.


\subsection{Slovesa pohybová se směrem}
\begin{center}
\begin{tabular}{||c|c||c||}
\hline
しんじゅく&へ&いきます\\
\hline
Místo&e&Vdm\\
\hline
\end{tabular}
\end{center}
Půjdu do Shinjuku.


いきます "jít"  きます"přijít"   かえります "vrátit se domů" jsou slovesa pohybu se směrem a jsou často používány s partikulí へ která označuje směr nebo destinaci vůči které je pohyb dělám.

a)わたしはきょうだいがくへいきます。 Dnes jdu na univerzitu.



\subsection{Výrazy indikující čas}

\subsubsection{Říkání času}

〜じ znamená "- hodin" a  〜ふん/ーぷん znamená "minuty"  〜ふん se zkracuje nebo se mění výslovnost když použito s čísly 1,3,4,6,8 a 10. じゅっぷん "deset minut". Půl hodiny po nějaké hodině je  〜じはん.

A: いま なんじですか。Kolik je teď hodin?

B: 4じはんです。Je půl páté (půl hodiny po 4).


\subsubsection{Čas akce/události}
\begin{center}
\begin{tabular}{||c|c||c||}
\hline
6じ&に&おきます。\\
\hline
Specifický čas&ni&sloveso\\
\hline
\end{tabular}
\end{center}
Vstanu v 6 hodin.

に je partikule označující specifický bod v čase kdy se akce nebo událost stane. "Ve 3 hodiny" se vyjádří jako さんじに. Když se vyjadřuje přibližný čas,  ごろ "okolo" je použito.  さんじごろ tedy znamená "okolo 3 hodin"


A: Kdy asi přibližně půjdete ven?

Smith: Půjdu ven v 9 hodin.

Poznamenejte si že za slovy indikující relativní čas Dnes, zítra atd. nemůže být partikule

\subsection{Partikule}

\subsubsection{E (へ) :Směr  (Lekce 2)}

\subsubsection{に :Specifický bod v čase (Lekce 3)}


\subsection{Příslovce}


Obojí ちょっと   "na chvilku" a  すぐ  "okamžitě" jsou příslovce a mohou modifikovat slovesa jako níže

a) ちょっとぎんこうへいってきます。Jdu na chvíli do banky. 



















