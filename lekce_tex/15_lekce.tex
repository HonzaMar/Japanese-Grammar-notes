\section{15. lekce}
\label{sec:lekce_15}


\subsection{Klasifikace sloves a jejich časování}

\subsubsection{1だん slovesa  a 5だん slovesa}

Až na dvě nepravidelná slovesa します きます jsou všechny slovesa klasifikována na  1だん slovesa a 5だん slovesa 

1) 1だん slovesa

Horizontální řady v tabulce níže se nazývají だん, pouze 2 jsou používány pro první typ sloves. První například pro sloveso  みます tak jeho formy se změní použítím  み  (い だん) a  nebo u たべます se vždy mění s použitím べ (えーだん)


2)5だん 
Vertikální sloupce v tabulce se nazývají ぎょう a hiragana ve stejném ぎょう (tedy かーぎょう atd.) je používaná pro časování 5だん sloves. いだん je používaná pro masu-formu.  
\begin{center}
\begin{tabular}{|c||c|c|c|c|c|c|c|c|c|}
\hline
だん/ぎょう& か& が& さ& た& な& ば& ま& ら& わ\\
\hline
あ &か& が& さ& た& な& ば& ま& ら& わ\\
\hline
い &き &ぎ &し &ち &に &び &み& り& い\\
\hline
う& く &ぐ& す& つ& ぬ& ぶ& む& る& う\\
\hline
え &け &げ &せ &て &ね &べ &め &れ& え\\
\hline
お &こ& ご& そ& と& の& ぼ& も& ろ& お\\
\hline
\end{tabular}
\end{center}


\subsubsection{Slovníková forma}
1) 1だん Slovesa

Místo ます dáme る 

見ます→ みる  ねます→ ねる
おきます→ おきる たべます →食べる

2) 5だん slovesa 
Místo ます a vyměníme いだん hiraganu před ます s  うだん hiraganou ve stejném ぎょう sloupci

かきます ー> かく あそびます → あそぶ
およぎます → およぐ  よみます → よむ
はなします → はなす  つくります → つくる
まちます → まつ  書います → かう
しにます → しぬ

3) Nepravidelná slovesa

します dělat → する きます přijít ー> くる

\subsubsection{Nai-forma sloves}
Negativní verze slovníkové formy je nazývána nai-forma. Nai-forma slovesa みる je みない.

1) 1だん slovesa 
Místo る ze slovníkové formy dáme ない
みる vidět → みない   ねる spát → ねない
おきる vsát ー> おきない  食べる jíst → たべない

2) 5だん slovesa

Vyměníme poslední うだん hiraganu slovníkové formy za あだん hiraganu ve stejném ぎょう sloupci a přidáme ない

かく psát → 書かない あそぶ hrát si → あそばない
およぐ plavat → およがない よむ číst → よまない
はなす mluvit → はなさない つくる udělat → つくらない
まつ čekat → またない かう koupit → かわない
死ぬ zemřít → しなない ある existovat → ない


3) Nepravidelná slovesa
する dělat → しない  くる přijít → こな


\subsection{Plain forma: Přítomnost}

Všechny slovesa, i-adjektiva, (Na-adj. + です) a (Noun +  です) mají svojí planou formu. 

1) Slovesa
Plain kladná forma je stejná jako slovníková. Záporná je stejná jako nai-forma

2) I-adjektiva
Planá, neminulá, pozitivní forma i-adjektiv je vytvořena přidáním い a planá, neminulá a negativní forma končí na くない.

3) Na adjektiva a podstatná jména
Planá, neminulá, pozitivní forma je vytvořena přidáním だ za na-adjektivum nebo podstatné jméno. Negativum je vytvořeno přidáním ではない/じゃない

\begin{center}
\begin{tabular}{|c|c|c|c|}
\hline
	 &ます/です&Planá pozitivní& planá negativní\\
	\hline
1だん slovesa &みます &みる &みない\\
\hline
5だん slovesa& かきます &かく &かかない\\
\hline
Nepravidelná slovesa&	します/きます& する/くる& しない/こない\\
\hline
I-adjektiva &	たかいです& たかい &たかくない\\
\hline
Na-adjektiva 	&きれいです& きれいだ &きれいではない/じゃない\\
\hline
Podstatná jména&	ほんです& ほんだ& 本ではない/じゃない\\
\hline
\end{tabular}	
\end{center}

\subsection{〜と おもう: Myslím že}

\begin{center}
\begin{tabular}{|c|c|c|}
\hline
父は四時ごろかえってくる& と& 思います。\\
\hline
Planá forma & to & omou\\
\hline
\end{tabular}
\end{center}
Myslím že táta dorazí okolo 4 hodin.


と je partikule dělající citaci a 〜とおもいます znamená "myslím že--" 〜ます 〜です formy nemohou být použity v citující části. Příslovce jako たぶん reprezentují značně jistý názor.

a) はさみは引き出しの中にあるとおもいます。Myslím že nůžky jsou v šuplíku.


私は není potřeba použít, pouze pokud bychom chtěli zdůraznit že "my" si to myslíme v kontrastu s někým jiným.


\subsection{〜ている: Výsledný stav}

〜ている zkombinováno se slovesy se směrovým pohybem, jako いく, くる, かえる, でかける indikuje přítomný stav který je výsledkem nějaké dokončené akce. Tedy いっている znamená "je pryč a není tady"  Výrazy いっている きている nemohou vyjádřit průběhový stav.


a) 母は大阪に住んでいますが、今東京に来ています。Matka žije v Ósace ale teď je v Kjótu.



















